\documentclass[10pt,a4paper,twocolumn,openany]{book}
%\documentclass[10pt,a4paper,openany]{book}

\usepackage[bg-letter]{lib/rpg-book} % Options: bg-a4, bg-letter, bg-full, bg-print, bg-none.
\usepackage[english]{babel}
%\usepackage[utf8]{inputenc}
\usepackage{lipsum}


% Start document
\begin{document}
\fontfamily{ppl}
\selectfont % Set text font
\frontmatter

\rpgMakeCover[
    image = img/cover,
    logo = img/logo,
    title = Rpg Template,
    subtitle = Created by Krozark the \today\\with XeTeX\\\url{https://github.com/Krozark/RPG-LaTeX-Template}
]


\tableofcontents

% Your content goes here
\mainmatter
\part{Part name}
\chapter{Chapter name}

\section{First Section}
\lipsum[2]

\subsection{a subsection}
\lipsum[1]
\subsubsection{a subsubsection}
\lipsum[1]


\section{Some boxes}
\begin{rpg-titlebox}{title}
	rpg-titlebox
\end{rpg-titlebox}

\begin{rpg-commentbox}
	rpg-commentbox
\end{rpg-commentbox}

\begin{rpg-warnbox}
	you can add some warnings using this box
\end{rpg-warnbox}

\begin{rpg-suggestionbox}
	rgp-suggestionbox
\end{rpg-suggestionbox}

\begin{rpg-quotebox}
    rpg-quotebox
\end{rpg-quotebox}

\begin{rpg-examplebox}
	rgp-examplebox
\end{rpg-examplebox}

%\newpage % Acts as columbreak because of twocolumn option; for pagebreak use \clearpage

\section{Some tables}
\header{default rpg-table (2 column)}
\begin{rpg-table}
   	\textbf{Table head 1}  & \textbf{Table head 2} \\
   	Some value  & Some value \\
   	Some value  & Some value \\
   	Some value  & Some value
\end{rpg-table}

% For more columns, you can say \begin{rpg-table}[your options here].
% For instance, if you wanted three columns, you could say
% \begin{rpg-table}[XXX]. The usual host of tabular parameters are
% aailable as well.
\header{rpg-table with more columns}
\begin{rpg-table}[XXX]
    \textbf{Table head 1}  & \textbf{Table head 2} & \textbf{Table head 3}\\
   	Some value  & Some value & Some value\\
   	Some value  & Some value & Some value\\
   	Some value  & Some value & Some value
\end{rpg-table}

\header{default rpg-table2 (2 column)}
\begin{rpg-table2}
   	\textbf{Table head 1}  & \textbf{Table head 2} \\
   	Some value  & Some value \\
   	Some value  & Some value \\
   	Some value  & Some value
\end{rpg-table2}

\header{rpg-longtable}
\tablehead{%
    \hline
    \textbf{first column} & \textbf{second column} & \textbf{third column} \\
    \hline
}
\begin{rpg-longtable}[ccc]
    1 & 2 & 3 \\ 1 & 2 & 3 \\ 1 & 2 & 3 \\ 1 & 2 & 3 \\
    1 & 2 & 3 \\ 1 & 2 & 3 \\ 1 & 2 & 3 \\ 1 & 2 & 3 \\
    1 & 2 & 3 \\ 1 & 2 & 3 \\ 1 & 2 & 3 \\ 1 & 2 & 3 \\
    1 & 2 & 3 \\ 1 & 2 & 3 \\ 1 & 2 & 3 \\ 1 & 2 & 3 \\
    1 & 2 & 3 \\ 1 & 2 & 3 \\ 1 & 2 & 3 \\ 1 & 2 & 3 \\
    1 & 2 & 3 \\ 1 & 2 & 3 \\ 1 & 2 & 3 \\ 1 & 2 & 3 \\
    1 & 2 & 3 \\ 1 & 2 & 3 \\ 1 & 2 & 3 \\ 1 & 2 & 3 \\
    1 & 2 & 3 \\ 1 & 2 & 3 \\ 1 & 2 & 3 \\ 1 & 2 & 3 \\
    1 & 2 & 3 \\ 1 & 2 & 3 \\ 1 & 2 & 3 \\ 1 & 2 & 3 \\
    1 & 2 & 3 \\ 1 & 2 & 3 \\ 1 & 2 & 3 \\ 1 & 2 & 3 \\
    1 & 2 & 3 \\ 1 & 2 & 3 \\ 1 & 2 & 3 \\ 1 & 2 & 3 \\
    1 & 2 & 3 \\ 1 & 2 & 3 \\ 1 & 2 & 3 \\ 1 & 2 & 3 \\
    1 & 2 & 3 \\ 1 & 2 & 3 \\ 1 & 2 & 3 \\ 1 & 2 & 3 \\
    1 & 2 & 3 \\ 1 & 2 & 3 \\ 1 & 2 & 3 \\ 1 & 2 & 3 \\
    1 & 2 & 3 \\ 1 & 2 & 3 \\ 1 & 2 & 3 \\ 1 & 2 & 3 \\
\end{rpg-longtable}

\section{List}
\begin{rpg-list}
    \item first list item
    \item second list item
\end{rpg-list}


\section{Monster}
% You can optionally not include the background by saying
%\begin{rpg-monsterboxnobg}{monsterboxnob}
\begin{monsterbox}{rpg-monsterbox}
	\textit{Small metasyntatic variable (golbinoid), neutral evil}\\
	\rpghline
	\basics[%
	armorclass = 12,
    hitpoints  = \rpgdice{3d8 + 3},
	speed      = 50 t
	]
	\rpghline%
	\stats[ % This stat command will autocomplete the modifier for you
    STR = 12, 
    DEX = 7
	]
	\rpghline%
	\details[%
	% If you want to use commas in these sections, enclose the
	% description in braces.
	% I'm so sorry.
	languages = {Common Lisp, Erlang},
	]
	\rpghline%
	\begin{rpg-monsteraction}[rpg-monsteraction]
		This Monster has some serious superpowers!
	\end{rpg-monsteraction}

	\rpgmonstersection{rpgmonstersection}
	\begin{rpg-monsteraction}[rpg-monsteraction]
		This one can generate tremendous amounts of text! Though only when it wants to.
	\end{rpg-monsteraction}

	\begin{rpg-monsteraction}[rpg-monsteraction]
    See, here he goes again! Yet more text.
	\end{rpg-monsteraction}
\end{monsterbox}


\section{Spell}
\begin{rpg-spell}
	{Beautiful Typesetting}
	{4th-level illusion}
	{1 action}
	{5 feet}
	{S, M (ink and parchment, which the spell consumes)}
	{Until dispelled}
	You are able to transform a written message of any length into a beautiful scroll. All creatures within range that can see the scroll must make a wisdom saving throw or be charmed by you until the spell ends.

	While the creature is charmed by you, they cannot take their eyes off the scroll and cannot willingly move away from the scroll. Also, the targets can make a wisdom saving throw at the end of each of their turns. On a success, they are no longer charmed.
\end{rpg-spell}

\section{Arts}

\rpgart{t}{img/art-top}
\rpgart{b}{img/art-bottom}

\lipsum
\lipsum

\chapter{Colors}

This package provides several global color variables to style \lstinline!rpg-commentbox!, \lstinline!rpg-quotebox!, \lstinline!rpg-examplebox!, and \lstinline!rpg-table! environments.


TODO

%\begin{rpgtable}[lX]
%  \textbf{Color}         & \textbf{Description} \\
%  \lstinline!commentboxcolor! & Controls \lstinline!commentbox! background. \\
%  \lstinline!paperboxcolor!   & Controls \lstinline!paperbox! background. \\
%  \lstinline!quoteboxcolor!   & Controls \lstinline!quotebox! background. \\
%  \lstinline!tablecolor!      & Controls background of even \lstinline!rpgtable! rows. \\
%\end{rpgtable}
%
%See Table~\ref{tab:colors} for a list of accent colors that match the core books.
%
%\begin{table*}
%  \begin{rpgtable}[XX]
%    \textbf{Color}                            & \textbf{Description} \\
%    \lstinline!commentgreen!                      & Light green used in PHB Part 1 \\
%    \lstinline!commentblue!                       & Light cyan used in PHB Part 2 \\
%    \lstinline!PhbMauve!                           & Pale purple used in PHB Part 3 \\
%    \lstinline!PhbTan!                             & Light brown used in PHB appendix \\
%    \lstinline!DmgLavender!                        & Pale purple used in DMG Part 1 \\
%    \lstinline!DmgCoral!                           & Orange-pink used in DMG Part 2 \\
%    \lstinline!DmgSlateGray! (\lstinline!DmgSlateGrey!) & Blue-gray used in PHB Part 3 \\
%    \lstinline!DmgLilac!                           & Purple-gray used in DMG appendix \\
%  \end{rpgtable}
%  \caption{Colors supported by this package}%
%  \label{tab:colors}
%\end{table*}
%
%\begin{itemize}
%  \item Use \lstinline!\setthemecolor[<color>]! to set \lstinline!themecolor!, \lstinline!commentcolor!, \lstinline!paperboxcolor!, and \lstinline!tablecolor! to a specific color.
%  \item Calling \lstinline!\setthemecolor! without an argument sets those colors to the current \lstinline!themecolor!.
%  \item \lstinline!commentbox!, \lstinline!rpgtable!, \lstinline!paperbox!, and \lstinline!quoteboxcolor! also accept an optional color argument to set the color for a single instance.
%\end{itemize}
%
%\subsection{Examples}
%
%\subsubsection{Using \lstinline!themecolor!}
%
%\begin{lstlisting}
%\setthemecolor[PhbMauve]
%
%\begin{paperbox}{Example}
%  \lipsum[2]
%\end{paperbox}
%
%\setthemecolor[PhbLightCyan]
%
%\header{Example}
%\begin{rpg-table}[cX]
%  \textbf{d8} & \textbf{Item} \\
%  1           & Small wooden button \\
%  2           & Red feather \\
%  3           & Human tooth \\
%  4           & Vial of green liquid \\
%  6           & Tasty biscuit \\
%  7           & Broken axe handle \\
%  8           & Tarnished silver locket \\
%\end{rpg-table}
%\end{lstlisting}
%
%\begingroup
%\setthemecolor[PhbMauve]
%
%\begin{paperbox}{Example}
%  \lipsum[2]
%\end{paperbox}
%
%\setthemecolor[PhbLightCyan]
%
%\header{Example}
%\begin{rpgtable}[cX]
%  \textbf{d8} & \textbf{Item} \\
%  1           & Small wooden button \\
%  2           & Red feather \\
%  3           & Human tooth \\
%  4           & Vial of green liquid \\
%  6           & Tasty biscuit \\
%  7           & Broken axe handle \\
%  8           & Tarnished silver locket \\
%\end{rpgtable}
%\endgroup
%
%\subsubsection{Using element color arguments}
%
%\begin{lstlisting}
%\begin{rpgtable}[cX][DmgCoral]
%  \textbf{d8} & \textbf{Item} \\
%  1           & Small wooden button \\
%  2           & Red feather \\
%  3           & Human tooth \\
%  4           & Vial of green liquid \\
%  6           & Tasty biscuit \\
%  7           & Broken axe handle \\
%  8           & Tarnished silver locket \\
%\end{rpgtable}
%\end{lstlisting}
%
%\begin{rpgtable}[cX][DmgCoral]
%  \textbf{d8} & \textbf{Item} \\
%  1           & Small wooden button \\
%  2           & Red feather \\
%  3           & Human tooth \\
%  4           & Vial of green liquid \\
%  6           & Tasty biscuit \\
%  7           & Broken axe handle \\
%  8           & Tarnished silver locket \\
%\end{rpgtable}



% End document
\end{document}
