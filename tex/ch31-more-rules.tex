\chapter{More Rules}
\label{ch:more-rules}

Game Masters typically need to know more rules than the players. They need to handle situations that go beyond a single character, like doing mass battles or travelling by ships, optional creature plunder, etc. This chapter consolidates such rules for the benefit of Game Masters.


\section{Plunder Rating}

Although Fantasy D100 is not a game of ‘Killing Things and Taking their Stuff’, it is sometimes useful and expected that creatures that the Player Characters meet upon their Quests will have treasure, both mundane and magical.

Normally the needs of the story can dictate what treasure and supernatural items a creature possesses, but if a quick random roll is necessary the following guidelines can be consulted.

Each creature has a ‘Plunder Rating’ which is a rating of how much treasure the creature is likely to be carrying. For creatures that form groups, increase the Plunder Rating by at least one, for groups of up to 20 creatures, by two for larger groups of up to a hundred creatures, and by 3 for groups of over a hundred. In this case the Plunder will be held in a defended and guarded treasure room which the leader of the creatures will have access to.

The following table has a list of potential treasure for creatures depending on their Plunder Rating. If applicable (e.g. campaign has supernatural Disciplines, like magic) also roll for the supernatural items.

\begin{table}
\begin{center}
\caption{Plunder}
\label{tab:plunder}
\begin{rpg-table}[|c|X|]
	\hline
        \textbf{Plunder Rating}  & \textbf{Treasure Found}\\
        \hline
	0 & Not a hoarder. No treasure whatsoever.\\
        1 & Chance hoarder. A couple of coppers, loose change (1D6 CP). Very remote (05\%) chance of a supernatural item, which is either used by accident (my lucky talisman) or which the creature is completely oblivious to.\\
	2 & Hoards enough for a rainy day. About 5D20 in SP, 1D10 GD. If the creature has supernatural abilities, there is a POW \% chance of 1D4 Minor supernatural items appropriate to the type.\\
	3 & Hoards for a better future. Collects treasure for its worth and appreciates its value. 5D100 in SP, 3D20 in GD. If the creature has supernatural abilities, there is a POW X 2\% chance of 1D4 appropriate Minor supernatural items.\\
	4 & Significant hoard. Hoards for hoarding’s sake. 10D100 SP, 1D100 GD. POW X 3\% of 1D6 Minor supernatural items and POW \% chance of 1D4 Major supernatural items.\\
	5 & Treasure trove. The wealth of a minor Lord. Examples: Grave goods of a dead noble worth about 1D6 thousand Silver Pieces, with 1D6 Minor supernatural Items and POW X 3\% chance of 1D6 Major supernatural Items.\\
	6 & Wealth of Kings. eg. Dragon’s Hoard, a hoard almost beyond comprehension 1D4 Million Silver pieces, 2D10 Minor, 1D8 Major and one unique supernatural item.\\
	\hline
\end{rpg-table}
\end{center}
\end{table}


\section{Hero Points for Plot Edits}
In Fantasy D100 it is usually the Games Master who describes the situation the player characters find themselves in and the outcome of any skill test. This optional rule allows the player to take control of the narrative and change the direction that the story is going in by spending Hero Points. 
\begin{description}
\item[1 point for a minor edit:] changes small details in favour of the player. For example, the player character suddenly has an important item of equipment that they previously forgot to bring with them on the quest, or the guard forgets to lock the door to the dungeon that the characters are imprisoned in.
\item[2 points for a major edit:] puts the player character at an advantage. For example, not only is the dungeon door open but the jail guard is asleep at his table.
\item[3 points for a drastic edit:] something dramatic and almost impossible happens to put the player character at a major advantage. For example, the king trips up on his flowing robes and, as he falls over, he brings down the three bodyguards who are standing close by, giving the player character assassin a clear bow shot at the tyrant.
\item[5 points for an implausible edit:] the player stretches the boundaries of plausibility (even within a fantasy setting), to advantage. For example, a passing dragon swoops down and attacks the castle the player characters are imprisoned in, allowing them to escape as the guards are busy fighting off the flying fire-breathing reptile.
\end{description}
Plot edits must always come with sensible narration from the player so that even with the five point implausible edit must not break the group’s suspension of disbelief. The Games Master has the final say on whether a plot edit is allowed or not.

Players should not rely on plot edits to constantly overcome obstacles, but save them for moments where they are truly stuck or have a cool situation in mind.

Plot edits may never completely remove obstacles such as opposing characters or imposing physical challenges, but they can be used to temporarily give players the upper hand. For example you can’t use a plot edit to remove a mountain range or instantly kill a major recurring villain, but you can use one to have your player character find an obscure mountain pass or have the villain temporarily knocked unconscious.


\section{Ships and Sailing}

\subsection{Construction}
There are, broadly speaking, three types of sailing ship: sloops (small, fast, but comparatively fragile one-masted vessels), brigs (fast and manoeuvrable two-masted vessels), and ships (larger vessels with at least three masts, whether warships or cargo vessels).

Weapons are handled abstractly; ship-mounted weapons are not accurate, and large numbers of shots are fired in order to have a chance to hit an enemy ship. Thus a ship’s weapons are rated abstractly as a single percentage chance to hit an enemy vessel in combat; almost certainly many weapons are fired for each “hit roll”. A hit generally does D8 damage, subtracted from the other ship’s structure points.

Every 10\% in weapons reduces cargo capacity by 2 tons, and means two extra crew are needed. The weapons level cannot be increased above 100\%.

Even beyond weapons carried, not all ships are identical; any ship will have one of the following special features.  It might have more than one such feature; in this case, add +50\% of the original cost to the total cost per feature added.

\begin{description}
\item[Armored:] AP 2 against any attacks.
\item[Fast:] Add +1 knot to speed.
\item[Heavy Weapons:] Hits from weapons do D12 rather then D8 damage.
\item[High Capacity:] Increase the cargo size by +40\%.
\item[Manoeuvrable:] Add +20\% to manoeuvrability.
\item[Marines:] The ship can carry (and provide board and lodging for) a number of marines equal to the size of its crew.
\item[Ram:] The ship can ram other vessels in combat without suffering damage.
\item[Reduced Crew:] The crew size needed to run the ship (as indicated table~\ref{tab:ship-types}) is halved.
\end{description}

\begin{table*}
\begin{center}
\caption{Ship Types}
\label{tab:ship-types}
\begin{rpg-table}[|l|Y|Y|c|Y|Y|c|Y|]
	\hline
	\textbf{Type of Ship}  & \textbf{Crew} & \textbf{Cost (SP)} & \textbf{Manoeuvrability} & \textbf{Speed} & \textbf{Structure Points} & \textbf{Cargo} \\
        \hline
	One-masted & 10 & 5000 & +20\% & 6 Knots & 20 & 8 Tons\\
	Two-masted & 20 & 15000 & - & 5 Knots & 40 & 15 Tons\\
	Three-masted & 30 & 50000 & -20\% & 4 Knots & 60 & 30 Tons\\
	\hline
\end{rpg-table}
\end{center}
\end{table*}


\subsection{Sailting Tests}
Most potential manoeuvres a vessel can make are governed by the captain’s Sailing skill, modified by the manoeuvrability of the vessel. A further modifier is the average Sailing skill level of the crew:
\begin{table}[H]
\begin{center}
\begin{rpg-table}[|X|Y|]
        \hline
	25\% or less (no idea) & -20\%\\
	26\%-50\% (competent) & -\\
	51\%-75\% (veteran) & +20\%\\
	76\% or more (expert) & +40\%\\
	\hline
\end{rpg-table}
\end{center}
\end{table}


\subsection{Travel}
In normal sailing conditions, a sailing vessel can move 20 miles per day per knot of speed. Speeds given are averages. Very  favourable  conditions – for example a  good  strong wind in the desired direction of travel (possibly magically arranged) – can double these speeds. As can a rowing crew who critical their Rowing skill test. 

On the other hand, if a ship is becalmed, with no wind at all, it cannot move.

Every day out of sight of land there is a 5\% chance of a storm. Storms do 2D6 structure points of damage to a ship; a ship reduced to zero structure points begins to sink (and will sink almost instantly if its structure points are reduced to the negative of the original amount). Further, sailors on deck must make Dodge tests to stay on board; a sailor swept overboard and not immediately rescued must make an Athletics test to survive.

Fortunately, the captain can make a Sailing test (modified by manoeuvrability) to halve damage from a storm. Better yet, it is possible to plot a course to avoid an incoming storm if it is detected in time (perhaps using magic or skills such as Natural Lore).

\subsection{Naval Combat}
We consider two ranges of distance between ships.

\subsubsection{Contact}
The vessels can see each other. If both vessels wish to close to combat range, or leave contact, this action is of course automatic, and takes about an hour.

If the vessels want different things, roll opposed Sailing tests, as above.

\subsubsection{Combat Range}
Combat between ships is similar to normal combat. Initiative is decided for each ship, rather than between individuals, by saying that the vessel with the better speed goes first.  

A single test is made to fire a ship’s weapons; no defence roll is made against these attacks. If desired, a character can be appointed weapons officer; he oversees the firing of a ship’s weapons.  That character should make a Ranged Combat skill test; if the test succeeds, the ship’s weapons test has a +20\% bonus.

Hand-held weapons are too small to have any effect on an opposing ship, but can be used against those on the decks. Fire is the exception to this rule, being used to set flamable objects, such as decks, and sails, on fire.

The following special manoeuvres can be made by a ship in combat range. One manoeuvre is allowed per round. Each manoeuvre needs a Sailing skill test by the captain, as indicated above.

\subsubsection{Broadside}
If the skill test succeeds, two attacks with a ship’s weapons can be made instead of one.


\subsubsection{Evade}
If the Sailing test succeeds, the opponent cannot use the broadside, ram, or boarding manoeuvres. Further, the vessel can escape combat range (out to contact range) if the other vessel allows it or the Sailing test succeeds as an opposed roll.


\subsubsection{Ram}
The other vessel is rammed if an opposed Sailing test succeeds. A ramming attack does D6 points of damage per mast. If the ship performing this manoeuvre lacks a battering ram, it also takes half the damage inflicted.


\subsubsection{Boarding}
Boarding is possible if an opposed Sailing test succeeds. In this case, the vessels are roped together, and boarding can commence. A free boarding test is allowed immediately after a successful ramming manoeuvre if desired.

If both vessels want to board the other, this is automatic.


\section{Major Mental Damage}
This optional rule is best used for Dark Fantasy games, where the tropes of Fantasy are blended with those of Horror. A setting where the characters are Vampire Hunters confronting the dark masters of the night and their minions is a good example of such a game. 

When characters witness horrific events, the Games Master will ask the players to make a Persistence test. Should that fail, then the character has suffered a mental blow so severe their persona has become altered by it. Roll on the Mental Damage Table below to see what kind of effect that the character suffers from. 

Independent of whether the Persistence test was successful, they must immediately make a Resilience test with a -40\% modifier, or go into shock for 1D4 rounds. 

\begin{table*}
\begin{center}
\caption{Mental Damage Table}
\label{tab:mental-damage}
\begin{rpg-table}[|l|X|]
	\hline
	\textbf{Roll 1D6}  & \textbf{Effect}\\
        \hline
	1 & The shakes – the incident has left you with an uncontrollable but slight and permanent jittery shake. Lose 2 DEX.\\
	2 & Dislocation – you find it hard to connect with people, it seems easier to remain unfeeling, to simply let things wash right over you. CHA is reduced by 2.\\
	3 & Losing your rag – suddenly everything and everyone around you is a constant sign of irritation. This irritation you find is best expressed through physical violence. Each time such a situation arises, you must make a Persistence test. Pass and you’ve controlled your rage, fail and you have no recourse but to lash out – either with your fists or any weapons you are carrying.\\
	4 & Bottling it – when finding yourself in dangerous and stressful situations you have an overwhelming urge to flee, to find safety. Each time such a situation arises, you must make a Persistence test. Pass and you’ve controlled your urge to run, fail and you’ve bottled it totally. \\
	5 & Nightmares – every night they invade your dreams, forcing you to relive over and over again the things you have witnessed. Sleep becomes almost impossible, a curse rather than a blessing.  CON is reduced by 2.\\
	6 & Focus – You find yourself having difficulty focusing on the task in hand. Just keeping aware is a struggle. Both POW and INT are reduced by 2.\\
	\hline
\end{rpg-table}
\end{center}
\end{table*}

\begin{rpg-examplebox}
Rurik is helping search a darkened defiled chapel after the party have defeated a group of Ghouls who had taken up residence there. He picks up some sacks that had been left in a corner of the room.  The humidity and heat have done their work on the contents of these bags, turning the contents inside largely to mush. As Rurik lifts the bags, he can feel something solid moving about in that fluid, and realises with a shock that it contains the half-rotten remains for his cousin's head. The Games Master rules that this is such a horrific realisation that Rurik must make a Persistence Roll.  

Rurik has 34\% Persistence. He rolls 51 – a fail. Rurik has been mentally scared by this revelation. He now rolls 1D6 on the Mental Damage Table. He rolls a 5 – from now on his sleep will be plagued by the memory of this instant.  

Rurik then rolls against his Resilience of 32\%.  He rolls 23 so doesn’t go into shock.  
\end{rpg-examplebox}

\subsection{Modifying the Persistence Roll}
Some monsters are by their very nature more horrific and disturbing than the standard Ghoul or Zombie, such as Greater Demons, Vampire Lords and indescribable monsters of Cosmic Horror. When encountering such leviathans of terror, whose very existence saps the blood from the character’s skin, the Games Master may apply a -20\% or even in very rare world threatening circumstances a -40\% modifier to the Persistence Roll.


\subsection{Spending Hero Points}
Just as in a Major Wound, characters can spend a Hero Point to avoid the mental damage they would otherwise have incurred. Instead, the character goes into Shock for 1D4 rounds.


\subsection{Fumbling and Criticals}
Should your character fumble the Persistence test, then not only do they receive a mental blow but they also go into shock for 1D8 rounds with no chance of a Resilience roll test to avoid.

Should a character get a critical Persistence, then they have simply shrugged off what has happened and carry on as normal, so a Resilience roll is not required. 


\subsection{Going into Shock}
The character becomes numb and unresponsive, can take no further action in combat situations, and can in certain circumstances become a sitting duck. 


\section{Mass Combat}
The following rules can be used if you want a one roll solution to a battle.

Mass combat involves the commanders’ skills of the two opposing sides, modified by the armies involved. The command skills involved are Lore (Military Tactics), and either Influence or Performance.

The actual battle is resolved by opposed Lore (Military Tactics) tests made by the leaders. A successful test means a force inflicts casualties equal to half its size on the opposition. Half of these casualties are deaths; the other half are injuries. These numbers are doubled on a critical success.

Further, the commander of a losing side in a battle must make an Influence or Performance test to prevent a rout. Routing troops either flee in panic, or surrender when they cannot flee. A further 10\% of an army’s numbers are lost in a rout. A critical success on the Influence or Performance test is needed for a force to continue fighting rather than retreating in a more orderly manner. If a force has nowhere to retreat to, it can continue to fight.

The Lore (Military Tactics) test is modified by various situational factors:
\begin{rpg-list}
\item Better equipped than enemy: +20\% bonus.
\item Better trained than enemy: +20\% bonus.
\item Has significant special forces (e.g. artillery, cavalry, combat mages) that the enemy lacks: +20\% bonus for each.
\item Outnumber enemy by two to one or more: +20\% bonus.
\item Outnumber enemy by four to one or more: +40\% bonus.
\item Enemy in defensive position: -20\% penalty.
\item Enemy fortifications: -40\% penalty.
\item Player character heroics (eg: taking out significant enemy, capturing strategic position): +20\% bonus, or -20\% penalty if attempted heroics go horribly wrong.
\end{rpg-list}

Larger forces might split into several armies, each with their own commander. The rules are still as above, but each army must pick another force to attack.
