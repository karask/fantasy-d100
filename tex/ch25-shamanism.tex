\chapter{Shamanism}
\label{ch:shamanism}

Shamanism is the belief that everything in the world has a spirit, which can be communicated with to gain knowledge and power and that these spirits have a direct effect on the world. They exist in a Spirit World which exists alongside, but invisible to, the normal world. A village might have a guardian spirit that affects the fertility of the villagers and their livestock and the bounty of their harvests. If pleased and honoured with offerings at a well kept shrine, lots of healthy children and animals are born and the fields yield bumper harvests. If displeased by inappropriate offerings, behaviour, or worst still neglect, then the local spirit can blight crops and make sure no children or animals are born.

It is, therefore, important that Shamans interact with this Spirit World, communicating with sympathetic spirits and driving off hostile spirits, on behalf of their tribe.  Disease and Pain spirits regularly have to be exorcised by the Shaman, while Magic and Healing spirits have to be contacted and encouraged to use their abilities for the benefit of the Shaman’s community. They are also responsible for caring and communicating with the spirits of dead ancestors who, if honoured regularly, help the community by lending their advice and magical abilities. The abilities and powers of these spirits are covered in page~\pageref{sec:spirits}.


\section{Spot Rules}

\subsection{Becoming a Shaman}
Usually a Shaman is chosen by the Spirit World and hears the call. During the period of change, where the character becomes attuned to the Spirit World, they might appear to have gone mad to their friends who are still rooted in the mundane world and can not see the character’s new friends in the Spirit World.  The character is usually then taken under the wing of an existing Shaman who teaches them the skills they will need in their new vocation.

In game terms, a character spends five Improvement Points and gains the skills of Shamanism at the base skill ranking. Becoming a Shaman is a big commitment and is usually not taken by characters during character generation, unless the Gamemaster allows it. The base chance for Shamanism is INT+POW.

This skill provides for a number of abilities. These abilities, although magical in origin, are always on or, in the case of Disassociate, can be instantly called upon. No Power Points are needed.

\begin{description}
\item[Improved Spirit Combat:] Shaman’s are more adept at dealing with the Spirit World and understand the weaknesses of their spiritual foes. A Shaman causes 1D6 Power Point loss each time they are successful in Spirit Combat, as opposed to the usual 1D4. In addition to this increase in damage a Shaman can use his Shamanism skill to conduct Spirit Combat if their skill is higher than the spirit’s Persistence.
\item[Spiritual Charism:] Shaman’s are better at controlling spirits, they can have a number of bound spirits equal to half their POW.
\item[Disassociate from Body:] the Shaman can put his body into a deep sleep, while his spirit travels the Spirit World. The two are connected by a slender silver cord, and if the body is destroyed the Shaman is effectively dead and his spirit is trapped in the Spirit World. If the Shaman is reduced to 1 or 0 Power Points, while in the Spirit World, his Spirit returns to his body immediately. In this disassociated form the Shaman can engage in Spirit Combat with an attack equal to his Shamanism score. During his time in the Spirit World, the Shaman has no physical body, therefore is considered STR, CON, DEX and SIZless. Any skills that are based upon those Characteristics or require a physical presence can not be used. The only way that a Shaman can interact with the physicial world is through casting spells or Spiritually attacking. While disassociated the Shaman is invisible to the physical world.
\item[See into the Spirit World:] The Shaman can always see what is happening in the Spirit World and therefore detect spirits that are invisible to non-Shamans.
\item[Assess harmony of the Spirit World:] This ability allows the Shaman to sense if something is wrong with the immediate Spirit World to a range of POW in kilometres. 
\item[Knowledge of the Spirit World:] The Shaman learns about the ‘geography’ of the Spirit World and its inhabitants.
\item[Awakening the Fetch:] When a character becomes a Shaman they may opt to undertake a ritual to awaken their Fetch. A Fetch is a spiritual guide, that is created from the Shaman’s own spirit and is as much as part of them as an arm or a leg. The Fetch aids the Shaman in their magic. This ritual cost an additional 2 Improvement Points. Not all Shamanic traditions have Fetches, See below for more details on the Shaman’s Fetch.  
\end{description}


\subsection{Spells}
Shamanism does not have a specific spell list. Typically they will have spells from the Magic discipline where they will commonly learn the following spells; Drive out Spirit, Spirit Bane, Spirit Shield and Call Spirit.


\subsection{Shaman's Fetch}
Many Shamans undergo spiritual quests or journeys, enduring great personal sacrifices and physical hardships in order to awaken their inner spirituality. At the end of their quest the Shaman awakens their Fetch, a magical guide that is like the Shaman’s shadow in the Spirit World. The powers of the Fetch are as follows.
\begin{rpg-list}
\item The Fetches INT, POW and CHA are equal to the Shaman’s at the time of its creation.
\item The Fetch is a discorporate Spirit, that can travel between the Spirit World and Mundane World to aid the Shaman, but under specific circumstances as explained below.
\item The Shaman and Fetch are in permanent telepathy, they can communicate instantly and cast spells on each other and have complete knowledge of each other’s magic. A Fetch can use its magic on the Shaman, even when the Shaman is unconscious, however it cannot use the Shaman’s magic at this time.
\item When a Shaman disassociates from his body his Fetch may enter the body in a benign possession, which lasts until the Shaman returns. The Fetch prevents the Shaman’s body being possessed by hostile spirits, it defends and heals the body with its magic, but it does not have the power to move the body.
\item The Fetch increases its POW when the Shaman’s increases.
\item The Fetch can be sent to attack opponents, be they corporeal or spirit, in Spirit Combat which it can engage in freely. A Fetch that defeats a spirit it can send it back to the Spirit Plane, or imprison it, like a hawk with a pigeon, for its POW in minutes, allowing the Shaman to perform a Spirit Binding Ritual upon it. A Fetch that defeats a corporeal being renders it unconscious or eats 1D6 POW as the Shaman’s chooses. If a Fetch is defeated in Spirit Combat it cannot be bound, but is sent to the Spirit Plane for 1D6 days and cannot be contacted by the Shaman.
\item The Fetch has a Spirit Combat skill which is equal to the Shamanism Skill of the Shaman and does 1D6 Spirit Damage.
\item The Fetch does not count towards the Shaman’s total of Bound Spirits.
\item A Fetch manifests in a manner that somehow represents the Shaman’s inner self, so a fierce and brave Shaman may have a Tiger like fetch and a monstrous Beastman Shaman may have a mutated clawed horror, with burning eyes.
\end{rpg-list}


\subsection{Spirit Combat}
\label{ssec:spirit-combat}
Not all spirits in the Spirit World will be friendly to the player characters. Some will be guardian spirits placed over treasure or locations to guard them against intruders who they will attack if they enter their area. Some will be hostile spirits, unleashed in combat by enemy Shamans. During his travels in the Spirit World the Shaman will encounter both malignant and beneficial spirits.

In cases of aggressive spirits Spirit Combat will occur. This is the clash of spiritual energies, each trying to overcome and dominate the other.

\subsubsection{Combat Process}
\begin{rpg-list}
\item This is a simultaneous opposed test. It runs like normal combat, however the characters bound in Spirit Combat cannot do anything else other than engage in Spirit Combat or cast magic to aid them in Spirit Combat. 
\item All Spirits have a skill representing their Spirit Combat ability, this could be a Spiritual Claw, Bite or similar. This is the skill that they make the opposed test with. This attack causes Spirit Damage, which unlike normal damage is taken from Power Points, not Hit Points.
\item Non-Shaman characters use their Persistence as their opposed skill in Spirit Combat. Shamans can use the Shamanism skill if it is higher than the Shamans Persistence. The base Spirit Damage is 1D4, however Shamans do 1D6.
\item Only Spirits or Shaman’s can initiate Spirit Combat, unless under the affect of certain magic. However anyone attacked by a Spirit can defend themselves. A discorporated Shaman or Fetch is able to initiate Spirit Combat with either a spirit or corporeal opponent, however while in this form they risk being affected by Spirit Binding.
\item The spirit combatant that loses the Spirit Combat Test loses Power Points equal to the Spirit Damage of their opponent. Some spells can reduce or increase this damage.
\item The more powerful a spirit is the more damage it does. For every ten points of POW above 20 a spirit combatant gains an extra +1D6 to the damage it can do in Spirit Combat.
\item If a combatant’s Power Points are reduced to 0 it loses the combat, they are not dead. The result of this loss can take several forms, dependent upon nature of the combatants.
\item If a spirit has been reduced to 0 Power Points and no other action is taken then it is forced back to the Spirit World and cannot return for a duration equal to the victor’s POW in days.
\item If the spirit’s opponent has cast a Spirit Binding and reduced the Spirit to 0 Power Points then it is bound to the service of the victor. 
\item If the loser had a corporeal body and lost to a Spirit then the Spirit may opt to possess the body of the loser (see Spirit Possession below).
\item If the spirit was possessing a body and was being subjected to a Spirit Combat as part of a Drive Out Spirit Magic, Exorcism Divine Spell or similar Spell and is being exorcised then it is forced out of the body and back to the Spirit Plane and it cannot return to possess the same victim again. The spirit of the possessed victim once again inhabits its body.
\item At any time during Spirit Combat one of the combatants wins the contest three times in a row they may break away from the contest if they succeed on a forth success round. The opponent may not engage in Spirit Combat again for the victors POW in minutes.
\end{rpg-list}


\subsection{Spirit Possession}
Possession is when a spirit steals or inhabits a corporeal body for its own ends. All forms of possessing spirits can be driven out by Exorcism, Drive out Spirits or similar magics. There are two types of possession:

\subsubsection{Covert}
This type of possession has the spirit hide in the victim’s body. It only assumes partial control of the victim when it needs to do something to the victim’s body. This is the form of possession typically used by Healing and Disease spirits. It is also the form of possession used by Ancestor Spirits.

\subsubsection{Dominant}
The spirit takes full control of the victim’s body and in turn the victim’s spirit is imprisoned in the body, unable to do anything until the hostile spirit is exorcised or leaves. This is the most dangerous form of possession, as the spirit often cares nothing for the body it inhabits, for once the body dies the spirit returns to its old existence in most cases, and possession by a suicidal ghost or homicidal demon is never good for the body in question.


