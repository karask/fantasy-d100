\chapter{Adventuring}
\label{ch:adventuring}


\section{Spot Rules}
This selection of rules is designed to deal with individual situations that may crop up throughout the game. 

\subsection{Travel}

The rates given below are based on average movement rates. If you need to precisely determine which of two groups reached a destination first, use an Opposed Athletics (for walking) or Riding test.
\begin{table}
\begin{center}
\caption{Daily Travel Rates}
\label{tab:daily-travel-rates}
\begin{rpg-table}[|l|c|X|]
        \hline
	\textbf{Type} & \textbf{Rate/day} & \textbf{Notes}\\
        \hline
	Hiking      & 50km   & Ten hours of steady walking on road or path with no wagons or animals. Need to make Fatigue Test at the end of the Hike to avoid becoming Fatigued.\\
	Marching    & 30km   & Marching in organised groups for ten hours, ready to fight at the end of the day. No need for a Fatigue test at the end of the march.\\
	Riding      & 30km   & Moving at a walk possibly accompanied by pack animals and wagons.\\
	Riding Fast & 90km   & Riding fast with no wagons. Both mount and rider need to make a Fatigue test at the end of the day.\\
        \hline
\end{rpg-table}
\end{center}
\end{table}

The above rates should also be modified by the type of terrain being crossed.

\begin{rpg-table}[|X|X|]
        \hline
	\textbf{Terrain} & \textbf{Effect on movement rate}\\
        \hline
	Road/Path                & 100\% of normal rate\\
	Light brush              & 80\% of normal rate\\
	Medium brush             & 70\% of normal rate\\
	Light woods              & 70\% of normal rate\\
	Rolling Hills            & 70\% of normal rate\\
	Heavy woodland           & 50\% of normal rate\\
        \hline
\end{rpg-table}


\subsection{Illumination \& Darkness}
Table~\ref{tab:illumination-and-darkness} gives examples of several different levels of illumination and darkness and the effects/penalties that it may have on the characters.
\begin{table*}
\begin{center}
\caption{Illumination \& Darkness}
\label{tab:illumination-and-darkness}
\begin{rpg-table}[|c|X|X|]
        \hline
	\textbf{Environtment} & \textbf{Example} & \textbf{Effects}\\
        \hline
	Brightly Illuminated & Blazing summer day  & None.\\
	Illuminated          & Heavy candlelit room, overcast day, with radius of illuminating item & None.\\
	Partial Darkness     & Cavern mouth, misty day, within 3 x radius of illuminating item (see below). & -20\% to vision-based Perception tests.\\
	Darkness             & Large cavern illuminated only by embers, foggy day, within 5 x radius of illuminating item. & -40\% to vision-based Perception tests. Movement rate halved.\\
	Pitch Black          & Sealed room with stone walls, cavern many miles underground, mountaintop whiteout. & Perception tests reliant on vision become near impossible, as are ranged attacks. Close combat attacks are at -80\%. Movement rate a quarter of normal.\\
        \hline
\end{rpg-table}
\end{center}
\end{table*}

\subsubsection{Dark Sight}
This allows the character to treat pitch black conditions as if dark. Normally possessed by subterranean or darkness aligned creatures.

\subsubsection{Night Sight}
This ability allows the character to treat partial darkness as illuminated and darkness as only partial darkness. This is normally possessed by nocturnal creatures.


\subsection{Fatigue}
Combat, sprinting, climbing, swimming against a strong current, are all examples of when a character can become fatigued and tired.

If the Games Master thinks that a character has been engaged in an activity that may have drained him of physical energy, then they may call for a Resilience roll. If the character fails the roll they suffer the effects of Fatigue.

\begin{rpg-examplebox}
Rurik has just been in a long, ten round, combat against a group of bandits. Although he has emerged victorious, the Games Master rules that Rurik has to roll a successful Resilience test or become Fatigued.
\end{rpg-examplebox}

This test is usually made after the activity has been completed, unless the activity is long and drawn out and there is a real danger that Fatigue will stop the task being completed successfully. For example, on a long hard march the characters are pressing on ahead so that they can reach a fort before an enemy army arrives there. In this case there is a real danger that they will arrive not only too late but tired and worn down.

\begin{table}
\begin{center}
\caption{Illuminating Items}
\label{tab:illuminating-items}
\begin{rpg-table}[|l|X|]
        \hline
	\textbf{Example} & \textbf{Radius}\\
        \hline
	Candle or embers      & 1m\\
	Torch or lantern      & 3m\\
	Campfire              & 5m\\
	Bonfire               & 10m\\
        \hline
\end{rpg-table}
\end{center}
\end{table}


\subsubsection{The Effects of Fatigue}
If a character fails the Resilience test then they become fatigued. All skill tests are at -20\%. Also movement rate drops by a quarter. The character also becomes sluggish, DEX and INT are each reduced by three points for the purposes of determining combat order.

If the fatigued character insists on engaging in heavy activity, such as combat, heavy labour or running, then another Resilience roll is made at -20\%. If the character fails this second skill test they become heavily fatigued and all the above penalties are doubled.

If a character fumbles any of their Resilience rolls, then they immediately fall unconscious for 3D6 minutes and upon waking are still fatigued.

\subsubsection{Recovering from Fatigue}
A character who completely rests for 20-CON hours will remove the effects of any Fatigue. Certain talents might also help remove the effects of Fatigue.


\subsection{Exposure, Starvation and Thirst}

