\chapter{Divine Magic}
\label{ch:divine}

This type of magic is gained through the worship of a god or goddess. Divine Magic spells come directly from the Deity and given to the character to use on their Deity’s behalf. 

The first step in learning Divine Magic is to join a religion that worships the Deity whose magic the character wants to learn.

\section{Religions}
Religions range in size from a handful of worshippers, meeting in secret to honour a dead hero of the revolution, to the millions of devotees of a world spanning sun god. There are temples where worshippers can learn Divine Magic directly from their Deity. They have rules and expectations of their worshippers and anyone found wanting is expelled from the comfort and support of the religion. 

\subsection{Religion Template}
Each religion is described using the following Religion format.
\begin{description}
\item[Name of God or Religion]
\item[Short description:] This short description briefly covers the religion’s mythology and its current place in the world.
\item[Type of Religion:] This is the type and size of religion. Great Deities are worshipped by millions and are at least acknowledged across the entire world. Major Deities are important in a specific region and have hundreds of thousands of worshipers. Minor Deities are usually the minor members of a religious pantheon appealing to a small group of specialist worshipers. Hero Religions worship dead heroes whose deeds and magic powers live on after their death.
\item[Worshippers:] The type of people who typically make up the religion membership.
\item[Worshipper Duties:] This is what the god and religion expect of its members. Break these rules and expect expulsion. On the other hand, follow these rules and promote them to others and the character will advance in the religion’s hierarchy.
\item[Religion skills:] These are skills favoured by the religion’s patron Deity and taught to its worshippers by its Priests.
\item[Religion spells:] Divine Magic that the god teaches.
\item[Special benefits:] Any bonuses to skill use or other special abilities or advantages that a worshipper gains by being a member of the religion.
\end{description}

Several examples can be seen in section~\ref{ssec:deities}.

\subsection{Worshipper Duties}
Each religion has a set of Worshipper Duties which represent the religion’s objectives in the world.

When a character does an action that fulfils one of the Worshipper Duties they gain one Improvement Point for a minor act and up to three points for a major act.

When a character does an action that goes against one of the Worshipper Duties they lose between one and three Improvement Points, depending on the grievousness of their transgression. If they have no Improvement Points left, then they start to lose Divine Magic spells learnt from the religion as a penance, on a one to one basis. The player may choose which spell to lose, but they must be ones that they have learnt from the religion.

\begin{rpg-examplebox}
Gerik the Pious acts in away that brings his god into disrepute and loses an Improvement Point. He has no Improvement Points to lose, since he has previously spent them on religion improvements, so he loses Shield 3 which he had previously learnt from the religion, which now becomes Shield 2.
\end{rpg-examplebox}

If the offending character has no Improvement Points or spells to lose, then they are excommunicated from the religion and may never join it again.

\subsection{Religion Ranks}
There are four ranks of membership: lay members, Initiates, Priests and Holy Warriors.

\subsubsection{Lay members}
Lay members are normal worshippers of the religion. They regularly attend the temple on holy days and do their best to uphold the strictures of the religion. In return, the religion protects them as best it can, and its Priests and Initiates cast Divine Magic on their behalf. Lay members cannot learn Divine Magic. To become a lay member of a religion a character must have Lore (Religion) of at least 20\%. 

\subsubsection{Initiates}
Initiates are worshippers who have dedicated their lives to the tenents of the religion. They always attend the temple on holy days and always uphold the strictures of the religion. In return, the religion will pay ransoms if they are captured and teach the Initiate Divine Magic. Initiates can learn up to 2 Magnitude of any Divine Magic spell available to the religion. To become an Initiate a member of a religion have a Lore (Religion) of at least 40\% and pay an Improvement Point cost of two points.

\subsubsection{Priests}
Priests are the living embodiment of their faith, instructed by their Deity to be its living representative in the mortal world. They lead the services for their temple on holy days. In return, the religion will pay ransoms if they are captured and teach them the inner secrets of their religion (this means all available Divine Magic at unlimited Magnitude). To become a Priest a character must have a Lore (Religion) and two of the cult skills at least 75\%, there must be a vacancy in the temple hierarchy, or the Priest be willing to become a missionary and establish a new temple. In addition the Player must pay five Improvement Points.

Upon becoming a Priest the character gains an Allied Spirit. This is a spirit associated with the Deity who is willing to work with one of their mortal worshipers to further the aims of the religion. The Allied Spirit is usually bound in either an animal or an item, sacred to the religion. If this item or animal is destroyed then the Allied Spirit returns to its home plane of existence. A Priest must go on a Quest of Repentance, which directly benefits his religion to gain a new Allied Spirit, since the Gods look dimly on Priests who lose their Allied Spirits.

An Allied Spirit starts with an INT of 2D6+6 and a POW of 3D6 and knows 3 points of Divine Magic known to the Religion. The spirit can see immaterial and invisible spirits, alerting its master to their presence in a twenty meter range. An Allied Spirit is in permanent Mind Link with its master, with a range equal to its POW x5 in meters. 

An Allied Spirit has whatever physical characteristics that its host animal or item has. Allied Spirits can be improved like player characters, by spending Improvement Points from their master’s total.


\subsubsection{Holy Warriors}
These are Holy Warriors who protect the temples and worshipers of their Deity. Not all Religions have Holy Warriors, especially those dedicated to peace, but where they do, these warriors ceaselessly crusade to protect the faithful and punish the Religion’s enemies. Like Priests they are expected to uphold the Worshipper duties unfailingly. Also, as the religion’s warriors, they are expected to take up arms against any aggressor who attacks its worshippers or the religion’s Temples.

These warriors are incredibly useful to the cult they belong to, which will always pay any ransom or make a rescue attempt when a Holy Warrior is captured. In addition they teach them any Divine Magic known to the religion.

The minimum requirement to become a Holy Warrior is to have Lore (Religion) of at least 50\% and a Weapon Skill of 75\% in the Religion’s holy weapon, usually the weapon that is most associated with the Deity that the Religion worships. In addition the Player must pay five Improvement Points.

When someone becomes a Holy Warrior they are gifted a specially consecrated weapon, that gives them a bonus when fighting to defend fellow worshippers, religion temples, and when attacking enemies of their faith. This bonus is usually +20\% to the appropriate weapon skill and double damage when fighting for their Religion. All damage done by such weapons is considered magical.

They also gain armour, which is magically blessed by the Religion’s Deity. Normally, this is at least double the normal AP of the armour type used, and it may have additional powers depending on the Deity.


\subsection{Player Character Priests}
Priests and Holy Warriors don’t just hang around their Temples doing their duties. They have plenty of Initiates and lay worshippers to do the more mundane administrative tasks, such as collecting tithes and feeding the poor. As player characters, their lives are more interesting and the source of constant Questing on behalf of their religion. Some of the Quests that they might get involved in are as follows:
\begin{rpg-list}
\item Going out and converting the unbelievers (or those who believe in the wrong Deity).
\item Actively fighting the enemies of the religion.
\item Recovering long-lost symbols and powerful artefacts of the faith. 
\item Attending a cross-faith to deal with all the politics and misunderstanding to come to a consensus about what to do about a common enemy.
\item Rushing to the aid of an embattled and besieged town of his faithful believers beset by enemies or some other form of spiritual peril.
\item Visiting the hinterlands to provide spiritual guidance and duties to those in need
\item Traveling to a distant Holy Mountain to commune directly with their Deity or otherwise performing idealistic inspirational acts, or to prove their worth.
\item Going on special mission, where success depends on Divine Magic.
\item Traveling as a special envoy of the Religion to show due deference to the King / Priest / High Emperor.
\end{rpg-list}


\subsection{Deity Examples}
\label{ssec:deities}
The following Religions are intended as examples or templates for your own creations or as simple pick up and play religions, that can be elaborated and detailed as a campaign progresses.

\subsubsection{The Night Mistress}
When the Sun Lord sleeps, the mistress of the Night stealthily creeps up from the Underworld to play.  Whether her games are harmful or beneficial depends on the person's view of the dark.

\begin{description}
\item[Type of Religion:] Great
\item[Worshippers:] Monsters of the Underdark, Thieves, Outcasts from society.
\item[Worshipper Duties:] Banish the light! Preserve the sanctity of Dark regions, prevent the forces of light invading the underworld. Remain mysterious and unfathomable.
\item[Religion skills:] Deception, Ranged Combat, Unarmed Combat
%\item[Magic:] Darkwall, Enhance Deception, Extinguish.
\item[Divine Magic:] All common spells, Call Shade, Fear.
\item[Special benefits:] +20\% to Deception during the Night.
\end{description}


\subsubsection{The Sun King}
The bright blazing ruler of the day. The everlasting source of life and light. To some cultures he is the Imperial Emperor, whose sacred word is to be obeyed without question. Also a source of healing and resurrection.

\begin{description}
\item[Type of Religion:] Great
\item[Worshippers:] Emperors, Charismatic Leaders
\item[Worshipper Duties:] Banish the dark, Guide the masses.
\item[Religion skills:] Healing, Close Combat, Ranged Combat, Influence.
%\item[Magic:] Enhance Influence, Fire Missile?, Fireblade?, Heal, Light, Multi Missile.
\item[Divine Magic:] All common spells, Call Salamander, Divine Heal, Resurrect, Sun Spear, Sun Disc, Radiant Appearance.
\item[Special benefits:] +20\% to Influence skills when dealing with lower social classes.
\end{description}



\subsubsection{The Sky Lord}
Arrogant and aloof, the Sky Lord brings storm and rain at the behest of his elder ruling brother the Sun King.  He strains at the unreasonable laws that bind him to his brother’s authority and is a constant rebel. In some lands he has cast off his brother’s chains and is acknowledged as the King of the Gods.

\begin{description}
\item[Type of Religion:] Great
\item[Worshippers:] Barbarians
\item[Worshipper Duties:] Ride the storm!, Fight against Tyrants, Stay free.
\item[Religion skills:] Close Combat, Natural Lore.
%\item[Magic:] Fanaticism, Extinguish, Vigour, Weapon Enhance.
\item[Divine Magic:] All common spells, Berserk, Call Sylph, Lightning Strike, Whirlwind, Enhance Machismo.
\item[Special benefits:] Suffers no penalty when doing skill tests in rainy or windy conditions.
\end{description}



\subsubsection{The Earth Mother}
The all embracing and loving Earth Mother is known throughout the world. Some people believe that she is the world itself. She is the source of all nature’s bounty, which clothes and feeds mankind, but also has a savage side that expresses itself in hurricanes, tidal waves and other natural disasters. 

\begin{description}
\item[Type of Religion:] Great
\item[Worshippers:] The religion is made up of people and creatures who live off the land. In civilised areas these are the peasants who farm the land and the woodsmen who hunt and gather in the forests. In the wilderness the Elves, Satyrs and Fey worship her. She is found wherever creatures have an acknowledged connection with nature.
\item[Worshipper Duties:] Respect the Earth. Don’t foul or pollute the environment. Practice the peaceful cut, a small prayer said in thanks to the animal spirit before killing it for food. The prayer ensures its return to the Earth Mother and through the cycle of rebirth into the world. 
\item[Religion skills:] Healing, Nature Lore, Resilience.
%\item[Magic:] Heal, Protection
\item[Divine Magic:] All common spells, Absorption, Berserk, Heal Body.
\item[Special benefits:] Any member of this cult gains a +20\% bonus to their Nature Lore, due to their connection to Nature, which they gain through their relationship with the Earth Mother.
\end{description}


\subsubsection{The Death Goddess}
Banished to the Underworld by the Sun King for a heinous crime, the Sun King’s sister is a twisted force that rails against the authority of her brother. Unable to leave the Underworld, she and her agents take that which is most precious to her brother, the very souls of his subjects, the living. She does this upon their physical death. Those judged unworthy by the Sun King, denied bliss in Eternal Golden Heaven, are taken by her cold embrace, into the Underworld. 

\begin{description}
\item[Type of Religion:] Great
\item[Worshippers:] The morbidly insane, Mercenaries, Graveyard attendants, Assassins.
\item[Worshipper Duties:] Respect the dead, Put down the Undead.
\item[Religion skills:] Close Combat, Undead Lore. 
%\item[Magic:] Demoralise, Spirit shield, Weapon Enhance.
\item[Divine Magic:] All common spells, Call (Undead), Death March, Resurrect, Touch of Death.
\item[Special benefits:] +20\% when inspecting corpse to determine time and cause of death.
\end{description}


\subsubsection{The Lord of Knowledge}
He is the Great Sage of Heaven, who exists only to drink in all the facts and information about the world. His book-loving followers emulate him, making a living by running Knowledge markets and selling advice and information.

\begin{description}
\item[Type of Religion:] Major
\item[Worshippers:] Explorers, Librarians, Scholars, Detectives.
\item[Worshipper Duties:] Find out new knowledge, catalogue and record information, maintain public libraries, punish knowledge thieves, remain unbiased and impartial.
\item[Religion skills:] Lores of various types, Influence, Languages.
%\item[Magic:] Detect X, Mindspeech.
\item[Divine Magic:] All common spells, Find X of various types, Soul Sight, See Past.
\item[Special benefits:] +20\% when using a Library to find information.
\end{description}



\subsubsection{The Trickster}
Culture hero or culture villain? This Deity aims to amuse him/herself by playing pranks on those who, in its opinion, deserve to be shamed before their peers. In some cultures the Trickster is revered as a Sacred Clown, who mocks authority when it is high-and-mighty and not working in the interest of the people. In others, he is outlaw, defying the Divine Right of the rulers to oppress the people.

\begin{description}
\item[Type of Religion:] Major
\item[Worshippers:] Thieves, Village idiots, non-conformists.
\item[Worshipper Duties:] Play pranks on the pompous and foolish.
\item[Religion skills:] Deception, Ranged Combat.
%\item[Magic:] Befuddle, Hinder.
\item[Divine Magic:] All common spells, Illusion, Reflection, Purity, Wax Effigy, Puppet, Jigsaw.
\item[Special benefits:] +20\% Deception when playing pranks.
\end{description}



\subsubsection{The Merchant}
He is a constant traveller, who gains joy by communicating with the new friends that he meets along the way. Long ago he learnt the art of commerce, as a way of making his contacts happy and a way of learning about the workings of the cultures he encounters. His followers know that a mule, a bag of shiny things, and a warm accepting smile, is all that is needed to open up a world of opportunity and reward.

\begin{description}
\item[Type of Religion:] Major
\item[Worshippers:] Merchants, Heralds, Traders, Shopkeepers.
\item[Worshipper Duties:] Spread the word to new places, Enrich both yourself and your temple. 
\item[Religion skills:] Influence, Perception, Ride, Drive, Languages.
%\item[Magic:] Clear the Path, Enhance Influence, Enhance Perception.
\item[Divine Magic:] All common spells, Treasury, Ward Camp, Create the Crystal Ship.
\item[Special benefits:] +20\% when using Perception or Influence as part of a financial deal.
\end{description}


\subsubsection{The Hearth Goddess}
This down to earth Deity is a daughter of the Earth Goddess who chose to live in the urban centres of mortals. She looks after the home. Her name is invoked to maintain domestic harmony and fertility.

\begin{description}
\item[Type of Religion:] Minor
\item[Worshippers:] House keepers and owners.
\item[Worshipper Duties:] Keep a clean and orderly home.
\item[Religion skills:] Craft, Influence.
%\item[Magic:] Heal, Enhance Craft.
\item[Divine Magic:] All common spells, Block Fertility, Enhance Fertility, Repair and Replace.
\item[Special benefits:] +20\% to any skill test in the character’s home.
\end{description}


\subsubsection{The Lord of War}
He is a blood-soaked Deity of violence and conflict. He is mentor and master to both the high-and-mighty General and the rank-and-file soldier. In civilized cultures he is worshipped through arcane ritual, where armies receive his blessing before battle. Amongst the Barbarians he is invoked through deed, in the fire of the battle itself.

\begin{description}
\item[Type of Religion:] Great
\item[Worshippers:] Generals, soldiers.
\item[Worshipper Duties:] Fight hard, Fight to win, Fight!
\item[Religion skills:] Close Combat, Dodge, Ranged Combat, Unarmed Combat.
%\item[Magic:] Coordination, Fanaticism, Strength, Vigour, Weapon Enhancement. 
\item[Divine Magic:] All common spells, Shield, Rout, True(Weapon), Unstoppable Charge, Ward Camp.
\item[Special benefits:] +20\% for any test when leading others into Combat.
\end{description}


\subsubsection{The Healer Goddess}
The White One will heal anyone, regardless of behaviour and allegiance. Her white-robed worshippers are found not only in settlements but also in the wagon-trains of armies. Healing gives even the most violent individual the chance to turn their life around and become a Warrior for Peace.

\begin{description}
\item[Type of Religion:] Major
\item[Worshippers:] Healers, Doctors
\item[Worshipper Duties:] Heal anyone regardless of outlook on life, maintain areas of sanctuary.
\item[Religion skills:] Healing, Perception, Influence.
%\item[Magic:] Heal
\item[Divine Magic:] All common spells, Divine Heal, Resurrect.
\item[Special benefits:] +20\% to all Healing attempts.
\end{description}


\subsubsection{The Sea Goddess}
She is the avaricious sister of the Earth Goddess, who is either the elder or younger sibling depending who you talk to. She constantly wars with her sister for surface area on the planet. In some places her tides eat the land and swallow remote islands. In others her waters relent and give back dry land previously sunk in Ancient times. Any sailor is wise to ask her permission before travelling across her watery realm.

\begin{description}
\item[Type of Religion:] Great
\item[Worshippers:] Sailors, fishermen, mermen, creatures of the sea.
\item[Worshipper Duties:] Respect the sea, ensure the goddess’ permission is sought by sailors. 
\item[Religion skills:] Sailing, Natural Lore
%\item[Magic:] Detect Treasures of the Sea, Water Breath.
\item[Divine Magic:] All common spells, Call Undine, Breathe Water.
\item[Special benefits:] +20\% to all sailing checks made at sea, and Athletics checks made to swim. 
\end{description}


\subsubsection{Chaos}
The writhing thing that is Chaos, strains and buckles on the boundaries of creation. Outside of the ordered universe, it howls to get in, and when it breaks through the cracks in reality it causes change and the warping of nature. 

\begin{description}
\item[Type of Religion:] Great
\item[Worshippers:] The insane, its foul monstrous spawn.
\item[Worshipper Duties:] Erode the very fabric of reality, destroy beauty, inflict pain on the living.
\item[Religion skills:] None.
%\item[Magic:] Befuddle, Countermagic, Demoralise, Disruption, Ignite.
\item[Divine Magic:] All common spells, Fear, Madness.
\item[Special benefits:] Immune to any kind of Mind Control magic.
\end{description}


\subsubsection{The Moon Hag}
There is an old woman who lives on the moon. She is the queen of the witches and all the spirits who live on the dark side of the moon. She is an enigma who will send you mad if you offend her sensibilities. 

\begin{description}
\item[Type of Religion:] Minor
\item[Worshippers:] Magicians, Astronomers.
\item[Worshipper Duties:] Observe the moon, maintain the mystery of the moon amongst non-religion members, study moon mysteries.
\item[Religion skills:] Natural Lore.
%\item[Magic:] Call (Magic, Spell spirit ), Counter magic.
\item[Divine Magic:] All common spells, Reflection, Mindblast, Mindlink, Madness.
\item[Special benefits:] +20\%  to any skill roll when dealing with Moon spirits.
\end{description}


\subsubsection{The Huntress}
The Divine Huntress stalks the land. In primitive and barbaric societies she is the patron of those who go into the wilderness to bring back essential meats to the tribe. In more civilised areas, she takes on the character of the supreme risk-taker, looking for more and more fabulous and exotic prey for glory and renown.

\begin{description}
\item[Type of Religion:] Minor
\item[Worshippers:] Hunters, Big Game Hunters.
\item[Worshipper Duties:] Be true to the hunt, do not deplete the hunting grounds, capture poachers.
\item[Religion skills:] Deception, Ranged Combat, Nature Lore.
%\item[Magic:] Multi Missile, Coordination, Clear Path, Speed Dart.
\item[Divine Magic:] All common spells, Sureshot, True (Bow or Spear).
\item[Special benefits:] +20\% Deception when stalking prey.
\end{description}



\subsection{Learning Divine Magic}
Divine Magic can be taught only to members of a religion with an appropriate Lore (Religion) skill and be of Initiate, Priest or Holy Warrior status. It costs six Improvement Points to get access to the Divine Magic discipline.


\subsection{Learning Spells}
A character with acess to the Divine Magic discipline can learn new spells by paying a cost in Improvement Points, equal to twice the Magnitude of the spell, to the Deity. This may be done in an incremental fashion, i.e. the player buys Shield 1 for two Improvement Points and then later increases this to Shield 3, by spending an additional four points. These points are not regained, even when the character leaves the religion.


\subsection{Casting Spells}
A character must be able to gesture with his hands and be able to chant in order to cast a spell. Whenever a spell is cast using Divine Magic, there will always be a sight and sound that nearby creatures can detect, be it a flash of light, a crack of thunder or a shimmering in the air. The exact effects, are up to the Games Master and Player to decide but will automatically be detected by any creatures within ten times the Magnitude of the spell, in metres. 

Casting a Divine Magic is automatically successful. No dice need be rolled, no chances of a fumble or critical either.

\subsubsection{Power Points}
Divine Magic does not cost any Power Points when it is cast.

\subsubsection{Casting Time}
Divine Magic spells always take only a single combat Action to cast and takes place on the INT order of the character casting the spell.

Distractions or attacks on the spellcaster as he casts will either automatically ruin the spell (if the spellcaster is blinded or disarmed, or suffers a Major Wound) or require Persistence tests for them to maintain concentration on the spell. 

\subsubsection{Cast Once Only}
Each Divine Magic spell may be cast only once, after which the character must return to a temple and pray or take part in a worshiping ceremony on the religion’s holy day to regain use of the spell. The Caster need not spend Improvement Points again. 

\subsubsection{Limitations}
Divine Magic spells do not stack, i.e. Shield 1 plus Shield 2 does not give the protection of a Shield 3 spell.

\subsubsection{Dismissing Spells}
A caster can dismiss any Permanent or Duration Divine Magic spell(s) he has cast as a single combat action. Ceasing to cast a Concentration spell is immediate and not a Combat Action.


\subsubsection{Splitting Magnitude}
Divine Magic allows the caster to ‘split’ a spell’s Magnitude into multiple spells. For instance, if the caster knows the Absorption spell at Magnitude 3, he may choose to cast it as a single Magnitude 3 spell, or he may split it into three Magnitude 1 Absorption spells, or one Magnitude 1 and one Magnitude 2 Absorption spell. The split spells are treated as separate instances and are cast separately.

\subsubsection{The Power of Divine Magic}
When in a direct contest with other forms of magic, Divine Magic is considered to have double its normal Magnitude.

\subsubsection{Common Divine Magic}
The following spells are listed as ‘All’, since all Cults teach them:

Consecrate, Create Blessed Item, Create Idol, Dismiss Magic, Divination, Excommunicate, Exorcism, Extension, Find X,  Mindlink, Soul Sight, Spirit Block, Spiritual Journey.


\subsection{Spell Traits}
The traits used by Divine Magic spells are detailed below. 

\begin{description}
	\item[Area (X):] The spell affects all targets within a radius specified in metres.
	\item[Concentration:] The spell’s effects will remain in place so long as the character concentrates on it. Concentrating on a spell is functionally identical to casting the spell, requiring the caster to continue to gesture with both arms, chant, and ignore distractions.
	\item[Duration (X):] The spell’s effects will stay in place for the number of minutes indicated. 
	\item[Instant:] The spell’s effects take place instantly. The spell itself then disappears. 
	\item[Magnitude (X):] The strength and power of the spell.
	\item[Permanent:] The spell’s effects remain in place until they are dispelled or dismissed.
	\item[Progressive:] This indicates that the spell can be learnt and cast at greater levels of Magnitude than the minimum.
	\item[Ranged:] Ranged spells may be cast upon targets up to a maximum distance of the character’s POW x 5 in metres.
	\item[Resist (Dodge/Persistence/Resilience):] The spell’s effects do not take effect automatically. The target may make a Dodge, Persistence or Resilience test (as specified by the spell) in order to avoid the effect of the spell entirely. Note that Resist (Dodge) spells require the target to be able to use Reactions in order to Dodge. In the case of Area spells, the Resist (Dodge) trait requires the target to dive in order to mitigate the spell’s effect. 
	\item[Touch:] Touch spells require the character to actually touch his target for the spell to take effect. The caster must remain in physical contact with the target for the entire casting.
	\item[Religion:] The type of religion that offers this spell to it’s worshippers. If the religion is listed as ‘All’, the spell is a Common spell available in all religions. The religion’s description will help determine which spells should or should not be available. 
\end{description}

\section{Spells}

\begin{samepage}
\begin{rpg-spell}
{Absorption}
{Duration 15, Magnitude 1, Progressive, Touch\\{[Religions: Earth, Night]}}

This spell absorbs incoming spells aimed at the target or his equipment, converting their magical energy into Power Points which are then available to the target. Once cast on a subject, Absorption will attempt to absorb the effects of any spells cast at the target. It will not have any effect on spells that are already affecting a character. The effects of Absorption depend on the relative Magnitude of both itself and the incoming spell. If the Absorption's Magnitude is equal or greater to the incoming spell then it is absorbed and Absorption remains; if not Absorption is eliminated and the incoming spell takes effect. Any spell absorbed by this spell is cancelled and has no effect. 

A character may not accumulate more Power Points than his POW while Absorption is in effect – excess Power Points garnered through Absorption simply vanish. Absorption is incompatible with Reflection, Shield and Spirit Block.
\end{rpg-spell}
\end{samepage}

\begin{samepage}
\begin{rpg-spell}
{Berserk}
{Duration 15, Magnitude 2, Touch\\{[Religions: Beasts, War]}}

The recipient of this spell is overcome with bloodlust, causing them to disregard their own safety and loyalties but imbuing them with tremendous stamina and combat ability. 

The recipient will automatically succeed any Resilience test for the duration of the spell. The recipient also automatically succeeds at any Fatigue tests and cannot be rendered unconscious. The Close Combat skills of the recipient receive a +40\% bonus for the spell’s duration. 

However, the subject may not Parry, Dodge or cast any magic spells while under the influence of Berserk. Normally, the recipient remains in the Berserk state for the entire 15 minute duration of the spell, but Games Masters may allow a Berserk character to shake off the effects with a Persistence test modified by -40\%. At the end of the spell, the recipient immediately becomes Fatigued. 

Berserk may not be combined with Fanaticism – Berserk will always take precedence in such cases. 
\end{rpg-spell}
\end{samepage}

\begin{samepage}
\begin{rpg-spell}
{Block Fertility}
{Magnitude 3, Permanent\\{[Religions: Earth]}}

While this spell is in place, the recipient is unable to conceive. This can be seen as a blessing or a curse depending on the view of the recipient. The spell can be dispelled by the caster whenever they want. Otherwise the effect of the spell is permanent. 
\end{rpg-spell}
\end{samepage}

\begin{samepage}
\begin{rpg-spell}
{Breath Water}
{Duration 15, Magnitude 2, Touch\\{[Religions: Sea, Water]}}

This spell allows an air-breathing creature to breathe water for the spell’s duration (the subject will still be able to breathe air as well). It may also be used upon a water-breathing creature to allow it to breathe air. 
\end{rpg-spell}
\end{samepage}

\begin{samepage}
\begin{rpg-spell}
{Call (Elemental)}
{Magnitude 1, Permanent, Progressive\\{[Religions: Any with affinity to elements]}}

This spell summons and binds to the service of the caster an elemental from another plane of existence, of a size dependant on the Magnitude of the spell:
1=Small, 2=Medium, 3=Large, 4=Huge (for more details on Elementals see page~\pageref{monster:elemental}). The elemental stays under the control of the Priest until it is killed or the Call spell is dispelled. 
	
To be successfully cast the spell requires an equal volume of the same material that the elemental is made up of. For example, a Large Undine (Water elemental) requires a pool of water of at least 50 m3 before it can be summoned.
\end{rpg-spell}
\end{samepage}

\begin{samepage}
\begin{rpg-spell}
{Call (Undead)}
{Magnitude 1, Permanent, Progressive\\{[Religions: Evil, Death]}}

This spell reanimates a dead human corpse and turns it into an undead creature, of a type determined by the Magnitude of the spell:
1=Skeleton, 2=Zombie, 3=Ghoul, 4=Vampire. The undead creature stays under the control of the Priest until it is killed or the Call spell is dispelled.
\end{rpg-spell}
\end{samepage}

\begin{samepage}
\begin{rpg-spell}
{Consecrate}
{Area Special, Magnitude 1, Permanent, Progressive\\{[Religions: All]}}

This spell is as much a part of a temple’s foundation as is its cornerstone, but may actually be cast almost anywhere. It creates a sphere with a radius of ten metres per point of Magnitude. The consecrated sphere is sacred to the caster’s god. Consecrate by itself does nothing to keep outsiders at bay, but the caster of the spell will know immediately if a spell, spirit or someone who is not a lay member of his cult crosses the boundaries of the Consecrate spell.
\end{rpg-spell}
\end{samepage}






AAAAAAAAAAAAAAAAAAAAAAAA

\section{Creating Magic Items}
Most Magic Items found in play in a game of Fantasy D100 will have been created by the characters or Non Player Character magicians.

Although the spells that their characters use to create Magic Items are detailed in their respective spell lists, its worth going through them briefly remind yourself what spell does what and how it is used.

\begin{description}
\item [Create Scrolls] This spell allows for the creation of magical writings. Scrolls can be used to transfer knowledge (spells) or directly to cast the written spell.
\item [Create Spell Matrix] Arcane Magic users can create items and store spells using this spell. Spells stored can then be used and manipulated using the weilder's Power Points.
\end{description}

Arcane Magic users also find very useful the Magic spell Create Power Point Store so that they can have a greater pool for manipulations.

\subsection{Identify a Magic Item}
There is no catch all “Detect Magical Properties” or “Know Magic item” skill in Fantsy D100. This is quite deliberate, keeping with the general policy that such items are not the equivalent of Magical shotguns.  Some options are:

Consult a Sage or other magical expert. This option will cost the characters lots of money. Take a baseline of one hundred silvers per point of spell magnitude OR some perilous quest that the character must do in return. Such experts are rare, because most high ranking Magicians have little time for magical research for others, and would be more interested in their own schemes. 

Detect Magic spells. This merely tells you the item is magical.  A critical casting may tell the caster how powerful the item is.

Trial and error. The character tries to find out the item’s use by experiment. Allow creative and imaginative plans to reveal partially what the item does.

Researching the myths and legends around the item.  This is the most certain way of finding out what a magic item does. Of course such myths may be obscure themselves, requiring a dangerous Quest to a long hidden repository of knowledge to find.

