\chapter{Divine Magic}
\label{ch:divine}

This type of magic is gained through the worship of a god or goddess. Divine Magic spells come directly from the Deity and given to the character to use on their Deity’s behalf. 

The first step in learning Divine Magic is to join a religion that worships the Deity whose magic the character wants to learn.

\section{Religions}
Religions range in size from a handful of worshippers, meeting in secret to honour a dead hero of the revolution, to the millions of devotees of a world spanning sun god. There are temples where worshippers can learn Divine Magic directly from their Deity. They have rules and expectations of their worshippers and anyone found wanting is expelled from the comfort and support of the religion. 

\subsection{Religion Template}
Each religion is described using the following Religion format.
\begin{description}
\item[Name of God or Religion]
\item[Short description:] This short description briefly covers the religion’s mythology and its current place in the world.
\item[Type of Religion:] This is the type and size of religion. Great Deities are worshipped by millions and are at least acknowledged across the entire world. Major Deities are important in a specific region and have hundreds of thousands of worshipers. Minor Deities are usually the minor members of a religious pantheon appealing to a small group of specialist worshipers. Hero Religions worship dead heroes whose deeds and magic powers live on after their death.
\item[Worshippers:] The type of people who typically make up the religion membership.
\item[Worshipper Duties:] This is what the god and religion expect of its members. Break these rules and expect expulsion. On the other hand, follow these rules and promote them to others and the character will advance in the religion’s hierarchy.
\item[Religion skills:] These are skills favoured by the religion’s patron Deity and taught to its worshippers by its Priests.
\item[Religion spells:] Divine Magic that the god teaches.
\item[Special benefits:] Any bonuses to skill use or other special abilities or advantages that a worshipper gains by being a member of the religion.
\end{description}

Several examples can be seen in section~\ref{ssec:deities}.

\subsection{Worshipper Duties}
Each religion has a set of Worshipper Duties which represent the religion’s objectives in the world.

When a character does an action that fulfils one of the Worshipper Duties they gain one Improvement Point for a minor act and up to three points for a major act.

When a character does an action that goes against one of the Worshipper Duties they lose between one and three Improvement Points, depending on the grievousness of their transgression. If they have no Improvement Points left, then they start to lose Divine Magic spells learnt from the religion as a penance, on a one to one basis. The player may choose which spell to lose, but they must be ones that they have learnt from the religion.

\begin{rpg-examplebox}
Gerik the Pious acts in away that brings his god into disrepute and loses an Improvement Point. He has no Improvement Points to lose, since he has previously spent them on religion improvements, so he loses Shield 3 which he had previously learnt from the religion, which now becomes Shield 2.
\end{rpg-examplebox}

If the offending character has no Improvement Points or spells to lose, then they are excommunicated from the religion and may never join it again.

\subsection{Religion Ranks}
There are four ranks of membership: lay members, Initiates, Priests and Holy Warriors.

\subsubsection{Lay members}
Lay members are normal worshippers of the religion. They regularly attend the temple on holy days and do their best to uphold the strictures of the religion. In return, the religion protects them as best it can, and its Priests and Initiates cast Divine Magic on their behalf. Lay members cannot learn Divine Magic. To become a lay member of a religion a character must have Lore (Religion) of at least 20\%. 

\subsubsection{Initiates}
Initiates are worshippers who have dedicated their lives to the tenents of the religion. They always attend the temple on holy days and always uphold the strictures of the religion. In return, the religion will pay ransoms if they are captured and teach the Initiate Divine Magic. Initiates can learn up to 2 Magnitude of any Divine Magic spell available to the religion. To become an Initiate a member of a religion have a Lore (Religion) of at least 40\% and pay an Improvement Point cost of two points.

\subsubsection{Priests}
Priests are the living embodiment of their faith, instructed by their Deity to be its living representative in the mortal world. They lead the services for their temple on holy days. In return, the religion will pay ransoms if they are captured and teach them the inner secrets of their religion (this means all available Divine Magic at unlimited Magnitude). To become a Priest a character must have a Lore (Religion) and two of the cult skills at least 75\%, there must be a vacancy in the temple hierarchy, or the Priest be willing to become a missionary and establish a new temple. In addition the Player must pay five Improvement Points.

Upon becoming a Priest the character gains an Allied Spirit. This is a spirit associated with the Deity who is willing to work with one of their mortal worshipers to further the aims of the religion. The Allied Spirit is usually bound in either an animal or an item, sacred to the religion. If this item or animal is destroyed then the Allied Spirit returns to its home plane of existence. A Priest must go on a Quest of Repentance, which directly benefits his religion to gain a new Allied Spirit, since the Gods look dimly on Priests who lose their Allied Spirits.

An Allied Spirit starts with an INT of 2D6+6 and a POW of 3D6 and knows 3 points of Divine Magic known to the Religion. The spirit can see immaterial and invisible spirits, alerting its master to their presence in a twenty meter range. An Allied Spirit is in permanent Mind Link with its master, with a range equal to its POW x5 in meters. 

An Allied Spirit has whatever physical characteristics that its host animal or item has. Allied Spirits can be improved like player characters, by spending Improvement Points from their master’s total.


\subsubsection{Holy Warriors}
These are Holy Warriors who protect the temples and worshipers of their Deity. Not all Religions have Holy Warriors, especially those dedicated to peace, but where they do, these warriors ceaselessly crusade to protect the faithful and punish the Religion’s enemies. Like Priests they are expected to uphold the Worshipper duties unfailingly. Also, as the religion’s warriors, they are expected to take up arms against any aggressor who attacks its worshippers or the religion’s Temples.

These warriors are incredibly useful to the cult they belong to, which will always pay any ransom or make a rescue attempt when a Holy Warrior is captured. In addition they teach them any Divine Magic known to the religion.

The minimum requirement to become a Holy Warrior is to have Lore (Religion) of at least 50\% and a Weapon Skill of 75\% in the Religion’s holy weapon, usually the weapon that is most associated with the Deity that the Religion worships. In addition the Player must pay five Improvement Points.

When someone becomes a Holy Warrior they are gifted a specially consecrated weapon, that gives them a bonus when fighting to defend fellow worshippers, religion temples, and when attacking enemies of their faith. This bonus is usually +20\% to the appropriate weapon skill and double damage when fighting for their Religion. All damage done by such weapons is considered magical.

They also gain armour, which is magically blessed by the Religion’s Deity. Normally, this is at least double the normal AP of the armour type used, and it may have additional powers depending on the Deity.


\subsection{Player Character Priests}
Priests and Holy Warriors don’t just hang around their Temples doing their duties. They have plenty of Initiates and lay worshippers to do the more mundane administrative tasks, such as collecting tithes and feeding the poor. As player characters, their lives are more interesting and the source of constant Questing on behalf of their religion. Some of the Quests that they might get involved in are as follows:
\begin{rpg-list}
\item Going out and converting the unbelievers (or those who believe in the wrong Deity).
\item Actively fighting the enemies of the religion.
\item Recovering long-lost symbols and powerful artefacts of the faith. 
\item Attending a cross-faith to deal with all the politics and misunderstanding to come to a consensus about what to do about a common enemy.
\item Rushing to the aid of an embattled and besieged town of his faithful believers beset by enemies or some other form of spiritual peril.
\item Visiting the hinterlands to provide spiritual guidance and duties to those in need
\item Traveling to a distant Holy Mountain to commune directly with their Deity or otherwise performing idealistic inspirational acts, or to prove their worth.
\item Going on special mission, where success depends on Divine Magic.
\item Traveling as a special envoy of the Religion to show due deference to the King / Priest / High Emperor.
\end{rpg-list}


\subsection{Deity Examples}
\label{ssec:deities}
The following Religions are intended as templates for either your own creations as simple pick up and play religions, that can be elaborated and detailed as a campaign progresses.

\subsubsection{The Night Mistress}
When the Sun Lord sleeps, the mistress of the Night stealthily creeps up from the Underworld to play.  Whether her games are harmful or beneficial depends on the person's view of the dark.

\begin{description}
\item[Type of Religion:] Great
\item[Worshippers:] Monsters of the Underdark, Thieves, Outcasts from society.
\item[Worshipper Duties:] Banish the light! Preserve the sanctity of Dark regions, prevent the forces of light invading the underworld. Remain mysterious and unfathomable.
\item[Religion skills:] Deception, Ranged Combat, Unarmed Combat
%\item[Magic:] Darkwall, Enhance Deception, Extinguish.
\item[Divine Magic:] All common spells, Call Shade, Fear.
\item[Special benefits:] +20\% to Deception during the Night.
\end{description}


\subsubsection{The Sun King}
The bright blazing ruler of the day. The everlasting source of life and light. To some cultures he is the Imperial Emperor, whose sacred word is to be obeyed without question. Also a source of healing and resurrection.

\begin{description}
\item[Type of Religion:] Great
\item[Worshippers:] Emperors, Charismatic Leaders
\item[Worshipper Duties:] Banish the dark, Guide the masses.
\item[Religion skills:] Healing, Close Combat, Ranged Combat, Influence.
%\item[Magic:] Enhance Influence, Firearrow, Fireblade, Heal, Light, Multimissile.
\item[Divine Magic:] All common spells, Call Salamander, Divine Heal, Resurrect, Sun Spear, Sun Disc, Radiant Appearance.
\item[Special benefits:] +20\% to Influence skills when dealing with lower social classes.
\end{description}



\subsubsection{The Sky Lord}
Arrogant and aloof, the Sky Lord brings storm and rain at the behest of his elder ruling brother the Sun King.  He strains at the unreasonable laws that bind him to his brother’s authority and is a constant rebel. In some lands he has cast off his brother’s chains and is acknowledged as the King of the Gods.

\begin{description}
\item[Type of Religion:] Great
\item[Worshippers:] Barbarians
\item[Worshipper Duties:] Ride the storm!, Fight against Tyrants, Stay free.
\item[Religion skills:] Close Combat, Natural Lore.
%\item[Magic:] Fanaticism, Extinguish, Vigour, Weapon Enhance.
\item[Divine Magic:] All common spells, Berserk, Call Sylph, Lightning Strike, Whirlwind, Enhance Machismo.
\item[Special benefits:] Suffers no penalty when doing skill tests in rainy or windy conditions.
\end{description}



\subsubsection{The Earth Mother}
The all embracing and loving Earth Mother is known throughout the world. Some people believe that she is the world itself. She is the source of all nature’s bounty, which clothes and feeds mankind, but also has a savage side that expresses itself in hurricanes, tidal waves and other natural disasters. 

\begin{description}
\item[Type of Religion:] Great
\item[Worshippers:] The religion is made up of people and creatures who live off the land. In civilised areas these are the peasants who farm the land and the woodsmen who hunt and gather in the forests. In the wilderness the Elves, Satyrs and Fey worship her. She is found wherever creatures have an acknowledged connection with nature.
\item[Worshipper Duties:] Respect the Earth. Don’t foul or pollute the environment. Practice the peaceful cut, a small prayer said in thanks to the animal spirit before killing it for food. The prayer ensures its return to the Earth Mother and through the cycle of rebirth into the world. 
\item[Religion skills:] Healing, Nature Lore, Resilience.
%\item[Magic:] Heal, Protection
\item[Divine Magic:] All common spells, Absorption, Berserk, Heal Body.
\item[Special benefits:] Any member of this cult gains a +20\% bonus to their Nature Lore, due to their connection to Nature, which they gain through their relationship with the Earth Mother.
\end{description}


\subsubsection{The Death Goddess}
Banished to the Underworld by the Sun King for a heinous crime, the Sun King’s sister is a twisted force that rails against the authority of her brother. Unable to leave the Underworld, she and her agents take that which is most precious to her brother, the very souls of his subjects, the living. She does this upon their physical death. Those judged unworthy by the Sun King, denied bliss in Eternal Golden Heaven, are taken by her cold embrace ,into the Underworld. 

\begin{description}
\item[Type of Religion:] Great
\item[Worshippers:] The morbidly insane, Mercenaries, Graveyard attendants, Assassins.
\item[Worshipper Duties:] Respect the dead, Put down the Undead.
\item[Religion skills:] Close Combat, Undead Lore. 
%\item[Magic:] Demoralise, Spirit shield, Weapon Enhance.
\item[Divine Magic:] All common spells, Call (Undead), Death March, Resurrect, Touch of Death.
\item[Special benefits:] +20\% when inspecting corpse to determine time and cause of death.
\end{description}


\subsubsection{The Lord of Knowledge}
\subsubsection{The Trickster}
\subsubsection{The Merchant}
\subsubsection{The Hearth Goddess}
\subsubsection{The Lord of War}
\subsubsection{The Healer Goddess}
\subsubsection{The Sea Goddess}
\subsubsection{Chaos}
\subsubsection{The Moon Hag}
\subsubsection{The Huntress}





\section{Spot Rules}

\subsection{Learning Sorcery}
Sorcery is governed by the Sorcery Casting magical skill. The base chance for Sorcery Casting is INT. Spells are learnt separately, but the Sorcery Casting skill determines the success for casting all Sorcery spells. Characters learn Sorcery from other characters who know the practice. It costs ten Improvement Points to get access to the Sorcery discipline. 

\subsection{Learning Spells}
Before a spell can be cast using Sorcery, the following process must be followed:
The character must first learn the spell through research. In order to learn a particular Sorcery spell, the caster must possess the spell in written form or be taught it by a teacher. In game terms this means having access to a teacher who knows the spell or a book or scroll where it is written down. The player then spends two Improvement Points and writes the spell down on their character sheet. Once the Sorcery spell has been learned, the character will be ready to try casting it.



\subsection{Casting Spells}
A character must be able to gesture with his hands, and be able to chant, in order to cast a spell. Whenever a spell is cast using Sorcery, there will always be a sight and sound that nearby creatures can detect, be it a flash of light, a crack of thunder, or a shimmering in the air. The exact effects are up to the Games Master and Player to decide , but will automatically be detected by any creatures within ten times the Magnitude of the spell in metres. 

Casting a Sorcery spell requires a successful skill test using the Sorcery Casting skill. If successful, the spell takes effect. If the casting test fails, the spell does not take effect. 

\subsubsection{Power Points}
All Sorcery spells cost a base of one Magic Point to cast. Sorcery spells can be modifying as the caster wishes (if he has the appropriate power points). If a Manipulation effect is applied to a spell, each effect costs one Power Point to apply (see below). 

The result of the Sorcery casting test depends on its success:
\begin{description}
	\item[Success:] A number of Power Points are deducted from the spellcaster’s total, equal to the Manipulation effects Power Points plus one of the spell. The spell then takes effect.
	\item[Failure:] The spell does not take effect and the character loses the spell's Power Points.
	\item[Critical:] Any attempt to resist or counter the spell suffers -20\% penalty. Moreover, only the base cost of one Power Point is lost (not for any Manipulations).
	\item[Fumble:] The spell fails and the Sorcerer loses 1D6 Power Points in addition to normal Power Point loss.
\end{description}


\subsubsection{Casting Time}
No other Combat Action may be taken while casting a spell, though the character may slowly walk up to half their Movement. 

A spell takes effect at the end of its casting, which starts at the beginning of the Combat Round and ends on the INT of the Caster in the Combat order. Note that while spellcasting, a character will draw possible attacks from enemies they are adjacent to during a Combat Round. 

Distractions or attacks on the spellcaster as he casts will either automatically ruin the spell (if the spellcaster is blinded or disarmed, or suffers a Major Wound) or require Persistence tests for them to maintain concentration on the spell. 

\subsubsection{Spell Manipulations}
Sorcery spells have three basic effects which can be manipulated by the caster: Magnitude, Duration, and Range.

Each effect has a default value which the spell can be cast at, costing one Power Point. The default value for the spell effects are listed in table~\ref{tab:sorcery-manipulations}.

\begin{table}
\begin{center}
\caption{Sorcery Manipulations}
\label{tab:sorcery-manipulations}
\begin{rpg-table}[|c|c|c|Y|]
        \hline
	\textbf{Power Points}  & \textbf{Magniture} & \textbf{Duration} & \textbf{Range}\\
        \hline
	1 (Default) & 1 & 5 minutes & 10m\\
	+1 & 2 & 15 minutes & 20m\\
	+2 & 3 & 30 minutes & 40m\\
	+3 & 4 & 1 hour & 80m\\
	+4 & 5 & 2 hours & 160m\\
	+5 & 6 & 4 hours & 320m\\
	+6 & 7 & 12 hours & 640m\\
	+7 & 8 & 1 day & 1km\\
	+8 & 9 & 2 day & 2km\\
	+9 & 10 & 5 day & 5km\\
	+10 & 11 & 1 week & 10km\\
	+11 & 12 & 2 weeks & 20km\\
	+12 & 13 & 1 month & 50km\\
	+13 & 14 & 2 months & 100km\\
	+14 & 15 & 3 months & 200km\\
	+15 & 16 & 6 months & 500km\\
	+16 & 17 & 1 year & 1000km\\
	+17 & 18 & 2 years & 2000km\\
	+18 & 19 & 5 years & 5000km\\
	+19 & 20 & 10 years & 10000km\\
	\hline
\end{rpg-table}
\end{center}
\end{table}

The tens value of the caster’s Sorcery Casting skill determines the max number of additional Power Points that can spend on each of the manipulation types. 

\begin{rpg-examplebox}
Omar the Magnificent with a Sorcery Casting skill of 80\% can spend an additional 8 Power Points on manipulating each of the spell’s effects, in Magnitude, Duration and Range. That’s a manipulation of up to 8 levels for each effect, not 8 levels in total across all three effects.
\end{rpg-examplebox}

The decision of which effects to manipulate and how many extra Magic Points are to be spent is made before the spell is cast.

\begin{rpg-examplebox}
Lura casts Damage Boosting on Rurik’s sword, and wants it to be at a magnitude of 4 for an hour.

She has a Sorcery Casting skill of 60\%, which means she can spend an additional six Power Points on manipulating any spell’s effects. Looking at the Manipulation table, Lura can comfortably manage a Magnitude of 4, for three additional Power Points, and can manage a duration of an hour with another three points. 

Lura’s player rolls the dice and compares the result against Lura’s casting skill of 60\% to see whether she successful casts the spell.

In fact Lura, can spend a maximum of six points on a magnitude of range 640m, another six on a duration of 12 hours and another 6 on a magnitude of 7, which is a total of 19 Magic Points (18 for the manipulations and 1 for the spell itself).
\end{rpg-examplebox}


\subsubsection{Dismissing Spells}
In a single Combat Round, a caster can dismiss any Permanent spell(s) he has cast, as a free action. Ceasing to cast a Concentration spell is immediate and not an action. 


\subsubsection{Extra Power Points}
As you can probably work out from the example above, it is possible for a Sorcerer to cast a spell which needs more Power Points in its manipulated form than a Sorcerer will normally have. Sorcerers get round this by carrying either Power Point stores (see Magic spell Create Magic Store) or other artifacts that can store Power Points (e.g. from other disciplines).


\subsection{Spell Traits}
All Sorcery spells share the same basic qualities:

\begin{rpg-list}
\item Base Magnitude of one 
\item Duration of 5 minutes, and
\item Range of 10 metres.
\end{rpg-list}

Other traits used by spells are detailed below. 
\begin{description}
	\item[Concentration:] The spell’s effects will remain in place as long as the character concentrates on it. Concentrating on a spell is functionally identical to casting the spell, requiring the spell caster to continue to gesture with both arms, chant and ignore distractions. This trait overrides the normal Sorcery spell default Duration. 
	\item[Instant:] The spell’s effects take place instantly. The spell itself then disappears. This trait overrides the normal Sorcery spell default Duration. 
	\item[Permanent:] The spell’s effects remain in place until they are dispelled or dismissed. This trait overrides the normal Sorcery spell default Duration.
	\item[Resist (Dodge/Persistence/Resilience):] The spell’s intended effects do not succeed automatically. The target may make a Dodge, Persistence or Resilience test (as specified by the spell) in order to avoid the effect of the spell entirely. Note that Resist (Dodge) spells require the target to be able to use Reactions in order to Dodge. In the case of Area spells, the Resist (Dodge) trait requires the target to dive in order to mitigate the spell’s effect. 
	\item[Touch:] Touch spells require the character to actually touch his target for the spell to take effect, using an Unarmed skill test to make contact. The caster must remain in physical contact with the target for the entire casting. TODO NEED TO CLARIFY THIS. This trait overrides the normal Sorcery spell default Range. 
\end{description}

\section{Spells}

\begin{samepage}
\begin{rpg-spell}
{Animate (Substance)}
{Concentration}

This spell allows the Sorcerer to animate the substance indicated, up to one SIZ for every point of Magnitude. The Sorcerer can cause it to move about and interact clumsily (Movement of 1m per three points of Magnitude). 

The Sorcerer’s chance to have the animated object perform any physical skill successfully is equal to his own chance to perform that action halved (before any modifiers). If the appropriate Form/Set spell is cast immediately after this spell, the caster can perform much finer manipulation of the object. In this case, the animated object will use the caster’s full skill scores for physical activities. 

This spell can only be used on inanimate matter. 
\end{rpg-spell}
\end{samepage}


\begin{samepage}
\begin{rpg-spell}
{Time Travel (Time Period)}
{Instant}

This spell transports the caster and a number of creatures (of SIZ 12-18) equal to the Magnitude of the spell to a named Time era via a Time Tunnel that opens up and instantly sucks them through to their destination. The Duration of the spell is the time that the caster and group jumps forward or backwards through time. 

Sorcerers usually have some knowledge about the time period they are travelling to, and use an Anchor, a landmark such as a bronze statue, that exists in both the original and destination time period. If they are travelling blind without such an Anchor, the casting roll is at -20\% and the effects of a fumbled roll are even more catastrophic than the examples below suggest. 

If the spell casting is failed, the caster and group still travels, but they end up in the wrong location (1D10 Km away from the Anchor point) and time (1D10 time units away, the length of the time unit depends on Duration, e.g. if the duration was in days, the time unit is days).

If the spell casting is fumbled catastrophic events occur. Here are some example events; the creative Games Master is encouraged to create more:
\begin{rpg-list}
\item A Guardian creature from an Other World emerges from the portal and attacks the Sorcerer, in an attempt to close the portal. 
\item The Sorcerer, and all within 10m of him, is sucked through the portal which then promptly closes. The Sorcerer is so befuddled that he cannot remember the spell for D20+D4 hours. 
\item As above, but the Sorcerer and party arrive in a completely different Time Era or even an Alternative Reality.
\end{rpg-list}

Sorcerers with this spell can “change” time freely without having to worry about unintentional “butterfly effect” changes, or any alterations in their own existence or memory from changing “their” past. However, too regular use is likely to lead to the catastrophic effects of a fumble.
\end{rpg-spell}
\end{samepage}



\section{Creating Magic Items}
Most Magic Items found in play in a game of Fantasy D100 will have been created by the characters or Non Player Character magicians.

Although the spells that their characters use to create Magic Items are detailed in their respective spell lists, its worth going through them briefly remind yourself what spell does what and how it is used.

\begin{description}
\item [Create Scrolls] This spell allows for the creation of magical writings. Scrolls can be used to transfer knowledge (spells) or directly to cast the written spell.
\item [Create Spell Matrix] Sorcerers can create items and store spells using this spell. Spells stored can then be used and manipulated using the weilder's Power Points.
\end{description}

Sorcerers also find very useful the Magic spell Create Power Point Store so that they can have a greater pool for manipulations.

\subsection{Identify a Magic Item}
There is no catch all “Detect Magical Properties” or “Know Magic item” skill in Fantsy D100. This is quite deliberate, keeping with the general policy that such items are not the equivalent of Magical shotguns.  Some options are:

Consult a Sage or other magical expert. This option will cost the characters lots of money. Take a baseline of one hundred silvers per point of spell magnitude OR some perilous quest that the character must do in return. Such experts are rare, because most high ranking Magicians have little time for magical research for others, and would be more interested in their own schemes. 

Detect Magic spells. This merely tells you the item is magical.  A critical casting may tell the caster how powerful the item is.

Trial and error. The character tries to find out the item’s use by experiment. Allow creative and imaginative plans to reveal partially what the item does.

Researching the myths and legends around the item.  This is the most certain way of finding out what a magic item does. Of course such myths may be obscure themselves, requiring a dangerous Quest to a long hidden repository of knowledge to find.

