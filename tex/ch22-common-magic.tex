\chapter{Common Magic}
\label{ch:common-magic}

Common magic is generally of a practical nature, meant to address the common ills of the community: healing the sick, bringing love or luck, driving away evil forces, finding lost items, bringing good harvests, granting fertility, reading omens and so on. 

%The most frequently encountered spells, making up the majority of the spell list, are those that relate to combat, hence the name given to this approach to magic. Despite the name, not all the spells in this group are directly applicable to physical combat, there are a fair few that will aid the wily charmer and golden tongued manipulator.

\section{Spot Rules}

\subsection{Learning Common Magic}
Common Magic Casting is treated as a skill. The base chance for Common Magic Casting is POWX3. Spells are learnt separately, but the Common Magic Casting skill determines the success for casting all Common Magic spells. Characters learn Common Magic from other characters who know the practice. It costs five Improvement Points to get access to the Common Magic discipline.

\subsection{Learning Spells}
Characters learn Common Magic spells from other characters who know the appropriate spells. Learning spells costs one Improvement Point per Magnitude point. If a character knows a spell at a lower Magnitude, they only have to pay the difference in Improvement Points to gain the spell at a higher Magnitude.

\begin{rpg-examplebox}
Adjin already knows Animal Whisperer at 2 Magnitude. He wants to learn Animal Whisperer 3, so he must spend only spend one Improvement Point to gain the spell at that Magnitude.
\end{rpg-examplebox}

Common Magic can be learnt from a number of sources. The most typical is from local folklore and tradition - families hand down spells and the local wise woman can teach healing spells to good members of the community. In some cases, it can be learn from local priests who teach Common Magic associated with their gods’ mythological exploits.

In each case the player character must be in good standing with the teacher before they will teach them the spell. If the teacher is indifferent to the player character to start with then they will first need to undertake some kind of service, which can be the focus of a Quest.

\subsection{Casting Spells}
A character must be able to move his hands to make gestures and be able to chant in order to cast a spell and must be able to see his target. 

When the character is casting a spell under duress, such as in the midst of combat, they must pass a Common Magic Casting test to successfully cast the spell. In this regard Common Magic is like any other skill. If the character is relaxed and has all the time in the world then no casting test is needed, the spell is automatically cast.

The result of the Common Magic casting test depends on its success:
\begin{description}
	\item[Success:] A number of Power Points are deducted from the spellcaster’s total, equal to the Magnitude of the spell. The spell then takes effect.
	\item[Failure:] The spell does not take effect and the character loses one Power Point.
	\item[Critical:] The caster has been able to control the flow of the magic particularly effectively. The character loses one Power Point instead of the normal cost of the spell.
	\item[Fumble:] The caster has been unable to control the flow of the Common Magic. Rather than losing a single Power Point for failing to cast the spell, the caster loses a number of Power Points equal to its Magnitude. 
\end{description}

\subsubsection{Casting Time}
No other action may be taken whilst casting a spell, though the character may slowly walk up to half their Movement while spell casting. All spells take one combat round to cast.

Casting begins at the start of the combat round and a spell’s effect happens on the caster’s INT, instead of DEX, (which is used for close combat).  

Distractions, or attacks on the caster as he casts, will automatically ruin the spell, unless the caster successfully passes a Persistence test, thereby maintaining concentration on the spell. Examples of distraction include blinding, disarming, or wounding the caster.

\subsubsection{Dismissing Spells}
In a single Combat Round, a caster can dismiss any Permanent spell(s) he has cast, as a free action. Ceasing to cast a Concentration spell is immediate and not an action. 


\subsection{Spell Traits}
Unless otherwise stated all Common Magic spells have the following traits.

\begin{rpg-list}
\item They have Variable Magnitude. This means that the Magnitude of the spell starts from the stated Magnitude and then can be cast at a higher Magnitude, if the caster knows it, giving an increase in the effect of the spell. The maximum Magnitude that a caster can learn is equal to their POW divided by 3.
\item Base Magnitude is one. 
\item Range is equal to the caster’s POWx3 in metres.
\item All spells, unless noted, have a Duration of ten minutes.
\end{rpg-list}

Other traits used by spells are detailed below. 
\begin{description}
	\item[Area (X):] The spell affects all targets within a radius specified in metres. 
	\item[Concentration:] The spell’s effects will remain in place so long as the character continues to concentrate on it. Concentrating on a spell is functionally identical to casting the spell, requiring the caster to continue to chant and ignore distractions. 
	\item[Instant:] The spell’s effects take place instantly. The spell itself then disappears. 
	\item[Magnitude (X):] The strength and power of the spell. Also the minimum number of Magic Points required to cast it. 
	\item[Non-Variable:] The spell may only be cast at the stated Magnitude.
	\item[Permanent:] The spell’s effects remain in place until they are dispelled or dismissed. 
	\item[Resist (Dodge/Persistence/Resilience):] The spell’s intended effects do not succeed automatically. The target may make a Dodge, Persistence or Resilience test (as specified by the spell) in order to avoid the effect of the spell entirely. Note that Resist (Dodge) spells require the target to be able to use Reactions in order to Dodge. In the case of Area spells, the Resist (Dodge) trait requires the target to dive in order to mitigate the spell’s effect. 
	\item[Touch:] Touch spells require the character to actually touch his target for the spell to take effect, using a Unarmed skill test to make contact. The caster must remain in physical contact with the target for the entire casting. 
	\item[Trigger:] The spell will lie dormant until an event stated in the description takes place. The spell then takes effect and is expended.
\end{description}

\section{Spells}

\begin{rpg-spell}
{Animal Whisperer}
{Magnitude 2, Non-Variable, Touch}

The caster whispers into the ear of a distressed animal, calming it. If the distressed animal is under the influence of a spell such as Fear or Scare, then its gets another Persistence test to shake off the effect of the spell.
\end{rpg-spell}


\begin{rpg-spell}
{Avoidance}
{Instant, Trigger}

This spell lies dormant until the recipient is attacked. Then, after the normal reaction of the recipient, it fires off allowing the recipient to Dodge a number of times equal to the spell’s Magnitude. Once triggered, all the points of the spell are fired off at once.
\end{rpg-spell}


\begin{rpg-spell}
{Babel}
{Magnitude 2, Non-Variable, Resist (Persistence)}

If this spell is successful, it garbles the language of the affected creature. The target can still think and, for the most part, act normally, but anything it says comes out as gibberish. Thus, a commanding officer would be unable to give orders to his men and a spellcaster would be unable to cast spells.
\end{rpg-spell}


\begin{rpg-spell}
{Bearing Witness}
{Instant}

This spell grants the caster a +10\% bonus per point of Magnitude to their next Skill Test they make to discover lies, secrets or hidden objects.  It does not stack with any other spell-effect bonuses.
\end{rpg-spell}


\begin{rpg-spell}
{Beast Call}
{Magnitude 2, Non-Variable, Instant, Resist (Resilience)}

The Beast Call serves to attract an animal within range. When the spell is cast, it affects a targeted creature with a fixed INT of 7 or less. If it fails to resist, the creature will be naturally drawn to the place where the spell is cast, whereupon the spell effect terminates. Any barrier, immediate threat, or counter control, also ends the effects of the spell, leaving the creature to react naturally. 

For example, the Beast Call spell might cause a horse to turn and walk towards the spell, but a single yank of its reins by the rider would end the spell’s effect. This spell is a potent aid to hunters and herders.
\end{rpg-spell}


\begin{rpg-spell}
{Befuddle}
{Magnitude 2, Non-Variable, Resist (Persistence)}

The affected target may not cast spells and may only take non-offensive actions. The target may run if it so chooses and may dodge and parry normally in combat, though it may not make any attacks unless it is attacked first. 

This spell is effective against humanoids and natural creatures. Other creatures (such as spirits or magical beasts like dragons) are not affected by this spell. 
\end{rpg-spell}

 
\begin{rpg-spell}
{Block Sense (Sense)}
{Magnitude 3, Non-Variable, Resist (Persistence)}

This spell will Blind, Deafen, Bland taste or Numb touch on a failed resistance roll for the duration of the spell.
\end{rpg-spell}


\begin{rpg-spell}
{Care}
{Magnitude 2, Non-Variable, Touch}

This charm places the recipient under the care of the caster. If the caster has any active Protection or Countermagic spells, the Cared for character also benefits from the effects of these spells.
\end{rpg-spell}

\begin{rpg-spell}
{Clear Path}
{Touch}

This spell allows the caster to move through even the most tangled, thorny brush, as if they were on an open road. For each additional point of Magnitude, they may bring one person with him. 
\end{rpg-spell}


\begin{rpg-spell}
{Coordination}
{Touch}

For every point of Magnitude, the target’s combat order increases by +2, whether casting spells or fighting and 10\% is added to Dodge or DEX-based Athletics tests. 
\end{rpg-spell}


\begin{rpg-spell}
{Counter-Attack}
{Magnitude 2, Non-Variable, Trigger}

This spell lies dormant until the recipient is attacked. Then, after the normal defensive reaction of the recipient, it fires off, allowing the recipient to follow up with a counter attack. The counter attack is an additional action, on top of the recipient’s normal attacking action.
\end{rpg-spell}


\begin{rpg-spell}
{Counter-Defense}
{Magnitude 2, Non-Variable, Trigger}

This spell lies dormant until the recipient is successfully attacked. Then after the normal reaction of the recipient, it fires off allowing the recipient an extra defence.
\end{rpg-spell}



\begin{rpg-spell}
{Countermagic}
{Instant}

Countermagic is only ever used as a Reaction, and only when another spell is cast within Countermagic’s Range that the character wishes to counter. A successful Countermagic disrupts the other spell and nullifies it. As long as Countermagic’s Magnitude equals or exceeds the target spell’s Magnitude, the target spell is countered.
\end{rpg-spell}


\begin{rpg-spell}
{Create Charms}
{Permanent}

A charm is a physical item that stores one or more Common Magic spells. A charm could be a necklace that holds a Befuddle 4 spell, a shield etched with runes that holds a Countermagic 2 spell, or even a sheet of paper with a poem written on it that, when held against the skin, provides a Protection 1 spell.

\begin{rpg-list}
\item To create a charm a character must possess both the spell they wish to store and Create Charm at the same Magnitude.
\item The item into which the charm is to be cast must be prepared and in contact with the caster for the length of the casting.
\item If the caster spends one Improvement Point at the time of creation the spell within the Charm is reusable. Otherwise once the spell is cast the Charm is dispelled.
\item A spell stored in a Charm is used like any other spell that the possessor knows. It uses the wielder’s Common Magic Casting skill and is powered by the wielder’s Power Points.
\item The time taken to create a single-use Charm is one hour per point of Magnitude of the spell being stored; Reusable Charms take three hours per point of Magnitude to create.
\item Charms are mundane items in their own right and if the item is broken the Charm is dispelled.
\end{rpg-list}
\end{rpg-spell}

\begin{rpg-spell}
{Create Potions}
{Permanent}

Potions are liquids that store one or more Common Magic spells. The Magnitude of the Create Potion spell needs to equal or exceed the highest Magnitude of the spell being stored into the potion.

\begin{rpg-list}
\item All potions are one use. They must be drunk in one swift gulp to work. 
\item The potion automatically works and doesn’t incur a cost in power points to the person who is drinking it. 
\item The potion costs the enchanter power points. They must know the spell at the Magnitude enchanting at, with the power points of the spell being put into the potion. 
\item There is an associated cost of 5 Gold Ducat per Magnitude. 
\item To make the potion, the enchanter must roll successfully against Common Magic Casting for each spell being placed in the potion and against Lore (Potion Making). If they fail the potion is ruined and they lose the cost of the ingredients. 
\item Potions take one hour per point of Magnitude of spell(s) stored to create. 
\item A potion must be stored in an air tight container, or it evaporates, losing one point of Magnitude per week. 
\end{rpg-list}
\end{rpg-spell}


\begin{rpg-spell}
{Cushion Fall}
{}

Each point of Magnitude of this spell eliminates one dice of falling damage for the recipient.
\end{rpg-spell}


\begin{rpg-spell}
{Darkwall}
{Area 5, Magnitude 2, Non-Variable}

Light sources within a Darkwall area shed no light and normal sight ceases to function. Other senses such as a bat’s sonar function normally. 
The caster may move the Darkwall 15 metres per Combat Round. If this option is chosen, the spell gains the Concentration trait. 
\end{rpg-spell}


\begin{rpg-spell}
{Demoralise}
{Magnitude 2, Non-Variable, Resist (Persistence)}

This spell creates doubt and uncertainty into the very heart and soul of the target. The target of this spell has all Weapon skills halved and may not cast offensive spells. If this spell takes effect before combat begins, the target will try to avoid fighting and will either run or surrender. The effects of this spell are automatically cancelled by the Fanaticism spell and vice versa. 
\end{rpg-spell}


\begin{rpg-spell}
{Detext X}
{Magnitude 1, Non-Variable, Concentration}

This covers a family of spells that all operate in a similar fashion, allowing the caster to locate the closest target of the spell within its range. This effect is stopped by a thick substance, such as metal, earth or stone, is at least one metre thick. It is also blocked by Countermagic, though the caster will know the target is somewhere within range (though not its precise location) and that it is being protected by Countermagic. The separate Detect spells are listed below and each must be learned separately.

\begin{rpg-list}
\item Detect Enemy: Gives the location of the nearest creatures, that intend to harm the caster. 
\item Detect Magic: Gives the location of the nearest magic item, magical creature or active spell. 
\item Detect Species: Each Detect Species spell will give the location of the nearest creature of the specified species. Examples of this spell include Detect Goblin, Detect Rhino and Detect Elf. 
\item Detect Substance: Each Detect Substance spell will give the location of the nearest substance of the specified type. Examples of this spell include Detect Coal, Detect Gold and Detect Wood. 
\end{rpg-list}
\end{rpg-spell}


\begin{rpg-spell}
{Dispel Magic}
{Instant}

This spell will attack and eliminate other spells. Dispel Magic will eliminate a combined Magnitude of spells equal to its own Magnitude, starting with the most powerful affecting the target. If it fails to eliminate any spell (because the spell’s Magnitude is too high), then its effects immediately end and no more spells will be eliminated. A spell cannot be partially eliminated, so a target under the effects of a spell whose Magnitude is higher than that of Dispel Magic will not have any spells currently affecting it eliminated. 
\end{rpg-spell}


\begin{rpg-spell}
{Disruption}
{Instant, Resist (Resilience)}

Disruption literally pulls a target’s body apart. The target will suffer 1D4 points of damage per point of Magnitude, ignoring any Armour Points. 
\end{rpg-spell}


\begin{rpg-spell}
{Dragon Fire}
{Magnitude 2, Non-Variable, Instant, Resist (Dodge)}

With this spell, the caster throws a stream of fire at his target. If the fire is not dodged, it inflicts 1D10 points of heat damage. Armour Points are effective against this damage and it counts as both magical and fire damage.
\end{rpg-spell}


\begin{rpg-spell}
{Dull Weapon}
{}

This spell can be cast on any weapon. For every point of Magnitude it reduces the damage dealt by the target weapon by one. 
\end{rpg-spell}


\begin{rpg-spell}
{Enhance Skill X}
{Instant}

Like Detect X, this includes a number of different spells, each of which affects a different non-combat skill. For each point of Magnitude, the recipient gains +10\% to any skill test using the Enhanced skill.  Alternatively, for each additional point of Magnitude of the spell, the caster can affect one more target. The bonuses and targets can be split as necessary, providing each bonus is in multiples of 10\% and the total of bonuses equals the spells Magnitude x 10\%.

For example, Adjin may have Enhance Skill(Deception) 5.  He could cast it all on himself to give a whopping +50\% to his Deception, or could cast it on himself and an ally, giving himself +30\% and his ally +20\%. If in a larger group, he could even cast it on 5 allies, each of whom would gains +10\% to their Deception skill.

The most common spells of this type are:
\begin{rpg-list}
\item Enhance Skill (Deception), often used by thieves.
\item Enhance Skill (Trade), used by merchants.
\item Enhance Skill (Influence), used by lawyers, con-artists and officers.
\item Enhance Skill (Resilience), used by warriors.
\item Enhance Skill (Persistance) used by magicians.
\end{rpg-list}

These spells are sometimes called by other names, such as “Cover of Night” or “Shadowstealth” (for Enhance Deception), “Golden Tongue” (for Enhance Influence or Trade), or “Toughen” (for Enhance Resilience).
\end{rpg-spell}


\begin{rpg-spell}
{Extinguish}
{Instant}

This spell instantly puts out fires. At Magnitude 1 it can extinguish a Flame, Magnitude 2 a Small Fire, Magnitude 3 a Large Fire and Magnitude 4 will put out an Inferno.
\end{rpg-spell}


\begin{rpg-spell}
{Total Awareness}
{Magnitude 2, Non-Variable}

This spell grants the recipient awareness as if they had physically got eyes in the back of their head for the duration of the spell. This allows them to make Perception rolls, and be aware of others behind them as they are with senses in front of them.
\end{rpg-spell}


\begin{rpg-spell}
{Fanaticism}
{Magnitude 2, Non-Variable}

The target of this spell will have close combat and unarmed combat skills increased by +20\% but may not attempt to parry, dodge or cast spells. Also for the duration of the spell the target has a +40\% bonus to any Persistence test related to Morale. The effects of this spell are automatically cancelled by the Demoralise spell and vice versa.
\end{rpg-spell}


\begin{rpg-spell}
{Fire Missile}
{Magnitude 2, Non-Variable, Touch, Trigger}

Casting this spell on a missile will cause it to burst into flame when it is fired/thrown and strikes a target. When it hits a target, the missile will deal +1D6 points of magical fire in addition to its normal damage. Since Fire Missile does magical damage, it affects creatures that are immune to normal damage. A missile under the effects of Fire Missile cannot benefit from Multi Missile or Speed Dart. 
\end{rpg-spell}


\begin{rpg-spell}
{Fire Weapon}
{Magnitude 4, Non-Variable, Touch}

For the duration of the spell, the target weapon will deal +1D6 points of magical fire damage in addition to its normal damage. A weapon under the effects of Fire Weapon cannot benefit from Weapon Enhance. Since Fire Weapon does magical damage, it damages creatures immune to normal damage.
\end{rpg-spell}


\begin{rpg-spell}
{Fist of Gold}
{Instant}

This spell creates a minor illusion of 5D10 Gold Ducats per level of Magnitude that persists for the duration of the spell.
\end{rpg-spell}


\begin{rpg-spell}
{Frostbite}
{Magnitude 2, Non-Variable}

This attack spell allows the caster to freeze his opponent, dealing 1D8 points of damage, ignoring any Armour Points. Magical damage that protect against cold damage can block this effect but mundane items (such as cold weather gear) are ineffective.
\end{rpg-spell}


\begin{rpg-spell}
{Glue}
{Area, Touch}

This spell covers an area of one centimetre square for each Magnitude with extremely sticky glue. If a creature steps on the glue, it must make an Athletics roll vs the Magnitude x 10\% to avoid being stuck for one round. On subsequent rounds it  must make the same roll to break free. This spell can also be used for more conventional repairs, a broken sword for example, with Magnitude x 10\% being the chance that the item won’t break again, if used in circumstances that would cause it to.
\end{rpg-spell}


