\chapter{Gamemaster Advice \& Tools}
\label{ch:advice-tools}

In Fantasy D100, monsters can be as richly detailed as the characters themselves. As well as characteristics they have skills, weapons and supernatural abilities. They are not mere cannon fodder to be killed and looted. They have their own motives that often bring them into conflict with the player characters, and if sentient can be used to create player characters.

This chapter contains several creatures grouped in lists according to their special characteristics; animals, spirits, undead, etc. It is encouraged (if not strongly recommended) that Game Masters create campaign-specific monsters using this section's material as examples.
\vspace{1em}


\section{Hero Points for Plot Edits}
In Fantasy D100 it is usually the Gamemaster who describes the situation the player characters find themselves in and the outcome of any skill test. This optional rule allows the player to take control of the narrative and change the direction that the story is going in by spending Hero Points. 
\begin{description}
\item[1 point for a minor edit:] changes small details in favour of the player. For example, the player character suddenly has an important item of equipment that they previously forgot to bring with them on the quest, or the guard forgets to lock the door to the dungeon that the characters are imprisoned in.
\item[2 points for a major edit:] puts the player character at an advantage. For example, not only is the dungeon door open but the jail guard is asleep at his table.
\item[3 points for a drastic edit:] something dramatic and almost impossible happens to put the player character at a major advantage. For example, the king trips up on his flowing robes and, as he falls over, he brings down the three bodyguards who are standing close by, giving the player character assassin a clear bow shot at the tyrant.
\item[5 points for an implausible edit:] the player stretches the boundaries of plausibility (even within a fantasy setting), to advantage. For example, a passing dragon swoops down and attacks the castle the player characters are imprisoned in, allowing them to escape as the guards are busy fighting off the flying fire-breathing reptile.
\end{description}
Plot edits must always come with sensible narration from the player so that even with the five point implausible edit must not break the group’s suspension of disbelief. The Gamemaster has the final say on whether a plot edit is allowed or not.

Players should not rely on plot edits to constantly overcome obstacles, but save them for moments where they are truly stuck or have a cool situation in mind.

Plot edits may never completely remove obstacles such as opposing characters or imposing physical challenges, but they can be used to temporarily give players the upper hand. For example you can’t use a plot edit to remove a mountain range or instantly kill a major recurring villain, but you can use one to have your player character find an obscure mountain pass or have the villain temporarily knocked unconscious.



Probably generic Gamemaster advice...

e.g. page 10 from OSR skills chapter

- how to create new magic spells (end of battle magic osd chapter) - is there something already in  ch22??

