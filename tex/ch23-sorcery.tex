\chapter{Sorcery}
\label{ch:sorcery}

Sorcery is an approach to magic that acknowledges that there are magical rules that govern the Universe and that by studying these rules a Magician can manipulate reality to his will.

Often Sorcery is atheistic, regarding gods and spirits as merely intelligent forces of the Universe, that exist to be interacted with and dealt with on an equal footing. 

Practitioners of Sorcery develop in one of two ways. The majority are organised into schools of wizardry, which have their own books of spells and rules that they teach their apprentices. Alternatively, there is a long tradition of practitioners working in solitude, cut off from other Sorcerers and society at large, to focus purely on their magical activities. Occasionally they take on an apprentice, to teach their art, or simply as a helping hand around the magical laboratory.


\section{Spot Rules}

\subsection{Learning Sorcery}
Sorcery is governed by the Sorcery Casting magical skill. This skill is automatically acquired at its basic score (INT) when the character is first created. This skill may be improved normally though the use of Improvement Points.  Even non-Sorcerors have this skill , at base, since it is used to give them a chance to use Sorcerous magic items and scrolls that store Sorcery spells. 

\subsection{Learning Spells}
Before a spell can be cast using Sorcery, the following process must be followed:
The character must first learn the spell through research. In order to learn a particular Sorcery spell, the caster must possess the spell in written form or be taught it by a teacher. In game terms this means having access to a teacher who knows the spell or a book or scroll where it is written down. The player then spends two Improvement Points and writes the spell down on their character sheet. Once the Sorcery spell has been learned, the character will be ready to try casting it.



\subsection{Casting Spells}
A character must be able to gesture with his hands, and be able to chant, in order to cast a spell. Whenever a spell is cast using Sorcery, there will always be a sight and sound that nearby creatures can detect, be it a flash of light, a crack of thunder, or a shimmering in the air. The exact effects are up to the Games Master and Player to decide , but will automatically be detected by any creatures within ten times the Magnitude of the spell in metres. 

Casting a Sorcery spell requires a successful skill test using the Sorcery Casting skill. If successful, the spell takes effect. If the casting test fails, the spell does not take effect. 

\subsubsection{Power Points}
All Sorcery spells cost a base of one Magic Point to cast. Sorcery spells can be modifying as the caster wishes (if he has the appropriate power points). If a Manipulation effect is applied to a spell, each effect costs one Power Point to apply (see below). 

The result of the Sorcery casting test depends on its success:
\begin{description}
	\item[Success:] A number of Power Points are deducted from the spellcaster’s total, equal to the Manipulation effects Power Points plus one of the spell. The spell then takes effect.
	\item[Failure:] The spell does not take effect and the character loses the spell's Power Points.
	\item[Critical:] Any attempt to resist or counter the spell suffers -20\% penalty. Moreover, only the base cost of one Power Point is lost (not for any Manipulations).
	\item[Fumble:] The spell fails and the Sorcerer loses 1D6 Power Points in addition to normal Power Point loss.
\end{description}


\subsubsection{Casting Time}
No other Combat Action may be taken while casting a spell, though the character may slowly walk up to half their Movement. 

A spell takes effect at the end of its casting, which starts at the beginning of the Combat Round and ends on the INT of the Caster in the Combat order. Note that while spellcasting, a character will draw possible attacks from enemies they are adjacent to during a Combat Round. 

Distractions or attacks on the spellcaster as he casts will either automatically ruin the spell (if the spellcaster is blinded or disarmed, or suffers a Major Wound) or require Persistence tests for them to maintain concentration on the spell. 

\subsubsection{Spell Manipulations}
Sorcery spells have three basic effects which can be manipulated by the caster: Magnitude, Duration, and Range.

Each effect has a default value which the spell can be cast at, costing one Power Point. The default value for the spell effects are listed in table~\ref{tab:sorcery-manipulations}.

\begin{table}
\begin{center}
\caption{Sorcery Manipulations}
\label{tab:sorcery-manipulations}
\begin{rpg-table}[|c|c|c|Y|]
        \hline
	\textbf{Power Points}  & \textbf{Magniture} & \textbf{Duration} & \textbf{Range}\\
        \hline
	1 (Default) & 1 & 5 minutes & 10m\\
	+1 & 2 & 15 minutes & 20m\\
	+2 & 3 & 30 minutes & 40m\\
	+3 & 4 & 1 hour & 80m\\
	+4 & 5 & 2 hours & 160m\\
	+5 & 6 & 4 hours & 320m\\
	+6 & 7 & 12 hours & 640m\\
	+7 & 8 & 1 day & 1km\\
	+8 & 9 & 2 day & 2km\\
	+9 & 10 & 5 day & 5km\\
	+10 & 11 & 1 week & 10km\\
	+11 & 12 & 2 weeks & 20km\\
	+12 & 13 & 1 month & 50km\\
	+13 & 14 & 2 months & 100km\\
	+14 & 15 & 3 months & 200km\\
	+15 & 16 & 6 months & 500km\\
	+16 & 17 & 1 year & 1000km\\
	+17 & 18 & 2 years & 2000km\\
	+18 & 19 & 5 years & 5000km\\
	+19 & 20 & 10 years & 10000km\\
	\hline
\end{rpg-table}
\end{center}
\end{table}

The tens value of the caster’s Sorcery Casting skill determines the max number of additional Power Points that can spend on each of the manipulation types. 

\begin{rpg-examplebox}
Omar the Magnificent with a Sorcery Casting skill of 80\% can spend an additional 8 Power Points on manipulating each of the spell’s effects, in Magnitude, Duration and Range. That’s a manipulation of up to 8 levels for each effect, not 8 levels in total across all three effects.
\end{rpg-examplebox}

The decision of which effects to manipulate and how many extra Magic Points are to be spent is made before the spell is cast.

\begin{rpg-examplebox}
Lura casts Damage Boosting on Rurik’s sword, and wants it to be at a magnitude of 4 for an hour.

She has a Sorcery Casting skill of 60\%, which means she can spend an additional six Power Points on manipulating any spell’s effects. Looking at the Manipulation table, Lura can comfortably manage a Magnitude of 4, for three additional Power Points, and can manage a duration of an hour with another three points. 

Lura’s player rolls the dice and compares the result against Lura’s casting skill of 60\% to see whether she successful casts the spell.

In fact Lura, can spend a maximum of six points on a magnitude of range 640m, another six on a duration of 12 hours and another 6 on a magnitude of 7, which is a total of 19 Magic Points (18 for the manipulations and 1 for the spell itself).
\end{rpg-examplebox}


\subsubsection{Dismissing Spells}
In a single Combat Round, a caster can dismiss any Permanent spell(s) he has cast, as a free action. Ceasing to cast a Concentration spell is immediate and not an action. 


\subsubsection{Extra Power Points}
As you can probably work out from the example above, it is possible for a Sorcerer to cast a spell which needs more Power Points in its manipulated form than a Sorcerer will normally have. Sorcerers get round this by carrying either Power Point stores (see Magic spell Create Magic Store) or other artifacts that can store Power Points (e.g. from other disciplines).


\subsection{Spell Traits}
All Sorcery spells share the same basic qualities:

\begin{rpg-list}
\item Base Magnitude of one 
\item Duration of 5 minutes, and
\item Range of 10 metres.
\end{rpg-list}

Other traits used by spells are detailed below. 
\begin{description}
	\item[Concentration:] The spell’s effects will remain in place as long as the character concentrates on it. Concentrating on a spell is functionally identical to casting the spell, requiring the spell caster to continue to gesture with both arms, chant and ignore distractions. This trait overrides the normal Sorcery spell default Duration. 
	\item[Instant:] The spell’s effects take place instantly. The spell itself then disappears. This trait overrides the normal Sorcery spell default Duration. 
	\item[Permanent:] The spell’s effects remain in place until they are dispelled or dismissed. This trait overrides the normal Sorcery spell default Duration.
	\item[Resist (Dodge/Persistence/Resilience):] The spell’s intended effects do not succeed automatically. The target may make a Dodge, Persistence or Resilience test (as specified by the spell) in order to avoid the effect of the spell entirely. Note that Resist (Dodge) spells require the target to be able to use Reactions in order to Dodge. In the case of Area spells, the Resist (Dodge) trait requires the target to dive in order to mitigate the spell’s effect. 
	\item[Touch:] Touch spells require the character to actually touch his target for the spell to take effect, using an Unarmed skill test to make contact. The caster must remain in physical contact with the target for the entire casting. TODO NEED TO CLARIFY THIS. This trait overrides the normal Sorcery spell default Range. 
\end{description}

\section{Spells}

\begin{samepage}
\begin{rpg-spell}
{Animate (Substance)}
{Concentration}

This spell allows the Sorcerer to animate the substance indicated, up to one SIZ for every point of Magnitude. The Sorcerer can cause it to move about and interact clumsily (Movement of 1m per three points of Magnitude). 

The Sorcerer’s chance to have the animated object perform any physical skill successfully is equal to his own chance to perform that action halved (before any modifiers). If the appropriate Form/Set spell is cast immediately after this spell, the caster can perform much finer manipulation of the object. In this case, the animated object will use the caster’s full skill scores for physical activities. 

This spell can only be used on inanimate matter. 
\end{rpg-spell}
\end{samepage}


\begin{samepage}
\begin{rpg-spell}
{Cast Back}
{}

This protective spell offers a chance of sending hostile spells (sorcery and magic) back to the attacking spell caster. 

Cast Back only affects spells that target the user specifically and have the Resist trait. Such spells may affect the protected character normally, but if it is resisted, the spell is launched back at the person who cast it, as long as its Magnitude is not greater than the Cast Back’s Magnitude. 
\end{rpg-spell}
\end{samepage}


\begin{samepage}
\begin{rpg-spell}
{Create Familiar}
{Permanent}

This spell allows the caster to make a personal sacrifice in order to create a Familiar. The Nature of the Familiar is dependent on the setting, but the spell typically only works on a non-sentient creature or inanimate object. The spell takes one hour to perform. The typical cost of ingredients for the spell is 50 Gold Ducats and it requires 3 Improvement Points. 

\begin{rpg-list}
\item A Sorcerer can have up to ½ their POW in Familiars, but few have more than one or two.
\item When a Familiar is destroyed or killed the Sorcerer loses 1D6 Hit Points from the trauma.
\item A Familiar gains sentience when created equal to 2D6 INT and Magic equal to 2D6 POW, unless the animal had a higher score, in which case that POW is used. 
\item A Familiar has a permanent mental connection with its master; with a range equal to 1/2 the caster’s POW in kilometres. Through this link the Familiar can allow the caster to ‘see’ through its perceptual abilities. The Sorcerer can cast spells on the Familiar as if touching it and the Familiar can cast any spells it knows on its master.
\item A Familiar can perceive its surroundings. How this happens depends on the type of Familiar. Animals can sense the world through their ordinary perceptions. Magical objects can detect the world around them up to a range equal to their POW in metres.
\item A Familiar dies when its master dies, and its life span is normal for the object or animal in which it is bound. It has Hit Points equal to the object or animal from which it was created.
\item Familiars may learn skills, but only ones they are capable of performing. Most Familiars in objects can only learn knowledge and magical skills. Animals have the skills that come naturally to them.
\item An animal Familiar can only be up to 1/2 the SIZ of the caster. 
\end{rpg-list}

\end{rpg-spell}
\end{samepage}


\begin{samepage}
\begin{rpg-spell}
{Create Scroll}
{Permanent}

These are written items which store Sorcery Spells.

All scrolls have an attached cost of 1 Gold Ducat per magnitude of spell in ingredients for special inks/parchments, etc.

The resulting scroll is a one use item, which upon a successful Sorcery Casting test, casts the spell(s) with any manipulations, at the magnitude that was cast on the scroll.

Alternatively, upon a successful Sorcery Casting the reader of the scroll can learn the spell by spending the appropriate number of Improvement Points.

Either way, upon a successful use of the scroll, the spell fades from the scroll. If the casting roll merely fails the spell remains, but the reader can not attempt to use the scroll until their Sorcery Casting skill increases. If the casting roll is fumbled the spell fades from the scroll, without any benefit to the reader.
\end{rpg-spell}
\end{samepage}


\begin{samepage}
\begin{rpg-spell}
{Create Spell Matrix}
{Permanent}

This spell creates items that store Sorcery spells. 
The enchanter pays 1 Improvement Point per spell stored in the matrix.

All spell matrices have an attached cost of 10 Gold Ducats per spell in special materials.

The wielder can cast and manipulate the spell at the skill of the original enchanter, using their own Power Points as fuel.

Spell matrices are mundane items in their own right and if the item is broken, then the spell is dispelled. However at the time of enchantment the enchanter can spend another Improvement Point  to magically harden the item doubling its Hit Points and Armor Points.
\end{rpg-spell}
\end{samepage}

\begin{samepage}
\begin{rpg-spell}
{Damage Boosting}
{Touch}

This spell can be cast upon any weapon up to five ENC. Each point of Magnitude adds one point to the weapon’s damage (the basic spell will increase a hatchet from 1D6 damage to 1D6+1 damage, for instance). 
\end{rpg-spell}
\end{samepage}


\begin{samepage}
\begin{rpg-spell}
{Damage Resistance}
{Touch}

This spell protects the body of the recipient. Any incoming attack dealing damage equal to or less than the Magnitude of the spell is ignored. Any incoming attack dealing more damage than the Magnitude of Damage Resistance is unaffected and will deal its full damage as normal. Note that the protected character may still suffer from Knockback if applicable. 
\end{rpg-spell}
\end{samepage}




\iffalse
\begin{rpg-spell}
{Avoidance}
{Instant, Trigger}

This spell lies dormant until the recipient is attacked. Then, after the normal reaction of the recipient, it fires off allowing the recipient to Dodge a number of times equal to the spell’s Magnitude. Once triggered, all the points of the spell are fired off at once.
\end{rpg-spell}


\begin{rpg-spell}
{Babel}
{Magnitude 2, Non-Variable, Resist (Persistence)}

If this spell is successful, it garbles the language of the affected creature. The target can still think and, for the most part, act normally, but anything it says comes out as gibberish. Thus, a commanding officer would be unable to give orders to his men and a spellcaster would be unable to cast spells.
\end{rpg-spell}


\begin{rpg-spell}
{Bearing Witness}
{Instant}

This spell grants the caster a +10\% bonus per point of Magnitude to their next Skill Test they make to discover lies, secrets or hidden objects.  It does not stack with any other spell-effect bonuses.
\end{rpg-spell}


\begin{rpg-spell}
{Beast Call}
{Magnitude 2, Non-Variable, Instant, Resist (Resilience)}

The Beast Call serves to attract an animal within range. When the spell is cast, it affects a targeted creature with a fixed INT of 7 or less. If it fails to resist, the creature will be naturally drawn to the place where the spell is cast, whereupon the spell effect terminates. Any barrier, immediate threat, or counter control, also ends the effects of the spell, leaving the creature to react naturally. 

For example, the Beast Call spell might cause a horse to turn and walk towards the spell, but a single yank of its reins by the rider would end the spell’s effect. This spell is a potent aid to hunters and herders.
\end{rpg-spell}


\begin{rpg-spell}
{Befuddle}
{Magnitude 2, Non-Variable, Resist (Persistence)}

The affected target may not cast spells and may only take non-offensive actions. The target may run if it so chooses and may dodge and parry normally in combat, though it may not make any attacks unless it is attacked first. 

This spell is effective against humanoids and natural creatures. Other creatures (such as spirits or magical beasts like dragons) are not affected by this spell. 
\end{rpg-spell}

 
\begin{rpg-spell}
{Block Sense (Sense)}
{Magnitude 3, Non-Variable, Resist (Persistence)}

This spell will Blind, Deafen, Bland taste or Numb touch on a failed resistance roll for the duration of the spell.
\end{rpg-spell}


\begin{rpg-spell}
{Care}
{Magnitude 2, Non-Variable, Touch}

This charm places the recipient under the care of the caster. If the caster has any active Protection or Countermagic spells, the Cared for character also benefits from the effects of these spells.
\end{rpg-spell}

\begin{rpg-spell}
{Clear Path}
{Touch}

This spell allows the caster to move through even the most tangled, thorny brush, as if they were on an open road. For each additional point of Magnitude, they may bring one person with him. 
\end{rpg-spell}


\begin{rpg-spell}
{Coordination}
{Touch}

For every point of Magnitude, the target’s combat order increases by +2, whether casting spells or fighting and 10\% is added to Dodge or DEX-based Athletics tests. 
\end{rpg-spell}


\begin{rpg-spell}
{Counter-Attack}
{Magnitude 2, Non-Variable, Trigger}

This spell lies dormant until the recipient is attacked. Then, after the normal defensive reaction of the recipient, it fires off, allowing the recipient to follow up with a counter attack. The counter attack is an additional action, on top of the recipient’s normal attacking action.
\end{rpg-spell}


\begin{rpg-spell}
{Counter-Defense}
{Magnitude 2, Non-Variable, Trigger}

This spell lies dormant until the recipient is successfully attacked. Then after the normal reaction of the recipient, it fires off allowing the recipient an extra defence.
\end{rpg-spell}



\begin{rpg-spell}
{Countermagic}
{Instant}

Countermagic is only ever used as a Reaction, and only when another spell is cast within Countermagic’s Range that the character wishes to counter. A successful Countermagic disrupts the other spell and nullifies it. As long as Countermagic’s Magnitude equals or exceeds the target spell’s Magnitude, the target spell is countered.
\end{rpg-spell}


\begin{rpg-spell}
{Create Charms}
{Permanent}

A charm is a physical item that stores one or more Magic spells. A charm could be a necklace that holds a Befuddle 4 spell, a shield etched with runes that holds a Countermagic 2 spell, or even a sheet of paper with a poem written on it that, when held against the skin, provides a Protection 1 spell.

\begin{rpg-list}
\item To create a charm a character must possess both the spell they wish to store and Create Charm at the same Magnitude.
\item The item into which the charm is to be cast must be prepared and in contact with the caster for the length of the casting.
\item If the caster spends one Improvement Point at the time of creation the spell within the Charm is reusable. Otherwise once the spell is cast the Charm is dispelled.
\item A spell stored in a Charm is used like any other spell that the possessor knows. It uses the wielder’s Magic Casting skill and is powered by the wielder’s Power Points.
\item The time taken to create a single-use Charm is one hour per point of Magnitude of the spell being stored; Reusable Charms take three hours per point of Magnitude to create.
\item Charms are mundane items in their own right and if the item is broken the Charm is dispelled.
\end{rpg-list}
\end{rpg-spell}


\begin{rpg-spell}
{Create Power Point Store}
{Permanent}

This spell allows the caster to create an item which has Power Point storing capabilities. This allows the owner to have a pool of Power Points in addition to their own.

Typically crystals are used, due to their physical toughness, in game terms treat them as unbreakable. This also applies to charms, such as a sword with Weapon Enhancement 2 stored in it, to provide a pool of power points to cast the spell from.

Power Point stores take one hour per power point stored in them to create. For each Magnitude, one power point can be stored.

Unless one improvement point is spent when they are created they are non-reusable. Once the Power Points are used the item loses its ability to store power points. If the improvement point is spent the item then becomes reusable. Once all the Power Points are used, the item can be refilled instantly from the user’s own Power Points.

The caster must fill the item with their own Power Points as part of the spell. The amount of Power Points put into the item at the time of casting becomes the maximum that can be put into the item. This maximum can not be increased after the spell is cast.

If the item is destroyed the power points are released harmlessly into the surrounding area.
\end{rpg-spell}


\begin{rpg-spell}
{Create Potions}
{Permanent}

Potions are liquids that store one or more Magic spells. The Magnitude of the Create Potion spell needs to equal or exceed the highest Magnitude of the spell being stored into the potion.

\begin{rpg-list}
\item All potions are one use. They must be drunk in one swift gulp to work. 
\item The potion automatically works and doesn’t incur a cost in power points to the person who is drinking it. 
\item The potion costs the enchanter power points. They must know the spell at the Magnitude enchanting at, with the power points of the spell being put into the potion. 
\item There is an associated cost of 5 Gold Ducat per Magnitude. 
\item To make the potion, the enchanter must roll successfully against Magic Casting for each spell being placed in the potion and against Lore (Potion Making). If they fail the potion is ruined and they lose the cost of the ingredients. 
\item Potions take one hour per point of Magnitude of spell(s) stored to create. 
\item A potion must be stored in an air tight container, or it evaporates, losing one point of Magnitude per week. 
\end{rpg-list}
\end{rpg-spell}


\begin{rpg-spell}
{Cushion Fall}
{}

Each point of Magnitude of this spell eliminates one dice of falling damage for the recipient.
\end{rpg-spell}


\begin{rpg-spell}
{Darkwall}
{Area 5, Magnitude 2, Non-Variable}

Light sources within a Darkwall area shed no light and normal sight ceases to function. Other senses such as a bat’s sonar function normally. 
The caster may move the Darkwall 15 metres per Combat Round. If this option is chosen, the spell gains the Concentration trait. 
\end{rpg-spell}


\begin{rpg-spell}
{Demoralise}
{Magnitude 2, Non-Variable, Resist (Persistence)}

This spell creates doubt and uncertainty into the very heart and soul of the target. The target of this spell has all Weapon skills halved and may not cast offensive spells. If this spell takes effect before combat begins, the target will try to avoid fighting and will either run or surrender. The effects of this spell are automatically cancelled by the Fanaticism spell and vice versa. 
\end{rpg-spell}


\begin{rpg-spell}
{Detext (X)}
{Magnitude 1, Non-Variable, Concentration}

This covers a family of spells that all operate in a similar fashion, allowing the caster to locate the closest target of the spell within its range. This effect is stopped by a thick substance, such as metal, earth or stone, is at least one metre thick. It is also blocked by Countermagic, though the caster will know the target is somewhere within range (though not its precise location) and that it is being protected by Countermagic. The separate Detect spells are listed below and each must be learned separately.

\begin{rpg-list}
\item Detect Enemy: Gives the location of the nearest creatures, that intend to harm the caster. 
\item Detect Magic: Gives the location of the nearest magic item, magical creature or active spell. 
\item Detect Species: Each Detect Species spell will give the location of the nearest creature of the specified species. Examples of this spell include Detect Goblin, Detect Rhino and Detect Elf. 
\item Detect Substance: Each Detect Substance spell will give the location of the nearest substance of the specified type. Examples of this spell include Detect Coal, Detect Gold and Detect Wood. 
\end{rpg-list}
\end{rpg-spell}


\begin{rpg-spell}
{Dispel Magic}
{Instant}

This spell will attack and eliminate other spells. Dispel Magic will eliminate a combined Magnitude of spells equal to its own Magnitude, starting with the most powerful affecting the target. If it fails to eliminate any spell (because the spell’s Magnitude is too high), then its effects immediately end and no more spells will be eliminated. A spell cannot be partially eliminated, so a target under the effects of a spell whose Magnitude is higher than that of Dispel Magic will not have any spells currently affecting it eliminated. 
\end{rpg-spell}


\begin{rpg-spell}
{Disruption}
{Instant, Resist (Resilience)}

Disruption literally pulls a target’s body apart. The target will suffer 1D4 points of damage per point of Magnitude, ignoring any Armour Points. 
\end{rpg-spell}


\begin{rpg-spell}
{Dragon Fire}
{Magnitude 2, Non-Variable, Instant, Resist (Dodge)}

With this spell, the caster throws a stream of fire at his target. If the fire is not dodged, it inflicts 1D10 points of heat damage. Armour Points are effective against this damage and it counts as both magical and fire damage.
\end{rpg-spell}


\begin{rpg-spell}
{Dull Weapon}
{}

This spell can be cast on any weapon. For every point of Magnitude it reduces the damage dealt by the target weapon by one. 
\end{rpg-spell}


\begin{rpg-spell}
{Enhance Skill (X)}
{Instant}

Like Detect (X), this includes a number of different spells, each of which affects a different non-combat skill. For each point of Magnitude, the recipient gains +10\% to any skill test using the Enhanced skill.  Alternatively, for each additional point of Magnitude of the spell, the caster can affect one more target. The bonuses and targets can be split as necessary, providing each bonus is in multiples of 10\% and the total of bonuses equals the spells Magnitude x 10\%.

For example, Adjin may have Enhance Skill(Deception) 5.  He could cast it all on himself to give a whopping +50\% to his Deception, or could cast it on himself and an ally, giving himself +30\% and his ally +20\%. If in a larger group, he could even cast it on 5 allies, each of whom would gains +10\% to their Deception skill.

The most common spells of this type are:
\begin{rpg-list}
\item Enhance Skill (Deception), often used by thieves.
\item Enhance Skill (Trade), used by merchants.
\item Enhance Skill (Influence), used by lawyers, con-artists and officers.
\item Enhance Skill (Resilience), used by warriors.
\item Enhance Skill (Persistence) used by magicians.
\end{rpg-list}

These spells are sometimes called by other names, such as “Cover of Night” or “Shadowstealth” (for Enhance Deception), “Golden Tongue” (for Enhance Influence or Trade), or “Toughen” (for Enhance Resilience).
\end{rpg-spell}


\begin{rpg-spell}
{Extinguish}
{Instant}

This spell instantly puts out fires. At Magnitude 1 it can extinguish a Flame, Magnitude 2 a Small Fire, Magnitude 3 a Large Fire and Magnitude 4 will put out an Inferno.
\end{rpg-spell}


\begin{rpg-spell}
{Total Awareness}
{Magnitude 2, Non-Variable}

This spell grants the recipient awareness as if they had physically got eyes in the back of their head for the duration of the spell. This allows them to make Perception rolls, and be aware of others behind them as they are with senses in front of them.
\end{rpg-spell}


\begin{rpg-spell}
{Fanaticism}
{Magnitude 2, Non-Variable}

The target of this spell will have close combat and unarmed combat skills increased by +20\% but may not attempt to parry, dodge or cast spells. Also for the duration of the spell the target has a +40\% bonus to any Persistence test related to Morale. The effects of this spell are automatically cancelled by the Demoralise spell and vice versa.
\end{rpg-spell}


\begin{rpg-spell}
{Fire Missile}
{Magnitude 2, Non-Variable, Touch, Trigger}

Casting this spell on a missile will cause it to burst into flame when it is fired/thrown and strikes a target. When it hits a target, the missile will deal +1D6 points of magical fire in addition to its normal damage. Since Fire Missile does magical damage, it affects creatures that are immune to normal damage. A missile under the effects of Fire Missile cannot benefit from Multi Missile or Speed Dart. 
\end{rpg-spell}


\begin{rpg-spell}
{Fire Weapon}
{Magnitude 4, Non-Variable, Touch}

For the duration of the spell, the target weapon will deal +1D6 points of magical fire damage in addition to its normal damage. A weapon under the effects of Fire Weapon cannot benefit from Weapon Enhance. Since Fire Weapon does magical damage, it damages creatures immune to normal damage.
\end{rpg-spell}


\begin{rpg-spell}
{Fist of Gold}
{Instant}

This spell creates a minor illusion of 5D10 Gold Ducats per level of Magnitude that persists for the duration of the spell.
\end{rpg-spell}


\begin{rpg-spell}
{Frostbite}
{Magnitude 2, Non-Variable}

This attack spell allows the caster to freeze his opponent, dealing 1D8 points of damage, ignoring any Armour Points. Magical damage that protect against cold damage can block this effect but mundane items (such as cold weather gear) are ineffective.
\end{rpg-spell}


\begin{rpg-spell}
{Glue}
{Area, Touch}

This spell covers an area of one centimetre square for each Magnitude with extremely sticky glue. If a creature steps on the glue, it must make an Athletics roll vs the Magnitude x 10\% to avoid being stuck for one round. On subsequent rounds it  must make the same roll to break free. This spell can also be used for more conventional repairs, a broken sword for example, with Magnitude x 10\% being the chance that the item won’t break again, if used in circumstances that would cause it to.
\end{rpg-spell}


\begin{rpg-spell}
{Hand of Death}
{Instant, Magnitude 4, Non-Variable, Resist (Resilience), Touch}

This fearsome spell allows the caster to deal an awful wound with the merest touch. Casting the Hand of Death, charges his body with the spell. Touching an unsuspecting target, or succeeding at an Unarmed attack against a wary target, releases the spell’s effect. If the Resilience test to resist the effect is failed, the victim immediately loses half their maximum Hit Points, and suffers a a Major Wound. If the Resilience test is a success, the target only loses 1D3 Hit Points. Armour does not protect against this damage.
\end{rpg-spell}


\begin{rpg-spell}
{Harden}
{Magnitude 1, Non-Variable, Touch}

This spell makes a target item unbreakable for the duration of the spell.  Therefore weapons with this spell cast on them will not break when a Fumble is rolled in combat, and it allows items that are normally too brittle to be wielded in combat to be used as improvised weapons.
\end{rpg-spell}


\begin{rpg-spell}
{Heal}
{Instant, Touch}

For every point of Magnitude of this spell, the caster can repair one Hit Point to damage of either himself or another target. In addition, a Heal spell of any Magnitude will stabilise a character suffering from a major wound, and/or revive a character who is unconscious. 

A Magnitude 4 (or two consequtive Heal 3 spells) or higher Heal spell will also cure any single poison or disease affecting the target. 

A Magnitude 6 (or two consequtive Heal 5 spells) or higher Heal spell will also repair the effects of a single major wound.
\end{rpg-spell}


\begin{rpg-spell}
{Hinder SKill (X)}
{Resist (Persistence)}

Like Enhance Skill (X), this is a number of different spells, each of which affects a different skill. For each point of Magnitude of the spell, the target gains a -10\% penalty to the next skill test using the affected skill.

Alternatively, for each additional point of Magnitude of the spell, the caster can affect one more target.  The bonuses and targets can be split as necessary providing each penalty is in multiples of 10\% and the total of bonuses equals the spells Magnitude x 10\%. If used in this way, each target is affected separately; if one target succeeds on resisting the spell, other targets may fail and be affected.

The most common spells of this type are: Hinder Skill (Perception), often used by thieves; Hinder Skill (Trade), used by the nastier traders; and Hinder Skill (Persistence) used by magicians against enemy spell-casters prior to casting spells upon them.
\end{rpg-spell}


\begin{rpg-spell}
{Ignite}
{Instant, Magnitude 1, Non-Variable}

Ignite will set fire to anything flammable within range, creating a flame. Skin or flesh cannot be ignited and if the target is attached to a living being (such as hair, fur or clothes) then the spell gains the Resist (Resilience) trait. 
\end{rpg-spell}


\begin{rpg-spell}
{Invisibility}
{Magnitude 4, Non-Variable, Concentration, Touch, Personal}

For the duration of the spell the recipient is completely invisible to sight.  They can still be heard, felt or smelled, with a -20\% to Perception tests. Also, the spell is automatically dispelled if the caster loses concentration, or the recipient casts a spell or makes an attack. The recipient also becomes visible immediately after the spell ending, so even if the caster immediately casts another Invisibility spell there will be a delay between castings where the recipient is visible.
\end{rpg-spell}


\begin{rpg-spell}
{Ironmind}
{Touch}

This spell hardens the resolve of the character that it is cast upon for its duration. Each level of Magnitude of the spell adds 10\% to all Persistence tests against magical attacks to the mind (e.g. Fear, Befuddle etc.) or opposed tests vs Influence.
\end{rpg-spell}


\begin{rpg-spell}
{Knock Back}
{Instant, Resist (Resilience)}

On a failed resistance roll the target of this spell is knocked back a number of metres equal to the spell’s magnitude.
\end{rpg-spell}


\begin{rpg-spell}
{Knockdown}
{Instant, Magnitude 2, Non-Variable, Resist (Resilience)}

On a failed resistance roll the target of this spell is knocked down prone.
\end{rpg-spell}


\begin{rpg-spell}
{Leap}
{Touch, Resist (Dodge)}

This spell causes the target to leap 2m up in the air for each point of Magnitude. If cast upon an unwilling target, who fails their resistance roll, they will then fall to earth taking normal falling damage (see page~\pageref{ssec:falling}).
\end{rpg-spell}


\begin{rpg-spell}
{Levitating Disc}
{Concentration, Area 1 per Magnitude}

This spell creates an invisible disc 1m in diameter for each point of Magnitude. It can carry weight equivalent to one person and their belongings per point of Magnitude, and moves at twice the Magnitude in metres per combat round.

So for example, a Levitating Disc with Magnitude 3 can carry 3 people, is 3m in diameter, and moves at a rate of 6m per combat round.
\end{rpg-spell}


\begin{rpg-spell}
{Light}
{Magnitude 1, Area 10}

Cast on a physical object (including living material), this spell causes the object to shed light across the area of effect. Note that only the specified area is illuminated – everything outside the area of effect is not. This spell creates raw light, not a flame.
\end{rpg-spell}


\begin{rpg-spell}
{Lock}
{Touch, Permanent}

This spell gives an item a resistance to being opened equal to the spell’s Magnitude x 10\%. The item must have a lock, such as might be found on a door or a chest, and the spell is focused on that lock. Once the lock has been forced/picked the spell is dispelled.
\end{rpg-spell}


\begin{rpg-spell}
{Mindspeech}
{}

This spell can affect one target for every point of Magnitude. It allows telepathy between the caster and any target, though targets will not have telepathy with one another. The words transmitted by telepathy must be whispered and will be heard directly in the head of the recipient, in the same language in which it was spoken. 
\end{rpg-spell}


\begin{rpg-spell}
{Mobility}
{}

For every point of Magnitude of this spell, the target’s Movement Rate will be increased by 2m.
\end{rpg-spell}


\begin{rpg-spell}
{Multi Attack}
{Instant}

Each point of Magnitude allows the caster to make one extra close-combat attack. These attacks happen in a blur of motion at the same DEX rank that a normal attack occurs. Each casting of the spell grants a single flurry of such attacks.
\end{rpg-spell}


\begin{rpg-spell}
{Multi Missile}
{Touch, Trigger}

If the caster succeeds in casting the spell, a missile weapon is charged with the spell for ten minutes. A missile under the effects of Multi Missile cannot benefit from Fire Missile or Speed Dart. 

When the Multi Missile-enchanted missile is fired/ thrown, one additional magical missile is created for every point of Magnitude. Each magical missile’s attack is rolled for separately and each does the same damage as the original (though they will not benefit from the character’s damage modifier). Magical missiles created through Multi Missile will not cause critical hits, though the original missile can. Magical missiles created through Multi Missile will affect creatures that can only be hurt by magic. 
\end{rpg-spell}


\begin{rpg-spell}
{Noxious Vapours}
{Magnitude 2, Non-Variable, Area 10, Resist (Resilience)}

This spell fills a volume 10 metres in radius with thick choking green gas. Any living creature that breathes oxygen who fails Resilience test takes 1D4 damage per round and is incapacitated due to heavy coughing. Next round make a Resilience test to see if they compose themselves enough to overcome the incapacitating coughing,.They still take 1D4 damage every round that they are in the cloud. The cloud also obscures vision, providing any creature within it with cover, so that an ranged attackers are at -40\% to their attack roll and that any melee in the cloud is at -20\%.
\end{rpg-spell}


\begin{rpg-spell}
{Personal Insight}
{Magnitude 2, Non-Variable}

This spell gives the caster or recipient a very direct insight into a small question directly relevant to them, in the form of an internal intuition.

For example the question “Why can I not harm the creature?” would get the answer “Because your sword is not enchanted”, while “Why can we not harm the creature?” would not get an answer.
\end{rpg-spell}


\begin{rpg-spell}
{Pierce}
{Touch}

This spell can be cast on any weapon with a blade or point. For every point of Magnitude, it ignores one armour point when it strikes armour. Pierce can bypass magical armour as easily as normal armour. 
\end{rpg-spell}


\begin{rpg-spell}
{Protection}
{}

For every point of Magnitude of this spell one armour point is added to the armour of the target. This stacks with any existing armour and is treated in the same way. 
\end{rpg-spell}


\begin{rpg-spell}
{Push/Pull}
{Instant, Resist (Resilience)}

This spell allows the caster to move an item of up to 3 SIZ or ENC per point of Magnitude either towards or away from them in a straight line, as if pushed suddenly from one direction or the other. The item is not moved with significant enough force to inflict damage unless it is naturally damaging (a bottle of acid, for instance) and the caster has no control over the distance pushed or pulled; as this depends on the location of the item or the surface it rests on. Living creatures targeted by this spell are allowed a Resilience roll to resist.
\end{rpg-spell}


\begin{rpg-spell}
{Read Emotion}
{Magnitude 1, Non-Variable, Instant, Resist (Persistence)}

This spell when cast tells you what the true emotional state of the target is, if they fail a Persistence roll.
\end{rpg-spell}


\begin{rpg-spell}
{Resist (Element)}
{}

This spell increases Resistance against hostile effects, magic or otherwise, from a given element (Air/Darkness/Earth/Fire/Water) by 10\% per Magnitude, and subtracts 1 point of damage from that element per Magnitude.
\end{rpg-spell}


\begin{rpg-spell}
{Restore Energy}
{Instant, Touch}

Each point of this spell’s Magnitude instantly restores one fatigue level to the recipient.
\end{rpg-spell}


\begin{rpg-spell}
{Sap Energy}
{Instant, Touch, Resist (Resilience)}

Each point of this spell’s Magnitude inflicts drains one fatigue level from the target upon a failed Persistence roll.
\end{rpg-spell}


\begin{rpg-spell}
{Scare}
{Magnitude 2, Non-Variable, Resist (Persistence)}

On a failed resistance roll, the target is scared for 1D6 rounds. Scared targets must withdraw from combat with the caster for the duration of the spell, and move as quickly as they are able, directly away from the caster.
\end{rpg-spell}


\begin{rpg-spell}
{Second Sight}
{Magnitude 3, Non-Variable}

Second Sight allows the caster to gauge the POW of every creature and magic item within range. The spell is blocked by anything that blocks normal vision. The caster will know if each aura created by the illuminated POW is less than his own POW, within three points of his own POW or greater than his own POW. 

Additionally, Second Sight provides a +20\% bonus on Perception tests to notice hidden magical items or hiding people or creatures. Second Sight will also reveal invisible entities; though only a hazy image will show (treat such targets as partially obscured). 
\end{rpg-spell}


\begin{rpg-spell}
{Scare}
{Magnitude 2, Non-Variable, Resist (Persistence)}

On a failed resistance roll, the target is scared for 1D6 rounds. Scared targets must withdraw from combat with the caster for the duration of the spell, and move as quickly as they are able, directly away from the caster.
\end{rpg-spell}


\begin{rpg-spell}
{Skybolt}
{Magnitude 3, Non-Variable, Instant, Resist (Dodge)}

The caster summons a lightning bolt from the heavens regardless of the weather. The target must be outdoors in plain view. Skybolt inflicts 2D6 points of damage to a single chosen target. Only magical Armour Points offer protection against this damage.
\end{rpg-spell}


\begin{rpg-spell}
{Slip}
{Magnitude 1, Non-Variable, Resist (Dodge)}

The caster makes the ground under the target’s feet as slippery as sheet black ice. The target must make an Athletics roll or fall over prone.
\end{rpg-spell}


\begin{rpg-spell}
{Slow}
{Resist (Resilience)}

For every point of Magnitude of this spell the target’s Movement Rate will be decreased by 2m. A target’s Movement may not be reduced to below one metre through use of this spell. 
\end{rpg-spell}


\begin{rpg-spell}
{Speed Dart}
{Magnitude 2, Non-Variable, Touch, Trigger}

Cast on a missile this spell is triggered when it is fired. It gives a +20\% to Ranged Combat and +3 damage while using the missile. A missile under the effects of Speed Dart cannot benefit from Fire Missile or Multi Missile.
\end{rpg-spell}


\begin{rpg-spell}
{Strength}
{Touch}

For every point of Magnitude of this spell, the target’s Damage increases by +1 and strength based athletics tests are +10\% per Magnitude. Note the Damage increase is not treated as magical damage.
\end{rpg-spell}


\begin{rpg-spell}
{Talk to Animal}
{Magnitude 3, Non-Variable}

With this spell the recipient is able to talk to any beast within ten metres of them. This communication is verbal, therefore the recipient must be able to speak and be heard by the target animal. 
\end{rpg-spell}


\begin{rpg-spell}
{Thunder's Voice}
{}

This spell grants the caster a thunderous voice of command. For every point of Magnitude of this spell, the caster has +10\% added to his Influence skill and can also be heard at up to the spell’s Magnitude x 100 in metres.
\end{rpg-spell}


\begin{rpg-spell}
{Tongues (Language)}
{Magnitude 2, Non-Variable}

This spell allows the recipient to speak another language perfectly for its duration. There is a different spell for each language.
\end{rpg-spell}


\begin{rpg-spell}
{Unlock}
{Touch, Instant}

This spell has a chance of opening a lock equal to the spell’s Magnitude x 20\%, minus any modifiers due to the intricacy of the lock. If cast on a lock that has had a Lock spell cast on it, the test is an Opposed Test vs the Magnitude x 20\% of the Lock spell.
\end{rpg-spell}


\begin{rpg-spell}
{Vigour}
{Touch}

For every point of Magnitude of this spell, the target’s Hit Points score increases by +2. A target cannot have its Hit Points increased in this way to more than twice its original score. Damage is taken from the ‘magical’ Hit Points first, so when the spell dissipates the damage that was inflicted on the magical Hit Points disappear too.  If you are using the Major Wounds optional rule then recalculate the Major Wound level while the spell is in effect.
\end{rpg-spell}


\begin{rpg-spell}
{Vomit}
{Resist (Resilience)}

This spell incapacitates its Victim for 1 round per point of Magnitude, due to uncontrollable vomiting. On a fumbled resilience roll the Victim takes 1D6 Hit Points damage.
\end{rpg-spell}


\begin{rpg-spell}
{Scare}
{Magnitude 2, Non-Variable, Resist (Persistence)}

On a failed resistance roll, the target is scared for 1D6 rounds. Scared targets must withdraw from combat with the caster for the duration of the spell, and move as quickly as they are able, directly away from the caster.
\end{rpg-spell}


\begin{rpg-spell}
{Walk on (Element)}
{Magnitude 3}

This spell allows the recipient to walk on the specified element (Air/Darkness/Earth/Fire/Water) without sinking or taking any harm from what is being walked on for the spell’s duration. With this spell for the appropriate element, the caster can walk across lava, quicksand, water, or even through the air. Each additional point of Magnitude increases the duration of the spell by 1 minute.
\end{rpg-spell}


\begin{rpg-spell}
{Water Breath}
{Touch}

This spell allows the target to breathe water for the duration of the spell. For every point of Magnitude, one additional person can be included in the spell, or an the duration, by one minute. Water Breath has no effect on the target’s ability to breathe air.
\end{rpg-spell}


\begin{rpg-spell}
{Weapon Enhance}
{Touch}

This spell can be cast on any close combat weapon or any unarmed attack. For every point of Magnitude, it increases the chance to hit with the weapon by +10\% and deals one point of extra damage. This extra damage is magical and will affect creatures that can only be hurt by magic. The weapon’s base damage remains non-magical. A weapon under the effects of this spell cannot benefit from Fire Weapon.
\end{rpg-spell}


\section{Creating Magic Items}

Most Magic Items found in play in a game of Fantasy D100 will have been created by the characters or Non Player Character magicians.

Although the spells that their characters use to create Magic Items are detailed in their respective spell lists, its worth going through them briefly remind yourself what spell does what and how it is used.

\begin{description}
\item [Create Charms] This is the basic spell for creating Magic Items. Use this spell to create rune-inscribed swords, paper talismans that protect against sprits, amd dragon skin armour that is resistant to fire (via a Resist Fire spell).
\item [Create Power Point Store] If you want to create a magic item that has Power Points already stored, so the user doesn’t have to use their own, this is the spell to use.
\item [Potions] This is a quick way of making non-reusable spell stores where you’ve already spent the magic points, for you or your allies to gulp down for instant effect during a combat. Think Healing + Create Potions and you have the classic Healing Potion.
\end{description}

\subsection{Identify a Magic Item}
There is no catch all “Detect Magical Properties” or “Know Magic item” skill in Fantsy D100. This is quite deliberate, keeping with the general policy that such items are not the equivalent of Magical shotguns.  Some options are:

Consult a Sage or other magical expert. This option will cost the characters lots of money. Take a baseline of one hundred silvers per point of spell magnitude OR some perilous quest that the character must do in return. Such experts are rare, because most high ranking Magicians have little time for magical research for others, and would be more interested in their own schemes. 

Detect Magic spells. This merely tells you the item is magical.  A critical casting may tell the caster how powerful the item is.

Trial and error. The character tries to find out the item’s use by experiment. Allow creative and imaginative plans to reveal partially what the item does.

Researching the myths and legends around the item.  This is the most certain way of finding out what a magic item does. Of course such myths may be obscure themselves, requiring a dangerous Quest to a long hidden repository of knowledge to find.

\fi
