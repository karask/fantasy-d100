\chapter{Sorcery}
\label{ch:sorcery}

Sorcery is an approach to magic that acknowledges that there are magical rules that govern the Universe and that by studying these rules a Magician can manipulate reality to his will.

Often Sorcery is atheistic, regarding gods and spirits as merely intelligent forces of the Universe, that exist to be interacted with and dealt with on an equal footing. 

Practitioners of Sorcery develop in one of two ways. The majority are organised into schools of wizardry, which have their own books of spells and rules that they teach their apprentices. Alternatively, there is a long tradition of practitioners working in solitude, cut off from other Sorcerers and society at large, to focus purely on their magical activities. Occasionally they take on an apprentice, to teach their art, or simply as a helping hand around the magical laboratory.


\section{Spot Rules}

\subsection{Learning Sorcery}
Sorcery is governed by the Sorcery Casting magical skill. This skill is automatically acquired at its basic score (INT) when the character is first created. This skill may be improved normally though the use of Improvement Points.  Even non-Sorcerors have this skill , at base, since it is used to give them a chance to use Sorcerous magic items and scrolls that store Sorcery spells. 

\subsection{Learning Spells}
Before a spell can be cast using Sorcery, the following process must be followed:
The character must first learn the spell through research. In order to learn a particular Sorcery spell, the caster must possess the spell in written form or be taught it by a teacher. In game terms this means having access to a teacher who knows the spell or a book or scroll where it is written down. The player then spends two Improvement Points and writes the spell down on their character sheet. Once the Sorcery spell has been learned, the character will be ready to try casting it.



\subsection{Casting Spells}
A character must be able to gesture with his hands, and be able to chant, in order to cast a spell. Whenever a spell is cast using Sorcery, there will always be a sight and sound that nearby creatures can detect, be it a flash of light, a crack of thunder, or a shimmering in the air. The exact effects are up to the Games Master and Player to decide , but will automatically be detected by any creatures within ten times the Magnitude of the spell in metres. 

Casting a Sorcery spell requires a successful skill test using the Sorcery Casting skill. If successful, the spell takes effect. If the casting test fails, the spell does not take effect. 

\subsubsection{Power Points}
All Sorcery spells cost a base of one Magic Point to cast. Sorcery spells can be modifying as the caster wishes (if he has the appropriate power points). If a Manipulation effect is applied to a spell, each effect costs one Power Point to apply (see below). 

The result of the Sorcery casting test depends on its success:
\begin{description}
	\item[Success:] A number of Power Points are deducted from the spellcaster’s total, equal to the Manipulation effects Power Points plus one of the spell. The spell then takes effect.
	\item[Failure:] The spell does not take effect and the character loses the spell's Power Points.
	\item[Critical:] Any attempt to resist or counter the spell suffers -20\% penalty. Moreover, only the base cost of one Power Point is lost (not for any Manipulations).
	\item[Fumble:] The spell fails and the Sorcerer loses 1D6 Power Points in addition to normal Power Point loss.
\end{description}


\subsubsection{Casting Time}
No other Combat Action may be taken while casting a spell, though the character may slowly walk up to half their Movement. 

A spell takes effect at the end of its casting, which starts at the beginning of the Combat Round and ends on the INT of the Caster in the Combat order. Note that while spellcasting, a character will draw possible attacks from enemies they are adjacent to during a Combat Round. 

Distractions or attacks on the spellcaster as he casts will either automatically ruin the spell (if the spellcaster is blinded or disarmed, or suffers a Major Wound) or require Persistence tests for them to maintain concentration on the spell. 

\subsubsection{Spell Manipulations}
Sorcery spells have three basic effects which can be manipulated by the caster: Magnitude, Duration, and Range.

Each effect has a default value which the spell can be cast at, costing one Power Point. The default value for the spell effects are listed in table~\ref{tab:sorcery-manipulations}.

\begin{table}
\begin{center}
\caption{Sorcery Manipulations}
\label{tab:sorcery-manipulations}
\begin{rpg-table}[|c|c|c|Y|]
        \hline
	\textbf{Power Points}  & \textbf{Magniture} & \textbf{Duration} & \textbf{Range}\\
        \hline
	1 (Default) & 1 & 5 minutes & 10m\\
	+1 & 2 & 15 minutes & 20m\\
	+2 & 3 & 30 minutes & 40m\\
	+3 & 4 & 1 hour & 80m\\
	+4 & 5 & 2 hours & 160m\\
	+5 & 6 & 4 hours & 320m\\
	+6 & 7 & 12 hours & 640m\\
	+7 & 8 & 1 day & 1km\\
	+8 & 9 & 2 day & 2km\\
	+9 & 10 & 5 day & 5km\\
	+10 & 11 & 1 week & 10km\\
	+11 & 12 & 2 weeks & 20km\\
	+12 & 13 & 1 month & 50km\\
	+13 & 14 & 2 months & 100km\\
	+14 & 15 & 3 months & 200km\\
	+15 & 16 & 6 months & 500km\\
	+16 & 17 & 1 year & 1000km\\
	+17 & 18 & 2 years & 2000km\\
	+18 & 19 & 5 years & 5000km\\
	+19 & 20 & 10 years & 10000km\\
	\hline
\end{rpg-table}
\end{center}
\end{table}

The tens value of the caster’s Sorcery Casting skill determines the max number of additional Power Points that can spend on each of the manipulation types. 

\begin{rpg-examplebox}
Omar the Magnificent with a Sorcery Casting skill of 80\% can spend an additional 8 Power Points on manipulating each of the spell’s effects, in Magnitude, Duration and Range. That’s a manipulation of up to 8 levels for each effect, not 8 levels in total across all three effects.
\end{rpg-examplebox}

The decision of which effects to manipulate and how many extra Magic Points are to be spent is made before the spell is cast.

\begin{rpg-examplebox}
Lura casts Damage Boosting on Rurik’s sword, and wants it to be at a magnitude of 4 for an hour.

She has a Sorcery Casting skill of 60\%, which means she can spend an additional six Power Points on manipulating any spell’s effects. Looking at the Manipulation table, Lura can comfortably manage a Magnitude of 4, for three additional Power Points, and can manage a duration of an hour with another three points. 

Lura’s player rolls the dice and compares the result against Lura’s casting skill of 60\% to see whether she successful casts the spell.

In fact Lura, can spend a maximum of six points on a magnitude of range 640m, another six on a duration of 12 hours and another 6 on a magnitude of 7, which is a total of 19 Magic Points (18 for the manipulations and 1 for the spell itself).
\end{rpg-examplebox}


\subsubsection{Dismissing Spells}
In a single Combat Round, a caster can dismiss any Permanent spell(s) he has cast, as a free action. Ceasing to cast a Concentration spell is immediate and not an action. 


\subsubsection{Extra Power Points}
As you can probably work out from the example above, it is possible for a Sorcerer to cast a spell which needs more Power Points in its manipulated form than a Sorcerer will normally have. Sorcerers get round this by carrying either Power Point stores (see Magic spell Create Magic Store) or other artifacts that can store Power Points (e.g. from other disciplines).


\subsection{Spell Traits}
All Sorcery spells share the same basic qualities:

\begin{rpg-list}
\item Base Magnitude of one 
\item Duration of 5 minutes, and
\item Range of 10 metres.
\end{rpg-list}

Other traits used by spells are detailed below. 
\begin{description}
	\item[Concentration:] The spell’s effects will remain in place as long as the character concentrates on it. Concentrating on a spell is functionally identical to casting the spell, requiring the spell caster to continue to gesture with both arms, chant and ignore distractions. This trait overrides the normal Sorcery spell default Duration. 
	\item[Instant:] The spell’s effects take place instantly. The spell itself then disappears. This trait overrides the normal Sorcery spell default Duration. 
	\item[Permanent:] The spell’s effects remain in place until they are dispelled or dismissed. This trait overrides the normal Sorcery spell default Duration.
	\item[Resist (Dodge/Persistence/Resilience):] The spell’s intended effects do not succeed automatically. The target may make a Dodge, Persistence or Resilience test (as specified by the spell) in order to avoid the effect of the spell entirely. Note that Resist (Dodge) spells require the target to be able to use Reactions in order to Dodge. In the case of Area spells, the Resist (Dodge) trait requires the target to dive in order to mitigate the spell’s effect. 
	\item[Touch:] Touch spells require the character to actually touch his target for the spell to take effect, using an Unarmed skill test to make contact. The caster must remain in physical contact with the target for the entire casting. TODO NEED TO CLARIFY THIS. This trait overrides the normal Sorcery spell default Range. 
\end{description}

\section{Spells}

\begin{samepage}
\begin{rpg-spell}
{Animate (Substance)}
{Concentration}

This spell allows the Sorcerer to animate the substance indicated, up to one SIZ for every point of Magnitude. The Sorcerer can cause it to move about and interact clumsily (Movement of 1m per three points of Magnitude). 

The Sorcerer’s chance to have the animated object perform any physical skill successfully is equal to his own chance to perform that action halved (before any modifiers). If the appropriate Form/Set spell is cast immediately after this spell, the caster can perform much finer manipulation of the object. In this case, the animated object will use the caster’s full skill scores for physical activities. 

This spell can only be used on inanimate matter. 
\end{rpg-spell}
\end{samepage}


\begin{samepage}
\begin{rpg-spell}
{Cast Back}
{}

This protective spell offers a chance of sending hostile spells (sorcery and magic) back to the attacking spell caster. 

Cast Back only affects spells that target the user specifically and have the Resist trait. Such spells may affect the protected character normally, but if it is resisted, the spell is launched back at the person who cast it, as long as its Magnitude is not greater than the Cast Back’s Magnitude. 
\end{rpg-spell}
\end{samepage}


\begin{samepage}
\begin{rpg-spell}
{Create Familiar}
{Permanent}

This spell allows the caster to make a personal sacrifice in order to create a Familiar. The Nature of the Familiar is dependent on the setting, but the spell typically only works on a non-sentient creature or inanimate object. The spell takes one hour to perform. The typical cost of ingredients for the spell is 50 Gold Ducats and it requires 3 Improvement Points. 

\begin{rpg-list}
\item A Sorcerer can have up to ½ their POW in Familiars, but few have more than one or two.
\item When a Familiar is destroyed or killed the Sorcerer loses 1D6 Hit Points from the trauma.
\item A Familiar gains sentience when created equal to 2D6 INT and Magic equal to 2D6 POW, unless the animal had a higher score, in which case that POW is used. 
\item A Familiar has a permanent mental connection with its master; with a range equal to 1/2 the caster’s POW in kilometres. Through this link the Familiar can allow the caster to ‘see’ through its perceptual abilities. The Sorcerer can cast spells on the Familiar as if touching it and the Familiar can cast any spells it knows on its master.
\item A Familiar can perceive its surroundings. How this happens depends on the type of Familiar. Animals can sense the world through their ordinary perceptions. Magical objects can detect the world around them up to a range equal to their POW in metres.
\item A Familiar dies when its master dies, and its life span is normal for the object or animal in which it is bound. It has Hit Points equal to the object or animal from which it was created.
\item Familiars may learn skills, but only ones they are capable of performing. Most Familiars in objects can only learn knowledge and magical skills. Animals have the skills that come naturally to them.
\item An animal Familiar can only be up to 1/2 the SIZ of the caster. 
\end{rpg-list}

\end{rpg-spell}
\end{samepage}


\begin{samepage}
\begin{rpg-spell}
{Create Scroll}
{Permanent}

These are written items which store Sorcery Spells.

All scrolls have an attached cost of 1 Gold Ducat per magnitude of spell in ingredients for special inks/parchments, etc.

The resulting scroll is a one use item, which upon a successful Sorcery Casting test, casts the spell(s) with any manipulations, at the magnitude that was cast on the scroll.

Alternatively, upon a successful Sorcery Casting the reader of the scroll can learn the spell by spending the appropriate number of Improvement Points.

Either way, upon a successful use of the scroll, the spell fades from the scroll. If the casting roll merely fails the spell remains, but the reader can not attempt to use the scroll until their Sorcery Casting skill increases. If the casting roll is fumbled the spell fades from the scroll, without any benefit to the reader.
\end{rpg-spell}
\end{samepage}


\begin{samepage}
\begin{rpg-spell}
{Create Spell Matrix}
{Permanent}

This spell creates items that store Sorcery spells. 
The enchanter pays 1 Improvement Point per spell stored in the matrix.

All spell matrices have an attached cost of 10 Gold Ducats per spell in special materials.

The wielder can cast and manipulate the spell at the skill of the original enchanter, using their own Power Points as fuel.

Spell matrices are mundane items in their own right and if the item is broken, then the spell is dispelled. However at the time of enchantment the enchanter can spend another Improvement Point  to magically harden the item doubling its Hit Points and Armor Points.
\end{rpg-spell}
\end{samepage}

\begin{samepage}
\begin{rpg-spell}
{Damage Boosting}
{Touch}

This spell can be cast upon any weapon up to five ENC. Each point of Magnitude adds one point to the weapon’s damage (the basic spell will increase a hatchet from 1D6 damage to 1D6+1 damage, for instance). 
\end{rpg-spell}
\end{samepage}


\begin{samepage}
\begin{rpg-spell}
{Damage Resistance}
{Touch}

This spell protects the body of the recipient. Any incoming attack dealing damage equal to or less than the Magnitude of the spell is ignored. Any incoming attack dealing more damage than the Magnitude of Damage Resistance is unaffected and will deal its full damage as normal. Note that the protected character may still suffer from Knockback if applicable. 
\end{rpg-spell}
\end{samepage}


\begin{samepage}
\begin{rpg-spell}
{Diminish (Characteristic)}
{Resist(Persistence/Resilience), Touch}

There are actually seven Diminish spells, one for each Characteristic. The spell will temporarily apply a penalty to the specified Characteristic equal to the Magnitude of the spell. The penalty applied by this spell may not reduce a Characteristic below one and a creature must have the Characteristic in question to be affected by this spell. 

Diminish (STR, DEX, CON and SIZ) are resisted with Resilience. Diminish (INT, POW and CHA) are resisted with Persistence. 

Applying a penalty to POW does not reduce the character’s Magic Points.

Note that not all uses of this spell are malicious. Thieves and others often value the timely use of a Diminish (SIZ) spell, as it can greatly enhance their ability to enter restricted areas. 
\end{rpg-spell}
\end{samepage}


\begin{samepage}
\begin{rpg-spell}
{Dominate (Species)}
{Resist(Persistence)}

This spell allows the caster to gain control over a creature belonging to a specific species. If the target fails to resist the spell, it must obey the commands of the caster for the duration of the spell. 

The controlled creature shares a telepathic link with the Sorcerer by which it can receive its orders. If the Sorcerer and the creature dominated do not share a common language, the Sorcerer can order it about by forming a mental image of the actions he wishes the dominated creature to perform.
\end{rpg-spell}
\end{samepage}


\begin{samepage}
\begin{rpg-spell}
{Energy Projection (Type)}
{Ranged, Instant, Resist(Dodge)}

Energy is either projected as a beam or a ball towards the target, which can avoid the attack by Dodging.

If the spell takes effect the target takes damage equal to double the Magnitude of the spell. Physical Armour does not protect against the damage, but magical protection does. Types of energy that can be projected by this spell are Cold (Dark), Lightning, Heat (Fire), Shards of Rock (Earth), Windblast (Air).
\end{rpg-spell}
\end{samepage}


\begin{samepage}
\begin{rpg-spell}
{Enhance (Characteristic)}
{Touch}

There are actually seven Enhance spells, one for each Characteristic. Essentially the reverse of the Diminish spell, Enhance allows the Sorcerer to temporarily apply a bonus to the specified Characteristic equal to the Magnitude of the spell. A creature must have the Characteristic in question to be affected by this spell. 

Applying a bonus to POW does not increase the character’s Power Points. 
\end{rpg-spell}
\end{samepage}


\begin{samepage}
\begin{rpg-spell}
{Fly}
{Concentration, Resist (Persistence)}

Using this spell allows the caster (or whomever or whatever he targets with the spell) to fly.  The caster may levitate a number of objects or characters (the caster counting as one of these characters if he so wishes). 

A levitated character may not be Overloaded and must have a SIZ Characteristic which is lower than the Sorcerer’s POW characteristic. 

Objects must have an ENC lower than the Sorcerer’s POW characteristic. 

Characters or objects moved by this spell have a base Movement Rate of 6m. All objects and characters moved by this spell move at the spellcaster’s behest, not their own. 

Each point of the spell’s Magnitude may either be used to increase the target’s Movement by +2m or to target an additional object or character – but not both. A Sorcerer casting this spell at Magnitude 4 may fly himself with a Movement of 12m, fly himself and a friend with a Movement of 8m each, or fly himself and three friends with a Movement of 6m each.
\end{rpg-spell}
\end{samepage}


\begin{samepage}
\begin{rpg-spell}
{Form/Set (Substance)}
{Instant}

There are an unlimited number of Form/Set spells in existence, one for every substance imaginable, from steel to smoke to water. 

Each point of Magnitude allows the caster to shape one ENC of solid substance or one cubic metre of an ethereal substance (like darkness). The caster must be Familiar with the shape he is forming. 

When the caster has finished the forming process, the substance retains its shape. Rigid substances like steel will hold the form they had at the end of the spell, while more mutable substances like water will immediately lose their shape. 

This spell can be used to mend damage done to an object. The Sorcerer must form the entire object and must succeed at an appropriate Craft test. If successful he will restore the item to its original condition. 

This spell can only be used on inanimate substances. 
\end{rpg-spell}
\end{samepage}


\begin{samepage}
\begin{rpg-spell}
{Glow}
{}

This spell causes a glowing point of light to appear on a solid substance. At its base, the spell creates an area of light one metre in radius, giving off the same illumination as a candle. Each additional point of Magnitude increases the radius of effect by one metre. At Magnitude 3, the brightness of the spell increases to that of a flaming brand at its centre. At Magnitude 5, it increases to that of a campfire and at Magnitude 10 to that of a bonfire. 

This spell can be cast on an opponent’s eyes. If cast on a living being the spell also gains the Resist (Dodge) trait. If the target fails to resist it, he will suffer a penalty to all attack, parry and Dodge tests, as well as any skills relying upon vision, equal to five times the spell’s Magnitude, until the spell ends or is dispelled. 
\end{rpg-spell}
\end{samepage}


\begin{samepage}
\begin{rpg-spell}
{Haste}
{}

Each point of Magnitude of Haste adds 1m to the Movement rate of the recipient. Every two points of Magnitude also adds +1 to the recipient’s Dexterity or Intelligence for the purposes of determining order in combat. 
\end{rpg-spell}
\end{samepage}


\begin{samepage}
\begin{rpg-spell}
{Hinder}
{Resist (Resilience)}

Each point of Magnitude of Hinder subtracts 1m from the Movement rate of the target. Every two points of Magnitude also subtracts 1 from the recipient’s Dexterity or Intelligence for the purposes of determining order in combat. 
\end{rpg-spell}
\end{samepage}


\begin{samepage}
\begin{rpg-spell}
{Holdfast}
{Touch}

This spell causes two adjacent ten centimetre by ten centimetre surfaces (roughly the size of a man’s palm) to commingle into one. The basic bond has a STR of 1. Each additional point of Magnitude will either increase the STR of the bond by +1 or double the area affected. 

This spell can affect organic and inorganic substances. If the caster is attempting to bond a living being with this spell, the spell gains the Resist (Resilience) trait.
\end{rpg-spell}
\end{samepage}


\begin{samepage}
\begin{rpg-spell}
{Mystic Vistion}
{Concentration}

This spell allows the recipient to literally see magic. By augmenting the recipient’s natural vision, the spell allows him to see a creature’s Power Points, as well as enchanted items with their own Power Points or spells. The recipient must be able to actually see the creature or object for this spell to work.

On a normal success the recipient of the spell will only know roughly how many Power Points an object or creature has (1–10, 11–20, 21–30 and so forth). On a critical they will know exactly. On a fumble the Games Master should give the player a misleading total.

By looking at a spell effect, a recipient of Mystic Vision will automatically be aware of its magical origin (divine, magic, or sorcery). By increasing the Magnitude of Mystic Vision, the caster can learn more about what he is seeing. Compare the Magnitude of Mystic Vision to the Magnitude of any spell that the target is either casting or under the influence of. As long as Mystic Vision’s Magnitude exceeds the other spell’s, the caster will be able to precisely determine the effects of the perceived spell, and a mental image of who cast the spell (if it is not obvious). 

By looking at an enchanted item, a recipient of Mystic Vision will automatically be aware of its gross magical effects (such as the types of enchantment currently on the item). Each point of Magnitude of Mystic Vision will also determine either the invested POW (and therefore the relevant strength) of a particular enchantment or a particular condition laid upon an enchantment, at the Games Master’s choice.
\end{rpg-spell}
\end{samepage}


\begin{samepage}
\begin{rpg-spell}
{Neutralise Magic}
{Instant}

This spell allows a caster to neutralise other spells. Neutralise Magic will eliminate a combined Magnitude of spells equal to its own Magnitude, starting with the most powerful affecting the target. If it fails to eliminate the most powerful spell then it will instead target the second-most powerful spell. As soon as Neutralise Magic can no longer dismiss a target’s spells, because all the remaining spell’s Magnitudes are too high, its effects immediately end. 

Neutralise Magic can be fired as a Reaction, but only when another spell is cast within Neutralise Magic’s Range that the character wishes to counter. A successful Neutralise Magic disrupts the other spell and nullifies it. As long as Neutralise Magic’s Magnitude equals or exceeds the target spell’s Magnitude, the target spell is countered. 
\end{rpg-spell}
\end{samepage}


\begin{samepage}
\begin{rpg-spell}
{Other World Portal (Other World)}
{Instant}

This spell creates a portal to a named Other World. The Magnitude of the spell is the number of creatures (SIZ 12-18) who can use the portal simultaneously. The portal exists as long as the spell is in effect. When the spell’s duration is reached, the portal closes instantly. 

If the spell casting is fumbled, catastrophic events occur. Here are some example events but the creative Games Master is encouraged to create more:
\begin{rpg-list}
\item A malignant creature from that Other World emerges from the portal and attacks the Sorcerer, in an attempt to close the portal. 
\item The Sorcerer and all within 10m of him are sucked through the portal, which then promptly closes. Worse the Sorcerer is so befuddled that he cannot remember this spell for D20+D4 hours.
\item The Other World, to which the portal is connected, invades the home realty in a 1D10 km diametre from the portal. The home reality protects itself by throwing up a magical barrier that lets things into the beachhead but not out. 
\end{rpg-list}
\end{rpg-spell}
\end{samepage}


\begin{samepage}
\begin{rpg-spell}
{Palsy}
{Resist (Resilience)}

If the caster is able to overcome his target with this spell, he can turn the victim’s own nervous system against him. The spell will paralyse the target, provided the spell’s Magnitude is greater than the quarter of target’s current Hit Points. 
\end{rpg-spell}
\end{samepage}


\begin{samepage}
\begin{rpg-spell}
{Protective Sphere}
{}

When completed, the Protective Sphere will create a sphere-shaped area of protection with a radius in metres equal to the spell’s Magnitude. If this spell is cast on the ground (or other immovable place) it cannot be moved. If cast on a vehicle (such as the bed of a wagon) or a person, it will move with the target.  After the sphere has been completed any one or all of the following spells can be added to provided the defensive capacities of the Sphere during the duration of the Sphere. The Sphere on its own provides no protection, that is down to the Resistance spells: Damage Resistance, Spell Resistance, Spirit Resistance TODO: Spirit discipline!.

The Protective Sphere’s perimetre contains the benefits of its combined Resistance spell(s). The Protective Sphere only inhibits spells or attacks entering the circle from the outside – attacks or spells originating within the circle are unaffected. Thus a Protective Sphere against spirits would block out outside spirits but have no effect on those already inside its perimetre. A Protective Sphere e against damage or spells would block out incoming attacks/spells, but have no effect on those attacks made within the sphere (including attacks targeting those outside the sphere). 
\end{rpg-spell}
\end{samepage}


\begin{samepage}
\begin{rpg-spell}
{Regenerate}
{Concentration Special, Instant, Touch}

This spell causes a severed or maimed limb to regrow or reattach. Regenerate cannot return a character from the embrace of death. 

The Magnitude of the spell must equal or exceed the maximum Hit Points lost as a result of the Major Wound taken. This spell will cause a limb severed by a Major Wound to regrow or, if the detached limb is still present, for the limb to reattach itself to its stump. 

Regenerate takes a number of minutes equal to the target’s SIZ to reattach a limb, during which time the caster must maintain concentration on the spell. The Hit Points lost due to the Major Wound are recovered at the end of this period.
\end{rpg-spell}
\end{samepage}


\begin{samepage}
\begin{rpg-spell}
{(Sense) Projection}
{Concentration}

Each (Sense) Projection spell is a separate spell. These spells encompass the five base senses but there are also variants for any unusual sensory mechanism appropriate to the game world (such as sonar). 

This spell allows the caster to project one of his senses anywhere within the spell’s Range. The spell forms an invisible and intangible sensor, some ten centimetres across, which receives the specified type of sensory input and transmits it to the caster. The sensor can move a number of metres per Combat Round equal to the spell’s Magnitude at the Sorcerer’s direction and allows him to use his Perception skill through the sensor. 

Spells can be cast through the sensor of some Projection spells. For instance, ranged spells require Sight Projection, while touch spells require Touch Projection (and likely Sight Projection too, simply so the Sorcerer can find his target efficiently). 

Characters using Mystic Vision can see the sensor and attack it if they wish, though it is only vulnerable to magic. Magical weapons and spells employed against the sensor will not destroy it but will instead transfer their damage directly to the caster.
\end{rpg-spell}
\end{samepage}


\iffalse

NOT INCLUDED SPELLS: Mirage, 

\section{Creating Magic Items}

Most Magic Items found in play in a game of Fantasy D100 will have been created by the characters or Non Player Character magicians.

Although the spells that their characters use to create Magic Items are detailed in their respective spell lists, its worth going through them briefly remind yourself what spell does what and how it is used.

\begin{description}
\item [Create Charms] This is the basic spell for creating Magic Items. Use this spell to create rune-inscribed swords, paper talismans that protect against sprits, amd dragon skin armour that is resistant to fire (via a Resist Fire spell).
\
\item [Create Power Point Store] If you want to create a magic item that has Power Points already stored, so the user doesn’t have to use their own, this is the spell to use.
\item [Potions] This is a quick way of making non-reusable spell stores where you’ve already spent the magic points, for you or your allies to gulp down for instant effect during a combat. Think Healing + Create Potions and you have the classic Healing Potion.
\end{description}

\subsection{Identify a Magic Item}
There is no catch all “Detect Magical Properties” or “Know Magic item” skill in Fantsy D100. This is quite deliberate, keeping with the general policy that such items are not the equivalent of Magical shotguns.  Some options are:

Consult a Sage or other magical expert. This option will cost the characters lots of money. Take a baseline of one hundred silvers per point of spell magnitude OR some perilous quest that the character must do in return. Such experts are rare, because most high ranking Magicians have little time for magical research for others, and would be more interested in their own schemes. 

Detect Magic spells. This merely tells you the item is magical.  A critical casting may tell the caster how powerful the item is.

Trial and error. The character tries to find out the item’s use by experiment. Allow creative and imaginative plans to reveal partially what the item does.

Researching the myths and legends around the item.  This is the most certain way of finding out what a magic item does. Of course such myths may be obscure themselves, requiring a dangerous Quest to a long hidden repository of knowledge to find.

\fi
