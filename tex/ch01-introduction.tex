\chapter{Introduction}
\label{ch:introduction}


\section{About Fantasy D100}

Fantasy D100 is a fantasy RPG game system. It is essentially my take of the great OpenQuest rules (based on the SRD version).

This has much more than my custom rules though. It has several additions and (what I believe are) cleaner mechanics that abstract any extraordinary or supernatural abilities (like magic, special fighting techniques, psionics, etc.) into character Powers. At the same time it stays true to OQ's rational of being simple, flexible but still relatively realistic.

Fantasy D100 system is pretty open, both at character generation and during character advancement, in that they don’t tie a character down to a predestined path of skill and powers. Everything is a potential choice that a Player can make during play.

Character generation produces characters that have skills in all the basic areas of expertise, a couple of speciality advanced skills, some powers (e.g. magic) and some skill in at least one or two weapons. Most Fantasy D100 characters start out being able to do most things, a skill area or two that they excel at, have a decent chance in a fight and even have some powers to even out the odds.

Because characters start off more rounded there is less of an issue about getting the right mix of skills for the group so it can survive the adventure.

Part I of this book goes through a basic ruleset without concetrating in the intricasies or philosophy of any particular campaign world. It represents everything a Gamemaster needs for a mundane medieval fantasy game (i.e. without magic, fantasy races etc.). 


\section{Disciplines and Powers}
The Disciplines chapters in part two contain several different extraordinary and supernatural options to embellish your campaign. Each chapter is self-contained including everything needed with regard to this discipline. 

A Discipline embodies a group of extraordinary or supernatural abilities called Powers. A character might be knowledgeable in several disciplines. Disciplines can vary from special combat techniques that require extensive training to the ability to channel mystical energies. Some Disciplines are:
\begin{rpg-list}
\item Battle: the character possesses extraordinary battle techniques.
\item Magic: the character possesses magic powers.
\item Arcane Magic: the character understands magic's deeper secrets.
\item Divine Magic: the character can channel power directly from the Gods.
\item Shamanism: the character is in tune with the Spirit World.
%\item Psionic: Talents related to raw personal power.
%\item Elementalists: Talents related to raw elemental power.
\end{rpg-list}

Which disciplines are available depends on the world that the game is being played in. Fantasy D100 provides a flexible system were adding or removing a Discipline is seamless to the rest of the rules. Each Discipline has its own specific rules and sometimes an appropriate skill that represents the character's knowledge and ability in the respective discipline which applies to all Powers of that Discipline.

Each Gamemaster can choose to use only the Disciplines appropriate to their campaign's philosophy.


\section{Gamemaster}
The third part of this book contains rules for the organiser of the game. The Gamemaster has a wide variety of additional rules and options to include in the campaign world. Some examples are rules:
\begin{rpg-list}
\item related to adventuring,
\item related to ships and sailing,
\item related to mass combat,
\item on how to include fantasy races, like Elves, Dwarves, etc.,
\item on how to include spell-like abilities to characters,
\item on how to create and list minor NPCs,
\item for more epic games, and several others.
\end{rpg-list}

In addition to that there is an extensive creature list to help populate your campaign world.

