\documentclass[10pt,a4paper,twocolumn,openany]{book}
%\documentclass[10pt,a4paper,openany]{book}

\usepackage[bg-a4]{lib/rpg-book} % Options: bg-a4, bg-letter, bg-full, bg-print, bg-none.
\usepackage[english]{babel}
%\usepackage[utf8]{inputenc}
\usepackage{lipsum}


% Start document
\begin{document}
\fontfamily{ppl}
\selectfont % Set text font
\frontmatter

\rpgMakeCover[
    image = img/cover,
    logo = img/logo,
    title = OpenQuest KK,
    subtitle = Created by Konstantinos Karasavvas \\\today\\based on OpenQuest SRD
]


\tableofcontents

% Your content goes here
\mainmatter
\part{Game Mechanics}

\chapter{Character Generation}

\section{Player Characters}
A character is your representation in the game.  Your eyes, ears, touch, feel and smell in the imaginary world that you and your fellow players create.

On one hand the character is a collection of numbers which describe his/her characteristics, skills and magic spells that are written down on a character sheet.  This chapter will explain how you create these numbers, in a process known as Character Generation. But that is only half of what a character is.

The other half exists mainly in the imagination of the player, with perhaps some quick notes on the character sheet. This half is the personality of the character and other intangibles such as goals and past history. These are the things that you can’t express in cold hard numbers, which really bring the character to life and give the player guidelines on how the character acts and thinks.


\begin{rpg-titlebox}{Group Balance and Survivability}
OpenQuest’s skill and magic systems are pretty open, both at character generation and during character advancement, in that they don’t tie a character down to a predestined path of skill and magic increases dictated by the type of character that the Player chooses during character generation.

Character generation produces characters that have skills in all the basic areas of expertise, a couple of speciality advanced skills, some starting personal magic and some skill in at least one or two weapons. Most OpenQuest characters start out being able to do most things, a skill area or two that they excel at, have a decent chance in a fight and have some magic to even out the odds. 

Because OpenQuest characters start off more rounded there is less of an issue about getting the right mix of skills for the group so it can survive the adventure.
\end{rpg-titlebox}


\section{Character Generation}
The process of creating a character is known as Character Generation. OpenQuest character generation is a seven step process and at each step the Player makes decisions about what their character is like at the beginning of the game, when the character is just starting out on their adventuring career. 

\subsection{Concept}
A character concept is a one sentence summing up of what the character is all about.

\begin{rpg-examplebox}
\begin{rpg-list}
\item Rurik is ``A determined and foolhardy warrior seeking excitement and adventure.''
\item Lura is ``A mysterious and elegant sorceress.''
\item Mancala is ``The illegitimate son of a murdered Noble, who survives through being a rogue.''
\item Abnon is ``A pious priest who smites evil and protects the innocent.''
\end{rpg-list}
\end{rpg-examplebox}

Having a clear concept of what you want your character to be like at the beginning of character generation guides the whole process as you make choices to generate the numbers that you will roll against during play.

For example,  for Rurik it states clearly that he is a warrior, therefore when choosing skills Rurik puts points into Dodge and Unarmed combat, both skills that will be highly useful when he gets into a fight, rather than any of the Lores.

You are of course free to change the concept as you generate the character. Generally, as a rule, the stronger the character concept, the easier it is to create an interesting character.

Your Games Master may ask you what your character concept is before you start Character Generation, to make sure that it fits in with the sort of game that he has prepared.  For example creating a warlike barbarian may not be a good idea for a game that is going to revolve around a series of magical mysteries where the characters will need strong investigative and magical skills.

Comparing concepts with the other players before diving into character creation is strongly recommended. Your character will be part of an adventuring group that is made up of the other Players’ characters. These characters work together, even if they don’t like each other, towards a common goal of solving the mysteries and dilemmas thrown up by the Games Master during the adventures that they play through. The game is unlikely to be any fun if all the players have similar or near identical concepts, as compared with a game where the group is made up of characters with different concepts that can work together to create interesting role-playing opportunities.


\subsubsection{Step 1: Determine Concept}
In one sentence sum up what your character is all about. Use the guidelines above to give yourself ideas. Ask the other Players what their character concepts are to make sure the group has an interesting selection of characters.

Check with your Games Master that your character concept fits in with the type of game that the group is going to be playing.


\section{Characteristics}
These are the primary building blocks of the character. All characters and creatures have seven characteristics, which give the basic information about the character’s physical, mental and spiritual capabilities.

As well as being useful indicators of how to roleplay the character (see below) they are the scores that skills are initially based upon. The characteristics are:

\textbf{Strength} (STR):  A character’s brute force, Strength affects the amount of damage he deals, what weapons he can wield effectively, how much he can lift and so on. 

Constitution (CON):  A measure of the character’s health, Constitution affects how much damage he can sustain in combat, as well as his general resistance to disease and other illnesses.

Dexterity (DEX):  A character’s agility, co-ordination and speed of reaction, Dexterity aids him in many physical actions, including combat. 

Size (SIZ):   This is an indication of the character’s mass and, like Strength and Constitution, can affect the amount of damage a character can deal and how well he can absorb damage.

Intelligence (INT):  A character’s ability to think around problems, analyse information and memorise instructions. It is a very useful Characteristic for characters interested in becoming accomplished spellcasters. 

Power (POW):  Perhaps the most abstract Characteristic, Power is a measure of the character’s life force and the strength of his willpower.

Charisma (CHA):  This quantifies a character’s attractiveness and leadership qualities. 

\subsection{Step 2: Generating Characteristics}
Test



\section{Some boxes}
\begin{rpg-titlebox}{title}
	rpg-titlebox
\end{rpg-titlebox}

\begin{rpg-commentbox}
	rpg-commentbox
\end{rpg-commentbox}

\begin{rpg-warnbox}
	you can add some warnings using this box
\end{rpg-warnbox}

\begin{rpg-suggestionbox}
	rgp-suggestionbox
\end{rpg-suggestionbox}

\begin{rpg-quotebox}
    rpg-quotebox
\end{rpg-quotebox}

\begin{rpg-examplebox}
	rgp-examplebox
\end{rpg-examplebox}

%\newpage % Acts as columbreak because of twocolumn option; for pagebreak use \clearpage

\section{Some tables}
\header{default rpg-table (2 column)}
\begin{rpg-table}
   	\textbf{Table head 1}  & \textbf{Table head 2} \\
   	Some value  & Some value \\
   	Some value  & Some value \\
   	Some value  & Some value
\end{rpg-table}

% For more columns, you can say \begin{rpg-table}[your options here].
% For instance, if you wanted three columns, you could say
% \begin{rpg-table}[XXX]. The usual host of tabular parameters are
% aailable as well.
\header{rpg-table with more columns}
\begin{rpg-table}[XXX]
    \textbf{Table head 1}  & \textbf{Table head 2} & \textbf{Table head 3}\\
   	Some value  & Some value & Some value\\
   	Some value  & Some value & Some value\\
   	Some value  & Some value & Some value
\end{rpg-table}

\header{default rpg-table2 (2 column)}
\begin{rpg-table2}
   	\textbf{Table head 1}  & \textbf{Table head 2} \\
   	Some value  & Some value \\
   	Some value  & Some value \\
   	Some value  & Some value
\end{rpg-table2}

\header{rpg-longtable}
\tablehead{%
    \hline
    \textbf{first column} & \textbf{second column} & \textbf{third column} \\
    \hline
}
\begin{rpg-longtable}[ccc]
    1 & 2 & 3 \\ 1 & 2 & 3 \\ 1 & 2 & 3 \\ 1 & 2 & 3 \\
    1 & 2 & 3 \\ 1 & 2 & 3 \\ 1 & 2 & 3 \\ 1 & 2 & 3 \\
    1 & 2 & 3 \\ 1 & 2 & 3 \\ 1 & 2 & 3 \\ 1 & 2 & 3 \\
    1 & 2 & 3 \\ 1 & 2 & 3 \\ 1 & 2 & 3 \\ 1 & 2 & 3 \\
    1 & 2 & 3 \\ 1 & 2 & 3 \\ 1 & 2 & 3 \\ 1 & 2 & 3 \\
    1 & 2 & 3 \\ 1 & 2 & 3 \\ 1 & 2 & 3 \\ 1 & 2 & 3 \\
    1 & 2 & 3 \\ 1 & 2 & 3 \\ 1 & 2 & 3 \\ 1 & 2 & 3 \\
    1 & 2 & 3 \\ 1 & 2 & 3 \\ 1 & 2 & 3 \\ 1 & 2 & 3 \\
    1 & 2 & 3 \\ 1 & 2 & 3 \\ 1 & 2 & 3 \\ 1 & 2 & 3 \\
    1 & 2 & 3 \\ 1 & 2 & 3 \\ 1 & 2 & 3 \\ 1 & 2 & 3 \\
    1 & 2 & 3 \\ 1 & 2 & 3 \\ 1 & 2 & 3 \\ 1 & 2 & 3 \\
    1 & 2 & 3 \\ 1 & 2 & 3 \\ 1 & 2 & 3 \\ 1 & 2 & 3 \\
    1 & 2 & 3 \\ 1 & 2 & 3 \\ 1 & 2 & 3 \\ 1 & 2 & 3 \\
    1 & 2 & 3 \\ 1 & 2 & 3 \\ 1 & 2 & 3 \\ 1 & 2 & 3 \\
    1 & 2 & 3 \\ 1 & 2 & 3 \\ 1 & 2 & 3 \\ 1 & 2 & 3 \\
\end{rpg-longtable}

\section{List}
\begin{rpg-list}
    \item first list item
    \item second list item
\end{rpg-list}


\section{Monster}
% You can optionally not include the background by saying
%\begin{rpg-monsterboxnobg}{monsterboxnob}
\begin{monsterbox}{rpg-monsterbox}
	\textit{Small metasyntatic variable (golbinoid), neutral evil}\\
	\rpghline
	\basics[%
	armorclass = 12,
    hitpoints  = \rpgdice{3d8 + 3},
	speed      = 50 t
	]
	\rpghline%
	\stats[ % This stat command will autocomplete the modifier for you
    STR = 12, 
    DEX = 7
	]
	\rpghline%
	\details[%
	% If you want to use commas in these sections, enclose the
	% description in braces.
	% I'm so sorry.
	languages = {Common Lisp, Erlang},
	]
	\rpghline%
	\begin{rpg-monsteraction}[rpg-monsteraction]
		This Monster has some serious superpowers!
	\end{rpg-monsteraction}

	\rpgmonstersection{rpgmonstersection}
	\begin{rpg-monsteraction}[rpg-monsteraction]
		This one can generate tremendous amounts of text! Though only when it wants to.
	\end{rpg-monsteraction}

	\begin{rpg-monsteraction}[rpg-monsteraction]
    See, here he goes again! Yet more text.
	\end{rpg-monsteraction}
\end{monsterbox}


\section{Spell}
\begin{rpg-spell}
	{Beautiful Typesetting}
	{4th-level illusion}
	{1 action}
	{5 feet}
	{S, M (ink and parchment, which the spell consumes)}
	{Until dispelled}
	You are able to transform a written message of any length into a beautiful scroll. All creatures within range that can see the scroll must make a wisdom saving throw or be charmed by you until the spell ends.

	While the creature is charmed by you, they cannot take their eyes off the scroll and cannot willingly move away from the scroll. Also, the targets can make a wisdom saving throw at the end of each of their turns. On a success, they are no longer charmed.
\end{rpg-spell}

\section{Arts}

\rpgart{t}{img/art-top}
\rpgart{b}{img/art-bottom}

\lipsum
\lipsum

\part{Part name2}
\chapter{Colors}

This package provides several global color variables to style \lstinline!rpg-commentbox!, \lstinline!rpg-quotebox!, \lstinline!rpg-examplebox!, and \lstinline!rpg-table! environments.


TODO

%\begin{rpgtable}[lX]
%  \textbf{Color}         & \textbf{Description} \\
%  \lstinline!commentboxcolor! & Controls \lstinline!commentbox! background. \\
%  \lstinline!paperboxcolor!   & Controls \lstinline!paperbox! background. \\
%  \lstinline!quoteboxcolor!   & Controls \lstinline!quotebox! background. \\
%  \lstinline!tablecolor!      & Controls background of even \lstinline!rpgtable! rows. \\
%\end{rpgtable}
%
%See Table~\ref{tab:colors} for a list of accent colors that match the core books.
%
%\begin{table*}
%  \begin{rpgtable}[XX]
%    \textbf{Color}                            & \textbf{Description} \\
%    \lstinline!commentgreen!                      & Light green used in PHB Part 1 \\
%    \lstinline!commentblue!                       & Light cyan used in PHB Part 2 \\
%    \lstinline!PhbMauve!                           & Pale purple used in PHB Part 3 \\
%    \lstinline!PhbTan!                             & Light brown used in PHB appendix \\
%    \lstinline!DmgLavender!                        & Pale purple used in DMG Part 1 \\
%    \lstinline!DmgCoral!                           & Orange-pink used in DMG Part 2 \\
%    \lstinline!DmgSlateGray! (\lstinline!DmgSlateGrey!) & Blue-gray used in PHB Part 3 \\
%    \lstinline!DmgLilac!                           & Purple-gray used in DMG appendix \\
%  \end{rpgtable}
%  \caption{Colors supported by this package}%
%  \label{tab:colors}
%\end{table*}
%
%\begin{itemize}
%  \item Use \lstinline!\setthemecolor[<color>]! to set \lstinline!themecolor!, \lstinline!commentcolor!, \lstinline!paperboxcolor!, and \lstinline!tablecolor! to a specific color.
%  \item Calling \lstinline!\setthemecolor! without an argument sets those colors to the current \lstinline!themecolor!.
%  \item \lstinline!commentbox!, \lstinline!rpgtable!, \lstinline!paperbox!, and \lstinline!quoteboxcolor! also accept an optional color argument to set the color for a single instance.
%\end{itemize}
%
%\subsection{Examples}
%
%\subsubsection{Using \lstinline!themecolor!}
%
%\begin{lstlisting}
%\setthemecolor[PhbMauve]
%
%\begin{paperbox}{Example}
%  \lipsum[2]
%\end{paperbox}
%
%\setthemecolor[PhbLightCyan]
%
%\header{Example}
%\begin{rpg-table}[cX]
%  \textbf{d8} & \textbf{Item} \\
%  1           & Small wooden button \\
%  2           & Red feather \\
%  3           & Human tooth \\
%  4           & Vial of green liquid \\
%  6           & Tasty biscuit \\
%  7           & Broken axe handle \\
%  8           & Tarnished silver locket \\
%\end{rpg-table}
%\end{lstlisting}
%
%\begingroup
%\setthemecolor[PhbMauve]
%
%\begin{paperbox}{Example}
%  \lipsum[2]
%\end{paperbox}
%
%\setthemecolor[PhbLightCyan]
%
%\header{Example}
%\begin{rpgtable}[cX]
%  \textbf{d8} & \textbf{Item} \\
%  1           & Small wooden button \\
%  2           & Red feather \\
%  3           & Human tooth \\
%  4           & Vial of green liquid \\
%  6           & Tasty biscuit \\
%  7           & Broken axe handle \\
%  8           & Tarnished silver locket \\
%\end{rpgtable}
%\endgroup
%
%\subsubsection{Using element color arguments}
%
%\begin{lstlisting}
%\begin{rpgtable}[cX][DmgCoral]
%  \textbf{d8} & \textbf{Item} \\
%  1           & Small wooden button \\
%  2           & Red feather \\
%  3           & Human tooth \\
%  4           & Vial of green liquid \\
%  6           & Tasty biscuit \\
%  7           & Broken axe handle \\
%  8           & Tarnished silver locket \\
%\end{rpgtable}
%\end{lstlisting}
%
%\begin{rpgtable}[cX][DmgCoral]
%  \textbf{d8} & \textbf{Item} \\
%  1           & Small wooden button \\
%  2           & Red feather \\
%  3           & Human tooth \\
%  4           & Vial of green liquid \\
%  6           & Tasty biscuit \\
%  7           & Broken axe handle \\
%  8           & Tarnished silver locket \\
%\end{rpgtable}



% End document
\end{document}
